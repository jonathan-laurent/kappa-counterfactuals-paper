\section{Practical use of  counterfactual resimulation}
\label{sec:cosim-practice}

Our approach raises some practical questions. How do we know that a
causal narrative is incomplete or, equivalently, which counterfactual
experiments are worth trying? One way to proceed is by developing
heuristics, such as identifying correlations between events in samples
of factual traces (or in a long trace). For example, in our toy model,
the occurrence of $p$ is often preceded by the occurrence of
$pk$. This correlation, together with the absence of $pk$ from the
initial causal account based on enablement
(Figure~\ref{fig:dumb-story}), suggests to try a counterfactual
experiment on $pk$. More generally, if
\begin{inparaenum}[(i)]
\item a context $\mathcal C$ in which an event $e$ occurs is
  frequently more specific than is required by the left-hand side of
  the underlying rule and
\item this observation cannot be explained by the current causal
  narrative,
\end{inparaenum}
then a counterfactual experiment in which we block the last event
responsible for at least part of $\mathcal C$ seems worthwhile in
order to assess whether the current causal narrative needs to be
updated. In our toy example we are led to evaluate whether
$\mathbf{P}(\CFST{})$ with $\BLOCKED{\iota}{e'} = (e = pk)$ and
$\psi[\tau'] = (p \in \tau')$. However, this choice of $\iota$ and
$\psi$ may not be useful if rule $pk$ has high activity. Blocking a
single event occurrence might not have much of an effect, as the same
rule would immediately trigger again, cancelling the effect of the
intervention.  In this case, $\iota$ could consist in blocking every
instance of the potential event associated with $pk$, i.e.\@
``knocking out" rule $pk$, in a defined timeframe.
