% -*- TeX-master: "ijcai18.tex" -*-

\section{Counterfactuals And Prevention}\label{sec:inhibition}

% \ifshort

% The diagram shown Figure~\ref{fig:cex} extends the one in
% Figure~\ref{fig:dumb-story} and explains the counterfactual dependency
% between $pk$ and $p$ in \RefTrace{}. The dotted node corresponds to a
% \emph{counterfactual event}, which is absent from \RefTrace{}. It is
% related to \emph{factual} events by prevention arrows, shown in
% red. These arrows can be read as follows: ``$pk$ prevents $u$, which
% prevents $p$''. In this section, we give a rigorous account of
% prevention by connecting events in a factual trace to events in a
% cognate counterfactual trace as produced by counterfactual
% resimulation, and vice versa.

% \else

The diagram shown Figure~\ref{fig:cex} extends the one in
Figure~\ref{fig:dumb-story} and explains the counterfactual dependency
between $pk$ and $p$ in \RefTrace{}. The dotted node corresponds to a
\emph{counterfactual event}, which is absent from \RefTrace{}. It is
related to \emph{factual} events by prevention arrows, shown in
red. These arrows can be read as follows: ``$pk$ prevents $u$, which
prevents $p$''. In this section, we give a precise semantics to such
diagrams and discuss how they can be generated systematically.

As discussed in section~\ref{subsec:dumb-story}, a causal narrative like
in Figure~\ref{fig:dumb-story} results from a trace after causal
compression.  Its events are organized in a directed acyclic graphs whose edges
are enablement arrows. While enablement is straightforward to define, prevention
is trickier because it relates events that happened to events that did not. Our
insight is to use counterfactual traces to define prevention as connecting
events from a factual trace to events in a cognate counterfactual trace or vice
versa.

%\fi

\ifshort \else \medskip \fi

\emph{Notation and diction.} We use
the symbol $e$, which in previous sections referred to an event
template, to directly denote an event. Moreover, when we say that an
event \emph{tests} or \emph{modifies} a site, we really mean the tests
and actions involved when matching and rewriting, respectively, the
underlying rule in the mixture $\TSTATE{t}{\tau}$. For example, event
$p$ in \RefTrace{} tests three sites and modifies one. Finally, an
event $e$ occuring at time $t$ is said to be \emph{executable} in
trace $\tau$ if the associated template is realizable in mixture
$\TSTATE{t}{\tau}$.

\subsection{Prevention in Counterfactual Experiments}
\label{subsec:cex}\label{subsec:inhibition}

A \textit{counterfactual experiment} is a triple
$(\tau, \iota, \tau')$ for which there exists a schedule $\sigma$ such
that $\tau = T(\sigma)$ and $\tau' = \ATRAJ{}(\sigma)$. Such triples
are produced by counterfactual resimulation.

To formalize enablement and prevention, we need to define some terms
used in Kappa to talk about events. Without loss of generality, we
shall assume that a site on an agent carries either binding state or
tagging state but not both. The \textit{value} of a tagged site in a
mixture is its current tag and the value of a binding site is either
$\textsc{free}$ or $\textsc{bound-to}(s)$, where $s$ identifies
another site in the mixture.

\begin{figure}
  \vspace*{-0.5cm}
  \begin{center}
    \includegraphics[scale= \ifshort 0.65 \else 0.7 \fi]{figures/dot/cex.pdf}
  \end{center}
  \vspace{-0.8cm}
  \caption{A graphical explanation of the counterfactual dependency
    between $pk$ and $p$ in \protect\RefTrace{}, in
    terms of enablement and prevention arrows. It is based on the
    (compressed) counterfactual experiment $(\tau, \iota, \tau')$ where
    $\tau = (pk, b, p)$, $\iota$ blocks $pk$ and $\tau' = (b, u)$.}
  \label{fig:cex}
\end{figure}


\begin{definition}[Enablement]
  Let $\tau$ be a trace and $e, e' \in \tau$ two events.  We say that
  $e$ enables $e'$ if $e$ is the last event before $e'$ that modifies
  some site to the value it is tested for by $e'$.
\end{definition}

\begin{definition}[Prevention]
  Let $(\tau, \iota, \tau')$ be a counterfactual experiment. An event
  $e$ that occurs at time $t$ in $\tau$ is said to prevent an event
  $e'$ that occurs at time $t'$ in $\tau'$ if all of the following
  hold:
  \begin{inparaenum}[(1)]
  \item \label{inhibition:time} $t < t'$ ;
  \item \label{inhibition:breaks} there exists a site $s$ such that
    $e$ is the last event in $\tau$ before $t'$ that modifies the
    value of $s$ away from what $e'$ tests it for ;
  \item \label{inhibition:nointf} there are no events in $\tau'$ that
    modify $s$ during the time interval $(t, t')$.
  \end{inparaenum}
  The same definition holds switching $\tau$ and $\tau'$.
\end{definition}

Counterfactual experiments can be represented as directed acyclic
graphs like the one in Figure~\ref{fig:cex}. Such a graph features
three kinds of nodes:
\begin{inparaenum}[]
\item events that are proper to the factual trace (thick solid nodes),
\item events that are proper to the counterfactual trace (dotted
  nodes) and
\item events that are common to both traces (thin solid nodes).
\end{inparaenum}
% Edges consist in enablement and prevention arrows.

As illustrated Figure~\ref{fig:cex}, the influence of $pk$ on $p$ in
our example is mediated by the counterfactual event $u$. Such
mediating events always exist, as stated by the following theorem.

\begin{theorem}[Completeness of enablement and prevention]
  \label{thm:completeness}
  Let $(\tau, \iota, \tau')$ be a counterfactual experiment and $e$ an
  event that belongs only to $\tau$. Then, there exists an event
  $\hat e \in \tau$ that is blocked by $\iota$ and such that there is a directed
  path from $\hat e$ to $e$ with an even number of prevention arrows.
\end{theorem}
\noindent This theorem states that counterfactual dependencies can always be
explained in terms of enablement and prevention relations between individual
events. A proof is in Appendix~B.
%\ref{ap:completeness}. 
This result establishes a
bridge between two different visions of causality: the vision dominant in the
concurrency community, in which causality is defined predominantly in terms of
enablement, in opposition to concurrency  \cite{winskel1986event}, and the
vision based on counterfactuals, which is dominant in the causal inference
community \cite{pearl2009causality}.

\subsection{Compression of Counterfactual Experiments}

Counterfactual experiments produced with counterfactual resimulation
are usually very large. They typically feature a lot of redundancy,
including events that are irrelevant to the outcome of interest or
futile cycles as discussed in section~\ref{subsec:dumb-story}. A
compression step is usually necessary before concise causal narratives
can be extracted from such experiments. However, the two traces of a
counterfactual experiment cannot be compressed separately following
the procedure described in section~\ref{subsec:dumb-story}, as there
is no guarantee that the compressed traces can still be generated from
a unique schedule to form a valid counterfactual experiment. Instead,
compressing a counterfactual experiment consists in extracting a
minimal \emph{valid} sub-experiment.

A counterfactual experiment $(\tau_1, \iota, \tau_1')$ is said to be a
\emph{sub-experiment} of $(\tau_2, \iota, \tau_2')$ if $\tau_1$ is a
sub-trace of $\tau_2$ and $\tau_1'$ is a sub-trace of $\tau_2'$. Also,
valid counterfactual experiments can be characterized as follows.

\begin{proposition}%[Characterization of counterfactual experiments]
  \label{prop:valid-cex}
  A triple $(\tau, \iota, \tau')$ is a valid counterfactual experiment
  if and only if all of the following hold:
  \begin{inparaenum}[(1)]
  \item \label{valid-cex:valid-traces} both $\tau$ and $\tau'$ are
    valid traces ;
  \item \label{valid-cex:no-blocking} no event of $\tau'$ is blocked
    by $\iota$ ;
  \item \label{valid-cex:co-occur} for every event $e \in \tau$ such
    that $e \notin \tau'$, then either $e$ is not executable in
    $\tau'$ or $e$ is blocked by $\iota$ ;
  \item \label{valid-cex:co-occur2} for every event $e' \in \tau'$
    such that $e' \notin \tau$, then $e'$ is not executable in $\tau$.
  \end{inparaenum}
\end{proposition}

\noindent Compressing a counterfactual experiment consists in finding
a minimal valid sub-experiment such that
\begin{inparaenum}[(i)]
\item the outcome of interest appears in the factual trace but not in
  the counterfactual trace and
\item events that are blocked by $\iota$, or on which the outcome of
  interest was shown to be counterfactually dependent in previous
  analyses, are kept in the factual trace.
\end{inparaenum}
Because these constraints along with the properties featured in
Proposition~\ref{prop:valid-cex} can be encoded as boolean
satisfiability constraints, compressing a counterfactual experiment
can be done using standard \textsc{sat}-solving techniques.

