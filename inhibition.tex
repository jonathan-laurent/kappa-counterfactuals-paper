% -*- TeX-master: "ijcai18.tex" -*-

\section{Counterfactuals and prevention}\label{sec:inhibition}

The diagram shown Figure~\ref{fig:cex} extends the one in
Figure~\ref{fig:dumb-story} and explains the counterfactual dependency
between $pk$ and $p$ in \RefTrace{}. The dotted node corresponds to a
\emph{counterfactual event}, which is absent from \RefTrace{}. It is
related to \emph{factual} events by prevention arrows, which are
displayed in red. These arrows can be read as follows: ``$pk$ prevents
$u$, which would have prevented $p$''.  In this section, we give a
precise semantics to such diagrams and discuss how they can be
generated systematically.

As discussed in section~\ref{subsec:dumb-story}, a causal narrative
like the one in Figure~\ref{fig:dumb-story} is generated from a trace,
which usually results from a compression process, and whose events are
organized in a directed acyclic graphs, edges being enablement arrows.
Enablement arrows are easy to define in the context of a trace, and we
give a formal definition in section~\ref{subsec:inhibition}.
Prevention arrows are trickier because they must relate events that
happened to events that did not. For this reason, previous
work~\cite{DanosEtAl-CONCUR07} fails to provide a proper account of
them.  Our insight is to use counterfactual traces to give a rigorous
account of prevention using arrows that connect events from a factual
trace to events in an associated counterfactual trace and vice versa.

\vspace{-0.3cm}

\paragraph{Remark on notation} Although we used the letter $e$ to
denote potential events in previous sections, we use it to denote
\emph{events} in this section. We write $\TIME{e}$ the time
of occurence of event $e$. Finally, an event $e$ occuring at time $t$
is said to be executable in trace $\tau$ if the associated potential
event is executable in mixture $\TSTATE{t}{\tau}$.


\subsection{Prevention in  counterfactual experiments}
\label{subsec:cex}\label{subsec:inhibition}

A \textit{counterfactual experiment} is obtained by pairing a
\emph{factual} trace to a counterfactual counterpart. Formally, a
counterfactual experiment is a triple $(\tau, \iota, \tau')$ for which
there exists a schedule $\sigma$ such that $\tau = T(\sigma)$ and
$\tau' = \ATRAJ{}(\sigma)$. Such triples are produced by
counterfactual resimulation.

Before we formalize enablement and prevention in the context of a
counterfactual experiment, it is useful to define some vocabulary
about what it means for an event to \emph{test} or \emph{modify} a
site. Without loss of generality, we shall assume that sites on an
agent can either be binding sites or tagged sites (sites carrying an
internal state) but cannot have both functions. The \textit{value} of
a tagged site in a mixture is defined as its current tag and the value
of a binding site is either $\textsc{free}$ or $\textsc{bound-to}(s)$,
where $s$ identifies another site in the mixture.  An event is said to
\emph{modify} a site if it changes its value when executed. It is said
to \emph{test} a site $s$ to value $v$ if the left-hand side of the
associated rule requires $s$ to have value $v$ for it to be
executable. For example, event $p$ in \RefTrace{} tests three sites
and modifies one. We can now proceed in defining enablement and
prevention.

\begin{figure}
  \vspace*{-0.5cm}
  \begin{center}
    \includegraphics[scale= \ifshort 0.65 \else 0.7 \fi]{figures/dot/cex.pdf}
  \end{center}
  \vspace{-0.8cm}
  \caption{A graphical explanation of the counterfactual dependency
    between $pk$ and $p$ in \protect\RefTrace{}, in
    terms of enablement and prevention arrows. It is based on the
    (compressed) counterfactual experiment $(\tau, \iota, \tau')$ where
    $\tau = (pk, b, p)$, $\iota$ blocks $pk$ and $\tau' = (b, u)$.}
  \label{fig:cex}
\end{figure}


\begin{definition}[Enablement]
  Let $\tau$ a trace and $e, e'$ two events in $\tau$.  We say that
  $e$ enables $e'$ if $e$ is the last event before $e'$ that modifies
  some site to the value it is tested for by $e'$.
\end{definition}

\begin{definition}[Prevention]
  Let $(\tau, \iota, \tau')$ a counterfactual experiment. An event $e$
  that occurs at time $t$ in $\tau$ is said to prevent an event $e'$
  that occurs at time $t'$ in $\tau'$ if all of the following hold:
  \begin{inparaenum}[(1)]
  \item \label{inhibition:time} $t < t'$ ;
  \item \label{inhibition:breaks} there exists a site $s$ such that
    $e$ is the last event in $\tau$ before $t'$ that modifies the
    value of $s$ away from what $e'$ tests it for ;
  \item \label{inhibition:nointf} there are no events in $\tau'$ that
    modify $s$ during the time interval $(t, t')$.
  \end{inparaenum}
  The same definition holds switching $\tau$ and $\tau'$.
\end{definition}

Counterfactual experiments can be represented as directed acyclic
graphs like the one in Figure~\ref{fig:cex}. Such a graph features
three kinds of nodes:
\begin{inparaenum}[]
\item events that are proper to the factual trace (thick solid nodes),
\item events that are proper to the counterfactual trace (dotted
  nodes) and
\item events that are common to both traces (thin solid nodes).
\end{inparaenum}
% Edges consist in enablement and prevention arrows.


As illustrated Figure~\ref{fig:cex}, the influence of $pk$ on $p$ in
our example is mediated by the counterfactual event $u$. Such
mediating events always exist, as stated by the following theorem.

\begin{theorem}[Completeness of enablement and prevention]
  \label{thm:completeness}
  Let $(\tau, \iota, \tau')$ be a counterfactual experiment and $e$ an
  event that belongs only to $\tau$. Then, there exists an event
  $\hat e \in \tau$ that is blocked by $\iota$ and there is a directed
  path from $\hat e$ to $e$ with an even number of prevention arrows.
\end{theorem}
\noindent This theorem states that counterfactual dependencies can
always be explained in terms of enablement and prevention relations
between individual events. A proof is provided in
Appendix~\ref{ap:completeness}. In our opinion, this is a powerful
result, which establishes a bridge between two different visions of
causality: the vision based on event structures
\cite{winskel1986event} that is dominant in the concurrency community
and the vision based on counterfactuals which is dominant in the
causal inference community \cite{pearl2009causality}.

\subsection{Compression of counterfactual experiments}

Counterfactual experiments that are produced using counterfactual
resimulation are usually very large. They typically feature a lot of
redundancy, including events that are irrelevant to the outcome of
interest or futile cycles as discussed in
section~\ref{subsec:dumb-story}. Therefore, a compression step is
usually necessary before concise causal narratives can be extracted
from them. The two traces composing a counterfactual experiment cannot
be compressed separately following the procedure described in
section~\ref{subsec:dumb-story} though. Indeed, once compressed
separately, there is no guarantee that these two traces can still be
generated from a unique schedule and therefore form a valid
counterfactual experiment. Instead, compressing a
counterfactual experiment consists in extracting a minimal \emph{valid}
sub-experiment.


A counterfactual experiment $(\tau_1, \iota, \tau_1')$ is said to be a
\emph{sub-experiment} of $(\tau_2, \iota, \tau_2')$ if $\tau_1$ is a
sub-trace of $\tau_2$ and $\tau_1'$ is a sub-trace of $\tau_2'$. Also,
valid counterfactual experiments can be characterized as follows.

\begin{proposition}%[Characterization of counterfactual experiments]
  \label{prop:valid-cex}
  A triple $(\tau, \iota, \tau')$ is a valid counterfactual experiment if
  and only if all of the following hold:
  \begin{inparaenum}[(1)]
  \item \label{valid-cex:valid-traces} both $\tau$ and $\tau'$ are
    valid traces ;
  \item \label{valid-cex:no-blocking} no event of $\tau'$ is blocked
    by $\iota$ ;
  \item \label{valid-cex:co-occur} for every event $e \in \tau$
    such that $e \notin \tau'$, then either $e$ is not
    executable in $\tau'$ or $e$ is blocked by
    $\iota$ ;
  \item \label{valid-cex:co-occur2} for every event $e' \in \tau'$
    such that $e' \notin \tau$, then $e'$ is not triggerable in
    $\tau$.
  \end{inparaenum}
\end{proposition}

\noindent Then, compressing a counterfactual experiment consists
in finding a minimal valid sub-experiment such that
\begin{inparaenum}[(i)]
\item the outcome of interest appears in the factual trace but not in
the counterfactual trace and
\item events that are blocked by $\iota$ or on which the outcome of
  interest was shown to be counterfactually dependent in previous
  analyses are kept in the factual trace.
\end{inparaenum}
Because these constraints along with the properties featured in
Proposition~\ref{prop:valid-cex} can be encoded as boolean
satisfiability constraints, then compressing a counterfactual
experiment can be done using standard \textsc{sat}-solving techniques.

