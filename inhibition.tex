\section{Inhibition arrows}\label{sec:inhibition}

Returning to our example, we can use co-simulation to quantify the
influence of $pk$ on $p$ by estimating the probability of $p$
happening had $pk$ not occurred. However, we can go further by using
counterfactual traces to \textit{explain} this influence using both
activation and inhibition arrows (Figure~\ref{fig:cex}).

Activation arrows are easy to define and identify in a trace. We say
that an event $e$ activates $e'$ if $e$ is the last event before $e'$
that modifies some site to the value it is tested for by
$e'$. Inhibition arrows are trickier because they must relate events
that happened to events that did not. We use counterfactual traces to
give a rigorous account of inhibition using arrows that connect events
from the factual trace to events in the counterfactual trace and vice
versa.

A \textit{counterfactual experiment} is any triple
$(\tau, \iota, \tau')$ for which there exists a random realization
$\omega$ such that $\tau = T(\omega)$ and $\tau' =
\ATRAJ{}(\omega)$. Such triples are produced by co-simulation.

\begin{proposition}%[Characterization of counterfactual experiments]
  \label{prop:valid-cex}
  A triple $(\tau, \iota, \tau')$ is a counterfactual experiment if
  and only if all of the following hold:
  \begin{enumerate}[leftmargin=1.3cm, label=\textbf{VC\arabic*.}]
  \item \label{valid-cex:valid-traces} Both $\tau$ and $\tau'$ are
    valid traces
  \item \label{valid-cex:no-blocking} No event of $\tau'$ is blocked
    by $\iota$
  \item \label{valid-cex:co-occur} For every event $(e, t) \in \tau$
    such that $(e, t) \notin \tau'$, then either $e$ is not
    triggerable in $\TSTATE{t}{\tau'}$ or $(e, t)$ is blocked by
    $\iota$.  Besides, for every event $(e', t') \in \tau'$ such that
    $(e', t') \notin \tau$, then $e'$ is not triggerable in
    $\TSTATE{t}{\tau}$.
  \end{enumerate}
\end{proposition}


\begin{definition}
  Let $(\tau, \iota, \tau')$ a counterfactual experiment. An
  event $e$ that occurs at time $t$ in $\tau$ is said to inhibit an
  event $e'$ that occurs at time $t'$ in $\tau'$ if all of the
  following hold:
  \begin{enumerate}[leftmargin=1.2cm, label=\textbf{IC\arabic*.}]
  \item \label{inhibition:time} $t < t'$
  \item \label{inhibition:breaks} there exists a site $s$ such that
    $e$ is the last event in $\tau$ before $t'$ that modifies the
    value of $s$ away from what $e'$ tests it for
  \item \label{inhibition:nointf} there are no events in $\tau'$ that
    modify $s$ during the time interval $(t, t')$.
  \end{enumerate}
  The same definiton holds switching $\tau$ and $\tau'$.
\end{definition}



\begin{figure}
  \vspace*{-0.5cm}
  \begin{center}
    \includegraphics[scale= \ifshort 0.65 \else 0.7 \fi]{figures/dot/cex.pdf}
  \end{center}
  \vspace{-0.8cm}
  \caption{A graphical explanation of the counterfactual dependency
    between $pk$ and $p$ in \protect\RefTrace{}, in
    terms of enablement and prevention arrows. It is based on the
    (compressed) counterfactual experiment $(\tau, \iota, \tau')$ where
    $\tau = (pk, b, p)$, $\iota$ blocks $pk$ and $\tau' = (b, u)$.}
  \label{fig:cex}
\end{figure}


Figure~\ref{fig:cex} shows the influence of $pk$ on $p$ based on a
counterfactual experiment. Dotted nodes correspond to events proper to
the counterfactual trace $\tau'$, thick nodes to events proper to the
factual trace $\tau$, and the remaining nodes correspond to events
common to both traces. Activation arrows are depicted in black and
inhibition arrows in red.

The example illustrates the influence of $pk$ on $p$ mediated by the
counterfactual event $u$. Such mediating events always exist, as
stated by the following theorem.  %In general, we can always %exhibit
a sequence of intermediate events connected by activation and
inhibitio%n %arrows to relate an event that is proper to the factual
trace to another %one that is blocked by the intervention $\iota$.

\begin{theorem} Let $(\tau, \iota, \tau')$ be a counterfactual
  experiment and $e$ an event that belongs only to $\tau$. Then there
  exists an event $e_0 \in \tau$ that is blocked by $\iota$ and there
  is a path from $e_0$ to $e$ with an even number of inhibition
  arrows.
\end{theorem}

\subsection*{Proof of completeness}

It is convenient to prove the following result instead, which is
slightly more general. In the proof, we write $\TIME{e}$ the time of
occurence of event $e$.
\begin{theorem*} Let $(\tau, \iota, \tau')$ be a counterfactual
  experiment. If $e$ is an event that belongs only to $\tau$
  or $\tau'$, then there exists an event $\hat e$ in
  $\tau$ that is blocked by $\iota$ and there is a directed path 
  from $\hat e$ to $e$.
\end{theorem*}
\begin{proof}
  We prove this theorem by induction on the number of events before
  $e$ in both $\tau$ and $\tau'$. Let's consider $e \in \tau$ such
  that $e \notin \tau'$. If $\BLOCKED{\iota}{e}$ is true, then we are
  done. Otherwise, by item ($\ref{valid-cex:co-occur}$) of
  Proposition~\ref{prop:valid-cex}, $e$ is not executable in
  $\tau'$. Therefore, there exists a site $s$ such that event $e$
  tests $s$ to a different value than it has in $\TSTATE{t}{\tau'}$.
  Let $t = \TIME{e}$.  Let's define $e_0$ the last event in $\tau$
  modifying $s$ that occurs strictly before time $t$, and $e_0'$ the
  last event in $\tau'$ modifying $s$ that occurs strictly before time
  $t$. These events modify $s$ to different values so they cannot be
  the same. Therefore, we have $\TIME{e_0} \neq \TIME{e_0'}$ with
  probability one.

  \begin{itemize}
  \item If $\TIME{e_0} < \TIME{e_0'}$, then we have
    $e_0 \notin \tau'$ and so we can apply the induction hypothesis on
    $e_0$. As a consequence, there exists a path from an event
    $\hat e$ in $\tau$ which is blocked by $\iota$ to $e_0$. Besides,
    $e_0$ enables $e$. Therefore, there is a path between $\hat e$
    and $e$.
  \item If $\TIME{e_0'} < \TIME{e_0}$, then we have
    $e'_0 \notin \tau$ and so we can apply the induction
    hypothesis on $e'_0$. As a consequence, there exists a path from
    an event $\hat e$ in $\tau$ which is blocked by $\iota$ to
    $e'_0$. Besides, $e_0'$ prevents $e$. Therefore, there is a path
    between $e$ and $\hat e$.
  \end{itemize}
  The same proof holds for when $e \in \tau'$ and $e \notin \tau$,
  using item~($\ref{valid-cex:co-occur2}$) of
  Proposition~\ref{prop:valid-cex} instead of
  item~($\ref{valid-cex:co-occur}$).
\end{proof}
This result implies Theorem~\ref{thm:completeness}. In particular,
\emph{any} directed path from $\tau$ to $\tau$ necessarily contains an
even number of prevention arrows as these arrows always go from
$\tau$ to $\tau'$ or the other way around.
