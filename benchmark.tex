% -*- TeX-master: "ijcai18.tex" -*-

\newcommand{\subs}[2]{#1_{\textsf{#2}}}

We assessed the performance of our implementation of counterfactual
resimulation using the toy model of Figure~\ref{fig:model}, but with an initial
mixture consisting of a large number of kinase and substrate instances. Although
such a simple model is not of biological significance, it is adequate for an
initial assessment of performance.

\subsection{Experimental protocol}

%In this benchmark, whose results are summarized in Table~\ref{tab:bench}, 
We consider four kinds of intervention defined in terms of an event
$e_0=((r_0, \xi_0), t_0)$:
\begin{enumerate}[leftmargin=0.4cm]
\item \textit{Singular block:} This blocks only the specific event $e_0$. Formally: 
\[\BLOCKED{\iota}{e} \eqdef (e = e_0).\]
\item \textit{Template block:} This blocks every realization of the event template $(r_0, \xi_0)$ from time $t_0$ onward. In formal terms:
\[\BLOCKED{\iota}{((r, \xi), t)} \eqdef ((r, \xi) = (r_0, \xi_0)
\,\wedge\, t \geq t_0).\]
\item \textit{Agent-dependent rule block:} This blocks, from
$t_0$ onward, every event resulting from rule $r_0$ that tests an agent
modified by $e_0$. For example, we might wish to prevent kinase $K_{627}$ from binding any substrate. In formal terms: \[\BLOCKED{\iota}{((r, \xi), t)} \eqdef (r \!=\! r_0 \,\wedge\, \xi(L_r)\cap M_0 \!\neq\! \emptyset \,\wedge\, t \!\geq\! t_0)\] where $M_0$ is the set of agents modified by $e_0$.
\item \textit{Rule block:} This outright disables rule
$r_0$ from $t_0$ onward. Formally: \[\BLOCKED{\iota}{((r, \xi), t)} \eqdef (r = r_0 \,\wedge\, t \geq t_0).\]
\end{enumerate}
All four kinds of intervention might be useful for causal analysis in
different settings.

Our experimental setup consists of the model in Figure~\ref{fig:model} comprising $10^4$ substrates and $10^4$ kinases. Every agent starts out in an unbound and unphosphorylated state. The simulation is stopped as soon as the system contains more phosphorylated than unphosphorylated substrates. We then proceed as follows.
\begin{inparaenum}[(i)]
\item We first generate $n_{\tau}=10$ reference traces and record the \textsc{cpu} time $T$ it took the Kappa simulator to generate each of them. For each trace, we also identify the first application of each $r\in\{b,u,u^{\ast},p,pk\}$ and declare it to be the event $e_0$ underlying the intervention.
\item We test the $20$ possible
interventions, based on $e_0$, that can be formed by combining each of the five rules $r$ with each of the four intervention types.
\item For each such intervention $\iota$, we generate $n_{\tau',\iota} = 10$
  counterfactual traces $\tau'$.
\item For every counterfactual trace $\tau'$, we record 
 the \textsc{cpu} time $T'$ used by our implementation to generate it. We
also record the number $N_{\emptyset}$ of iterations that neither produced a
counterfactual event nor consumed a factual event (non-productive cycles).
Finally, we define the \textit{slowdown} $S$ of counterfactual resimulation
relative to simulation as the ratio of $T'$, normalized by the number of distinct events $|\tau \cup \tau'|$ in the counterfactual experiment $(\tau, \iota, \tau')$, to $T$, normalized by the number $|\tau|$ of events in
$\tau$:
\[ 
  S \,\eqdef\, \frac{|\tau|}{|\tau \cup \tau'|} \cdot\frac{T'}{T}. 
\] 
The average value and standard deviation of these quantities is shown
Table~\ref{tab:bench}. Note that each row of the table corresponds to one
intervention $\iota$ and to a sample set of $n_{\tau} \times n_{\tau',\iota} =
100$ counterfactual experiments. For every intervention, we also report a
measure of how much counterfactual traces differ from their cognate factual
trace on average: given a counterfactual experiment $(\tau, \iota, \tau')$, we
write $|\tau \!\setminus\! \tau'|$ for the number of events that are proper to
$\tau$ and $|\tau' \!\setminus\! \tau|$ the number of events that are proper to
$\tau'$ (also called counterfactual events). 
\end{inparaenum}

\subsection{Results}

The observed slowdown $S$ never exceeds 50\% on average. No intervention
produced a non-productive cycle. This is not too surprising, as all the
interventions we considered are regular, with the only exception of the
``\textit{template block}'' for rule $b$. Although this intervention can produce
non-productive cycles in theory, it is highly unlikely for a kinase to bind the same substrate twice in a large mixture. More
generally, an intervention $\iota$ that is irregular, because
$\BLOCKED{\iota}{((r, \xi), t)}$ features a conjunction of terms constraining
$\xi$ on different connected components of $L_r$, does not tend to induce many
non-productive cycles for a similar reason and, therefore, can often be handled
efficiently anyway.

As expected, we observe that the ``stronger'' the intervention, the bigger the divergence of counterfactual traces from their factual reference trace. Moreover, interventions that only affect a small number of
agents in a large mixture do not cause major divergences at the population level. In fact, only the five rule-blocking interventions had a major impact.

% Besides, the interventions
% for which counterfactual resimulation involves futile cycles are those
% of the second category (\textit{template knock-down}), which is not
% surprising as they are the only ones to be non-regular among those we
% considered. Our implementation performs pretty well on those
% interventions still, and the number of futile cycles never exceeds
% $1/50^{th}$ of the average size $|\tau|$ of the reference trace.


 \begin{table}\footnotesize
  \setstretch{1.3}
  \begin{center}
    \begin{tabular}{|l|l||c|c|c|c|c|}
\cline{3-7}\multicolumn{2}{l|}{} &
$T'$ & $S$ & $|\tau \setminus \tau'|$ & $|\tau' \setminus \tau|$ & $N_F$ \\ 
\hline\hline 
\multirow{5}{*}{\begin{sideways}\footnotesize Punctual \end{sideways}}
& $b$ & $5.77$ & $1.15\ \ {(\sigma=.03)}$ & $2.0$ & $0$ & $0$ \\ 
& $u$ & $5.76$ & $1.15\ \ {(\sigma=.02)}$ & $1.0$ & $1.0$ & $0$ \\ 
& $u*$ & $5.77$ & $1.15\ \ {(\sigma=.03)}$ & $6.0$ & $3.8$ & $0$ \\ 
& $p$ & $5.78$ & $1.15\ \ {(\sigma=.03)}$ & $1.0$ & $.9$ & $0$ \\ 
& $pk$ & $5.77$ & $1.15\ \ {(\sigma=.03)}$ & $11.6$ & $26.9$ & $0$ \\ 
\hline
\multirow{5}{*}{\begin{sideways}\footnotesize Template \end{sideways}}
& $b$ & $5.75$ & $1.15\ \ {(\sigma=.03)}$ & $2.0$ & $0$ & $0$ \\ 
& $u$ & $5.80$ & $1.16\ \ {(\sigma=.03)}$ & $38.5$ & $2.8$ & $9.7\mathrm{e}3$ \\ 
& $u*$ & $5.79$ & $1.16\ \ {(\sigma=.03)}$ & $28.8$ & $8.6$ & $9.2$ \\ 
& $p$ & $5.75$ & $1.15\ \ {(\sigma=.03)}$ & $1.0$ & $.9$ & $0$ \\ 
& $pk$ & $5.75$ & $1.15\ \ {(\sigma=.03)}$ & $11.6$ & $26.9$ & $0$ \\ 
\hline
\multirow{5}{*}{\begin{sideways}\footnotesize Affected \end{sideways}}
& $b$ & $6.90$ & $1.38\ \ {(\sigma=.03)}$ & $38.4$ & $2.3$ & $0$ \\ 
& $u$ & $6.80$ & $1.36\ \ {(\sigma=.03)}$ & $40.3$ & $1.6$ & $0$ \\ 
& $u*$ & $6.71$ & $1.34\ \ {(\sigma=.03)}$ & $29.7$ & $9.7$ & $0$ \\ 
& $p$ & $6.78$ & $1.35\ \ {(\sigma=.03)}$ & $1.0$ & $0$ & $0$ \\ 
& $pk$ & $6.67$ & $1.33\ \ {(\sigma=.03)}$ & $10.3$ & $25.4$ & $0$ \\ 
\hline
\multirow{5}{*}{\begin{sideways}\footnotesize Rule \end{sideways}}
& $b$ & $4.65$ & $.93\ \ {(\sigma=.03)}$ & $1.8\mathrm{e}5$ & $0$ & $0$ \\ 
& $u$ & $5.46$ & $1.01\ \ {(\sigma=.03)}$ & $1.6\mathrm{e}5$ & $1.4\mathrm{e}4$ & $0$ \\ 
& $u*$ & $5.93$ & $1.14\ \ {(\sigma=.03)}$ & $4.1\mathrm{e}4$ & $8.5\mathrm{e}3$ & $0$ \\ 
& $p$ & $5.96$ & $1.19\ \ {(\sigma=.03)}$ & $6.7\mathrm{e}3$ & $0$ & $0$ \\ 
& $pk$ & $6.55$ & $1.04\ \ {(\sigma=.02)}$ & $2.0\mathrm{e}4$ & $4.8\mathrm{e}4$ & $0$ \\ 
\hline
\end{tabular}

% Simulation time: 5.01E+00 (sigma = 1.11E-01)
% Reference trace size: 1.90E+05 (sigma = 1.58E+03)

  \end{center}
  \caption{
A benchmark of counterfactual resimulation. On average,
    $T = 6.01\pm 1.09$ s.  In addition,
    $|\tau| = 1.9\mathrm{e}5 \pm 1.2\mathrm{e}3$.}\label{tab:bench}
\end{table}
