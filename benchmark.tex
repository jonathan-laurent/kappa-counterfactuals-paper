% -*- TeX-master: "ijcai18.tex" -*-

\newcommand{\subs}[2]{#1_{\textsf{#2}}}

We assessed the performances of our implementation of counterfactual
resimulation on our example model (Figure~\ref{fig:model}), using an
initial mixture that features $10^4$ substrates and the same number of
kinases.  Although such a simple setting is not of particular
biological significance, it is adequate for an initial performance
benchmark.

In this benchmark, whose results are summarized in
Table~\ref{tab:bench}, we consider four different kinds of
interventions, which are all defined relative to a single event
$e_0=((r_0, \xi_0), t_0)$:
\begin{enumerate}[leftmargin=0.4cm]
\item \textit{Punctual knock-down:} block $e_0$ only. In other words,
  $\BLOCKED{\iota}{e} \eqdef (e = e_0)$.
\item \textit{Template knock-down:} block the event template
  associated to $e_0$, starting at time $t_0$. In other words,
  $\BLOCKED{\iota}{((r, \xi), t)} \eqdef ((r, \xi) = (r_0, \xi_0)
  \,\wedge\, t \geq t_0)$.
\item \textit{Rule knock-down for modified agents:} block any
  application of rule $r_0$ that tests an agent modified by $e_0$,
  starting at time $t_0$. In other words,
  $\BLOCKED{\iota}{((r, \xi), t)} \eqdef (r = r_0 \,\wedge\, \xi(L_r)
    \cap M_0 \neq \emptyset \,\wedge\, t \geq t_0)$ where $M_0$ is
  the set of agents modified by $e_0$.
\item \textit{Rule knock-down:} block the rule $r_0$, starting at time
  $t_0$. In other words,
  $\BLOCKED{\iota}{((r, \xi), t)} \eqdef (r = r_0 \,\wedge\, t \geq
  t_0)$.
\end{enumerate}
The \emph{slowdown} $S$ is defined as
\[ S \eqdef \frac{|\tau|}{|\tau \cup \tau'|} \cdot \frac{T'}{T} \]


\begin{table}\footnotesize
  \setstretch{1.3}
  \begin{center}
    \begin{tabular}{|l|l||c|c|c|c|c|}
\cline{3-7}\multicolumn{2}{l|}{} &
$T'\ (s)$ & $S$ & $|\tau \!\setminus\! \tau'|$ & $|\tau' \!\setminus\! \tau|$ & $N_{\emptyset}$ \\ 
\hline\hline 
\multirow{5}{*}{\begin{sideways}\footnotesize Singular \end{sideways}}
& $b$ & $4.70$ & $1.17\ \text{\scriptsize $ \pm\, .03$}$ & $2.0$ & $0$ & $0$ \\ 
& $u$ & $4.70$ & $1.17\ \text{\scriptsize $ \pm\, .03$}$ & $1.0$ & $1.0$ & $0$ \\ 
& $u*$ & $4.70$ & $1.17\ \text{\scriptsize $ \pm\, .04$}$ & $8.3$ & $5.1$ & $0$ \\ 
& $p$ & $4.70$ & $1.17\ \text{\scriptsize $ \pm\, .03$}$ & $1.0$ & $0.5$ & $0$ \\ 
& $pk$ & $4.68$ & $1.17\ \text{\scriptsize $ \pm\, .03$}$ & $8.0$ & $17.6$ & $0$ \\ 
\hline
\multirow{5}{*}{\begin{sideways}\footnotesize Template \end{sideways}}
& $b$ & $4.69$ & $1.17\ \text{\scriptsize $ \pm\, .04$}$ & $2.0$ & $0$ & $0$ \\ 
& $u$ & $5.52$ & $1.38\ \text{\scriptsize $ \pm\, .05$}$ & $30.3$ & $2.9$ & $0$ \\ 
& $u*$ & $5.43$ & $1.35\ \text{\scriptsize $ \pm\, .04$}$ & $30.3$ & $6.9$ & $0$ \\ 
& $p$ & $5.50$ & $1.37\ \text{\scriptsize $ \pm\, .04$}$ & $1.0$ & $0.5$ & $0$ \\ 
& $pk$ & $5.41$ & $1.35\ \text{\scriptsize $ \pm\, .04$}$ & $8.0$ & $17.6$ & $0$ \\ 
\hline
\multirow{5}{*}{\begin{sideways}\footnotesize Agent \end{sideways}}
& $b$ & $5.62$ & $1.40\ \text{\scriptsize $ \pm\, .04$}$ & $29.6$ & $0.9$ & $0$ \\ 
& $u$ & $5.55$ & $1.38\ \text{\scriptsize $ \pm\, .04$}$ & $30.3$ & $2.9$ & $0$ \\ 
& $u*$ & $5.45$ & $1.36\ \text{\scriptsize $ \pm\, .04$}$ & $30.3$ & $6.9$ & $0$ \\ 
& $p$ & $5.52$ & $1.38\ \text{\scriptsize $ \pm\, .04$}$ & $1.0$ & $0$ & $0$ \\ 
& $pk$ & $5.42$ & $1.35\ \text{\scriptsize $ \pm\, .04$}$ & $8.0$ & $17.6$ & $0$ \\ 
\hline
\multirow{5}{*}{\begin{sideways}\footnotesize Rule \end{sideways}}
& $b$ & $3.78$ & $.94\ \text{\scriptsize $ \pm\, .03$}$ & $1.4\mathrm{e}5$ & $0$ & $0$ \\ 
& $u$ & $4.43$ & $1.02\ \text{\scriptsize $ \pm\, .03$}$ & $1.3\mathrm{e}5$ & $1.2\mathrm{e}4$ & $0$ \\ 
& $u*$ & $4.82$ & $1.16\ \text{\scriptsize $ \pm\, .03$}$ & $2.5\mathrm{e}4$ & $5.3\mathrm{e}3$ & $0$ \\ 
& $p$ & $4.82$ & $1.20\ \text{\scriptsize $ \pm\, .03$}$ & $5.0\mathrm{e}3$ & $0$ & $0$ \\ 
& $pk$ & $5.19$ & $1.08\ \text{\scriptsize $ \pm\, .03$}$ & $1.4\mathrm{e}4$ & $3.0\mathrm{e}4$ & $0$ \\ 
\hline
\end{tabular}

% Simulation time: 4.01E+00 (sigma = 1.13E-01)
% Reference trace size: 1.55E+05 (sigma = 1.52E+03)

  \end{center}
  \caption{A benchmark of counterfactual resimulation. On average,
    $T = 5.01$s with a standard deviation of $0.11s$. Besides,
    $|\tau| = 1.9\mathrm{e}5$ with a standard deviation of
    $1.6\mathrm{e}3$.}\label{tab:bench}
\end{table}
