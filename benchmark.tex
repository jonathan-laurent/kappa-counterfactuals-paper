% -*- TeX-master: "ijcai18.tex" -*-

\newcommand{\subs}[2]{#1_{\textsf{#2}}}

We assessed the performances of our implementation of counterfactual
resimulation on our example model (Figure~\ref{fig:model}), using an
initial mixture that features $10^4$ substrates and the same number of
kinases.  Although such a simple setting is not of particular
biological significance, it is adequate for an initial performance
benchmark.

\subsection{Experience description}

%In this benchmark, whose results are summarized in Table~\ref{tab:bench}, 
We consider four different kinds of
interventions, which are all defined relative to an event
$e_0=((r_0, \xi_0), t_0)$:
\begin{enumerate}[leftmargin=0.4cm]
\item \textit{Punctual knock-down:} block $e_0$ only. In other words,
  $\BLOCKED{\iota}{e} \eqdef (e = e_0)$.
\item \textit{Template knock-down:} starting at time $t_0$, block
  every realization of the event template associated with $e_0$.
Formally,
  $\BLOCKED{\iota}{((r, \xi), t)} \eqdef ((r, \xi) = (r_0, \xi_0)
  \,\wedge\, t \geq t_0)$.
\item \textit{Rule knock-down for modified agents:} starting at time
  $t_0$, block any event instantiating rule $r_0$ that tests an agent
  modified by $e_0$. In other words,
  $\BLOCKED{\iota}{((r, \xi), t)} \eqdef (r = r_0 \,\wedge\, \xi(L_r)
  \cap M_0 \neq \emptyset \,\wedge\, t \geq t_0)$, where $M_0$ is the
  set of agents modified by $e_0$.
\item \textit{Rule knock-down:} starting at time $t_0$, disable rule
  $r_0$. In other words,
  $\BLOCKED{\iota}{((r, \xi), t)} \eqdef (r = r_0 \,\wedge\, t \geq
  t_0)$.
\end{enumerate}
All four kinds of interventions may be useful for causal analysis in
different settings, although we are still investigating best practices
(see our conclusion).

The benchmark proceeds as follows:
\begin{inparaenum}[(i)]
\item We first generate $n=10$ reference traces using the Kappa
  simulator, from an initial mixture that is composed of $10^4$
  substrates and the same number of kinases. Every agent starts in a
  free and unphosphorylated state. We stop the simulation when a
  majority of substrates become phosphorylated.
\item For every reference trace $\tau$, we write $T$ the \textsc{cpu} time
  needed by the Kappa simulator to generate $\tau$ and consider the
  $20$ possible interventions that can be formed by choosing a rule
  $r$ and then choosing an intervention in the list above, taking for
  $e_0$ the first application of $r$ in $\tau$.
\item For every such intervention $\iota$, we generate $n = 10$ counterfactual
  traces.
\item For every generated counterfactual trace $\tau'$, we write $T'$
  the \textsc{cpu} time that is used by our implementation to generate
  $\tau'$ and $N_F$ the number of non-productive (futile) iterations
  it went through in doing so. An iteration of counterfactual
  resimulation is said to be non-productive if it does not produce an
  event nor consume a factual event. Finally, we define the \textit{slowdown}
  $S$ of counterfactual resimulation relative to simulation as
  \[ S \,\eqdef\, \frac{|\tau|}{|\tau \cup \tau'|} \cdot \frac{T'}{T}. \]
  This can be seen as the ratio between $T'$ and $T$, where $T$ is
  normalized by the number $|\tau|$ of events in $\tau$ and $T'$ is
  normalized by the number of distinct events in the counterfactual experiment
  $(\tau, \iota, \tau')$, which we write $|\tau \cup \tau'|$.

  The average value (and sometimes the standard deviation) of these
  quantities is shown Table~\ref{tab:bench}. Note that each line of
  the table corresponds to one intervention and to a sample set of
  $n^2 = 100$ counterfactual experiments. For every intervention, we
  also give a measure of how much counterfactual traces differ from
  their cognate factual trace on average: given a counterfactual
  experiment $(\tau, \iota, \tau')$, we write
  $|\tau \!\setminus\! \tau'|$ the number of events that are proper to
  $\tau$ and $|\tau' \!\setminus\! \tau|$ the number of events that
  are proper to $\tau'$ (also called counterfactual events).
\end{inparaenum}

\subsection{Results}

The observed slowdown never exceeds 40\% on average. Besides, no
intervention was observed to ever produce a futile cycle. This should
not be surprising though, as all the interventions we considered are
regular, with the only exception of the ``\textit{template
  knock-down}'' for rule $b$ (sixth row in
Table~\ref{tab:bench}). Although this intervention can produce futile
cycles in theory, it is highly unlikely to do so.  Indeed, it is very
unlikely for a kinase to bind the same substrate twice in a large
mixture. More generally, an intervention $\iota$ that fails to be
regular because $\BLOCKED{\iota}{((r, \xi), t)}$ features a
conjunction of terms constraining $\xi$ on different connected
components of $L_r$ tends to not induce many futile cycles for a
similar reason and therefore can often be handled efficiently anyway.

Unsurprisingly, we can also see that the ``stronger'' the
intervention, the more counterfactual traces tend to diverge from
their reference trace. Besides, in a large mixture, interventions that
only affect a small number of agents do not cause major divergences at
the population level. Indeed, only the five ``rule-knocking''
interventions induced major divergences.



% Besides, the interventions
% for which counterfactual resimulation involves futile cycles are those
% of the second category (\textit{template knock-down}), which is not
% surprising as they are the only ones to be non-regular among those we
% considered. Our implementation performs pretty well on those
% interventions still, and the number of futile cycles never exceeds
% $1/50^{th}$ of the average size $|\tau|$ of the reference trace.


 \begin{table}\footnotesize
  \setstretch{1.3}
  \begin{center}
    \begin{tabular}{|l|l||c|c|c|c|c|}
\cline{3-7}\multicolumn{2}{l|}{} &
$T'$ & $S$ & $|\tau \setminus \tau'|$ & $|\tau' \setminus \tau|$ & $N_F$ \\ 
\hline\hline 
\multirow{5}{*}{\begin{sideways}\footnotesize Punctual \end{sideways}}
& $b$ & $5.77$ & $1.15\ \ {(\sigma=.03)}$ & $2.0$ & $0$ & $0$ \\ 
& $u$ & $5.76$ & $1.15\ \ {(\sigma=.02)}$ & $1.0$ & $1.0$ & $0$ \\ 
& $u*$ & $5.77$ & $1.15\ \ {(\sigma=.03)}$ & $6.0$ & $3.8$ & $0$ \\ 
& $p$ & $5.78$ & $1.15\ \ {(\sigma=.03)}$ & $1.0$ & $.9$ & $0$ \\ 
& $pk$ & $5.77$ & $1.15\ \ {(\sigma=.03)}$ & $11.6$ & $26.9$ & $0$ \\ 
\hline
\multirow{5}{*}{\begin{sideways}\footnotesize Template \end{sideways}}
& $b$ & $5.75$ & $1.15\ \ {(\sigma=.03)}$ & $2.0$ & $0$ & $0$ \\ 
& $u$ & $5.80$ & $1.16\ \ {(\sigma=.03)}$ & $38.5$ & $2.8$ & $9.7\mathrm{e}3$ \\ 
& $u*$ & $5.79$ & $1.16\ \ {(\sigma=.03)}$ & $28.8$ & $8.6$ & $9.2$ \\ 
& $p$ & $5.75$ & $1.15\ \ {(\sigma=.03)}$ & $1.0$ & $.9$ & $0$ \\ 
& $pk$ & $5.75$ & $1.15\ \ {(\sigma=.03)}$ & $11.6$ & $26.9$ & $0$ \\ 
\hline
\multirow{5}{*}{\begin{sideways}\footnotesize Affected \end{sideways}}
& $b$ & $6.90$ & $1.38\ \ {(\sigma=.03)}$ & $38.4$ & $2.3$ & $0$ \\ 
& $u$ & $6.80$ & $1.36\ \ {(\sigma=.03)}$ & $40.3$ & $1.6$ & $0$ \\ 
& $u*$ & $6.71$ & $1.34\ \ {(\sigma=.03)}$ & $29.7$ & $9.7$ & $0$ \\ 
& $p$ & $6.78$ & $1.35\ \ {(\sigma=.03)}$ & $1.0$ & $0$ & $0$ \\ 
& $pk$ & $6.67$ & $1.33\ \ {(\sigma=.03)}$ & $10.3$ & $25.4$ & $0$ \\ 
\hline
\multirow{5}{*}{\begin{sideways}\footnotesize Rule \end{sideways}}
& $b$ & $4.65$ & $.93\ \ {(\sigma=.03)}$ & $1.8\mathrm{e}5$ & $0$ & $0$ \\ 
& $u$ & $5.46$ & $1.01\ \ {(\sigma=.03)}$ & $1.6\mathrm{e}5$ & $1.4\mathrm{e}4$ & $0$ \\ 
& $u*$ & $5.93$ & $1.14\ \ {(\sigma=.03)}$ & $4.1\mathrm{e}4$ & $8.5\mathrm{e}3$ & $0$ \\ 
& $p$ & $5.96$ & $1.19\ \ {(\sigma=.03)}$ & $6.7\mathrm{e}3$ & $0$ & $0$ \\ 
& $pk$ & $6.55$ & $1.04\ \ {(\sigma=.02)}$ & $2.0\mathrm{e}4$ & $4.8\mathrm{e}4$ & $0$ \\ 
\hline
\end{tabular}

% Simulation time: 5.01E+00 (sigma = 1.11E-01)
% Reference trace size: 1.90E+05 (sigma = 1.58E+03)

  \end{center}
  \caption{A benchmark of counterfactual resimulation. On average,
    $T = 5.01$s with a standard deviation of $0.11s$. Besides,
    $|\tau| = 1.9\mathrm{e}5$ with a standard deviation of
    $1.6\mathrm{e}3$.}\label{tab:bench}
\end{table}
