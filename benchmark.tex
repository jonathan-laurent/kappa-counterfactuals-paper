% -*- TeX-master: "ijcai18.tex" -*-

\newcommand{\subs}[2]{#1_{\textsf{#2}}}

We assessed the performances of our implementation of counterfactual
resimulation on our example model (Figure~\ref{fig:model}), using an
initial mixture that features $10^4$ substrates and the same number of
kinases.  Although such a simple setting is not of particular
biological significance, it is adequate for an initial performance
benchmark.

In this benchmark, whose results are summarized in
Table~\ref{tab:bench}, we consider four different kinds of
interventions, which are all defined relative to a single event
$e_0=((r_0, \xi_0), t_0)$:
\begin{enumerate}[leftmargin=0.4cm]
\item \textit{Punctual knock-down:} block $e_0$ only. In other words,
  $\BLOCKED{\iota}{e} \eqdef (e = e_0)$.
\item \textit{Template knock-down:} block the event template
  associated to $e_0$, starting at time $t_0$. In other words,
  $\BLOCKED{\iota}{((r, \xi), t)} \eqdef ((r, \xi) = (r_0, \xi_0)
  \,\wedge\, t \geq t_0)$.
\item \textit{Rule knock-down for modified agents:} block any
  application of rule $r_0$ that tests an agent modified by $e_0$,
  starting at time $t_0$. In other words,
  $\BLOCKED{\iota}{((r, \xi), t)} \eqdef (r = r_0 \,\wedge\, \xi(L_r)
    \cap M_0 \neq \emptyset \,\wedge\, t \geq t_0)$ where $M_0$ is
  the set of agents modified by $e_0$.
\item \textit{Rule knock-down:} block the rule $r_0$, starting at time
  $t_0$. In other words,
  $\BLOCKED{\iota}{((r, \xi), t)} \eqdef (r = r_0 \,\wedge\, t \geq
  t_0)$.
\end{enumerate}
The \emph{slowdown} $S$ is defined as
\[ S \eqdef \frac{|\tau|}{|\tau \cup \tau'|} \cdot \frac{T'}{T} \]


\begin{table}\footnotesize
  \setstretch{1.3}
  \begin{center}
    \begin{tabular}{|l|l||c|c|c|c|c|}
\cline{3-7}\multicolumn{2}{l|}{} &
$T'$ & $S$ & $|\tau \setminus \tau'|$ & $|\tau' \setminus \tau|$ & $N_F$ \\ 
\hline\hline 
\multirow{5}{*}{\begin{sideways}\footnotesize Punctual \end{sideways}}
& $b$ & $5.77$ & $1.15\ \ {(\sigma=.03)}$ & $2.0$ & $0$ & $0$ \\ 
& $u$ & $5.76$ & $1.15\ \ {(\sigma=.02)}$ & $1.0$ & $1.0$ & $0$ \\ 
& $u*$ & $5.77$ & $1.15\ \ {(\sigma=.03)}$ & $6.0$ & $3.8$ & $0$ \\ 
& $p$ & $5.78$ & $1.15\ \ {(\sigma=.03)}$ & $1.0$ & $.9$ & $0$ \\ 
& $pk$ & $5.77$ & $1.15\ \ {(\sigma=.03)}$ & $11.6$ & $26.9$ & $0$ \\ 
\hline
\multirow{5}{*}{\begin{sideways}\footnotesize Template \end{sideways}}
& $b$ & $5.75$ & $1.15\ \ {(\sigma=.03)}$ & $2.0$ & $0$ & $0$ \\ 
& $u$ & $5.80$ & $1.16\ \ {(\sigma=.03)}$ & $38.5$ & $2.8$ & $9.7\mathrm{e}3$ \\ 
& $u*$ & $5.79$ & $1.16\ \ {(\sigma=.03)}$ & $28.8$ & $8.6$ & $9.2$ \\ 
& $p$ & $5.75$ & $1.15\ \ {(\sigma=.03)}$ & $1.0$ & $.9$ & $0$ \\ 
& $pk$ & $5.75$ & $1.15\ \ {(\sigma=.03)}$ & $11.6$ & $26.9$ & $0$ \\ 
\hline
\multirow{5}{*}{\begin{sideways}\footnotesize Affected \end{sideways}}
& $b$ & $6.90$ & $1.38\ \ {(\sigma=.03)}$ & $38.4$ & $2.3$ & $0$ \\ 
& $u$ & $6.80$ & $1.36\ \ {(\sigma=.03)}$ & $40.3$ & $1.6$ & $0$ \\ 
& $u*$ & $6.71$ & $1.34\ \ {(\sigma=.03)}$ & $29.7$ & $9.7$ & $0$ \\ 
& $p$ & $6.78$ & $1.35\ \ {(\sigma=.03)}$ & $1.0$ & $0$ & $0$ \\ 
& $pk$ & $6.67$ & $1.33\ \ {(\sigma=.03)}$ & $10.3$ & $25.4$ & $0$ \\ 
\hline
\multirow{5}{*}{\begin{sideways}\footnotesize Rule \end{sideways}}
& $b$ & $4.65$ & $.93\ \ {(\sigma=.03)}$ & $1.8\mathrm{e}5$ & $0$ & $0$ \\ 
& $u$ & $5.46$ & $1.01\ \ {(\sigma=.03)}$ & $1.6\mathrm{e}5$ & $1.4\mathrm{e}4$ & $0$ \\ 
& $u*$ & $5.93$ & $1.14\ \ {(\sigma=.03)}$ & $4.1\mathrm{e}4$ & $8.5\mathrm{e}3$ & $0$ \\ 
& $p$ & $5.96$ & $1.19\ \ {(\sigma=.03)}$ & $6.7\mathrm{e}3$ & $0$ & $0$ \\ 
& $pk$ & $6.55$ & $1.04\ \ {(\sigma=.02)}$ & $2.0\mathrm{e}4$ & $4.8\mathrm{e}4$ & $0$ \\ 
\hline
\end{tabular}

% Simulation time: 5.01E+00 (sigma = 1.11E-01)
% Reference trace size: 1.90E+05 (sigma = 1.58E+03)

  \end{center}
  \caption{A benchmark of counterfactual resimulation. On average,
    $T = 5.01$s with a standard deviation of $0.11s$. Besides,
    $|\tau| = 1.9\mathrm{e}5$ with a standard deviation of
    $1.6\mathrm{e}3$.}\label{tab:bench}
\end{table}
