% -*- TeX-master: "ijcai18.tex" -*-

\newcommand{\subs}[2]{#1_{\textsf{#2}}}

We assessed the performance of our implementation of counterfactual
resimulation using the toy model of Figure~\ref{fig:model}, but with an initial
mixture consisting of a large number of kinase and substrate instances. Although
such a simple model is not of biological significance, it is adequate for an
initial assessment of performance.

\subsection{Experimental Protocol}

%In this benchmark, whose results are summarized in Table~\ref{tab:bench}, 
We consider four kinds of intervention defined in terms of an event
$e_0=((r_0, \xi_0), t_0)$:
\begin{enumerate}[leftmargin=0.4cm]
\item \textit{Singular block:} This blocks only the specific event $e_0$:
\[\BLOCKED{\iota}{e} \eqdef (e = e_0).\]
\item \textit{Template block:} This blocks every realization of the event template $(r_0, \xi_0)$ from time $t_0$ onward. In formal terms:
\[\BLOCKED{\iota}{((r, \xi), t)} \eqdef ((r, \xi) = (r_0, \xi_0)
\,\wedge\, t \geq t_0).\]
\item \textit{Agent-dependent rule block:} This blocks, from
$t_0$ onward, every event resulting from rule $r_0$ that tests an agent
modified by $e_0$. For example, we might wish to prevent substrate $S_{627}$ from being phosphorylated by any kinase. In formal terms: \[\BLOCKED{\iota}{((r, \xi), t)} \eqdef (r \!=\! r_0 \,\wedge\, \xi(L_r)\cap M_0 \!\neq\! \emptyset \,\wedge\, t \!\geq\! t_0)\] where $M_0$ is the set of agents modified by $e_0$.
\item \textit{Rule block:} This outright disables rule
$r_0$ from $t_0$ onward:
\[\BLOCKED{\iota}{((r, \xi), t)} \eqdef (r = r_0 \,\wedge\, t \geq t_0).\]
\end{enumerate}
All four kinds of intervention might be useful for causal analysis in
different settings.

Our experimental setup consists of the model in Figure~\ref{fig:model} comprising $10^4$ substrates and $10^4$ kinases. Every agent starts out in an unbound and unphosphorylated state. The simulation is stopped as soon as the system contains more phosphorylated than unphosphorylated substrates. We then proceed as follows.
\begin{inparaenum}[(i)]
\item We first generate $n_{\tau}=10$ reference traces and record the
  CPU time $T$ it took the Kappa simulator to generate each of them on
  a personal computer with a 2.7GHz Intel Core i5 processor and 16GB
  of random-access memory.  For each trace, we also identify the first
  application of each $r\in\{b,u,u^{\ast},p,pk\}$ and declare it to be
  the event $e_0$ underlying the intervention.
\item We test the $20$ possible interventions, based on $e_0$, that
  can be formed by combining each of the five rules $r$ with each of
  the four intervention types.
\item For each such intervention $\iota$, we generate
  $n_{\tau',\iota} = 10$ counterfactual traces $\tau'$.
\item For every counterfactual trace $\tau'$, we record the
  \textsc{cpu} time $T'$ used by our implementation to generate it. We
  also record the number $N_{\emptyset}$ of iterations that neither
  produced a counterfactual event nor consumed a factual event
  (non-productive cycles).  Finally, we define the \textit{slowdown}
  $S$ of counterfactual resimulation relative to simulation as the
  ratio of $T'$, normalized by the number of distinct events
  $|\tau \cup \tau'|$ in the counterfactual experiment
  $(\tau, \iota, \tau')$, to $T$, normalized by the number $|\tau|$ of
  events in $\tau$:
  \vspace{-0.15cm}
  \begin{small}
    \begin{equation*}
      S \,\eqdef\, \frac{|\tau|}{|\tau \cup \tau'|} \cdot\frac{T'}{T}.
    \end{equation*}
  \end{small}
  The average value and standard deviation of these quantities is
  shown Table~\ref{tab:bench}. Note that each row of the table
  corresponds to one intervention $\iota$ and to a sample set of
  $n_{\tau} \times n_{\tau',\iota} = 100$ counterfactual
  experiments. For every intervention, we also report a measure of how
  much counterfactual traces differ from their cognate factual trace
  on average: given a counterfactual experiment
  $(\tau, \iota, \tau')$, we write $|\tau \!\setminus\! \tau'|$ for
  the number of events that are proper to $\tau$ and
  $|\tau' \!\setminus\! \tau|$ the number of events that are proper to
  $\tau'$ (also called counterfactual events).
\end{inparaenum}

\subsection{Results}

The observed slowdown $S$ never exceeds 50\% on average. No
intervention produced a non-productive cycle. This is not too
surprising, as all the interventions we considered are regular, with
the only exception of the ``\textit{template block}'' for rule
$b$. Although this intervention can produce non-productive cycles in
theory, it is highly unlikely for a kinase to bind the same substrate
twice in a large mixture. More generally, an intervention $\iota$ that
is irregular, because $\BLOCKED{\iota}{((r, \xi), t)}$ features a
conjunction of terms constraining $\xi$ on different connected
components of $L_r$, does not tend to induce many non-productive
cycles for a similar reason and, therefore, can often be handled
efficiently anyway.

As expected, we observe that the ``stronger'' the intervention, the
bigger the divergence of counterfactual traces from their factual
reference trace. Moreover, interventions that only affect a small
number of agents in a large mixture do not cause major divergences at
the population level. In fact, only the five rule-blocking
interventions had a major impact.

% Besides, the interventions
% for which counterfactual resimulation involves futile cycles are those
% of the second category (\textit{template knock-down}), which is not
% surprising as they are the only ones to be non-regular among those we
% considered. Our implementation performs pretty well on those
% interventions still, and the number of futile cycles never exceeds
% $1/50^{th}$ of the average size $|\tau|$ of the reference trace.


 \begin{table}\footnotesize
  \setstretch{1.3}
  \begin{center}
    \begin{tabular}{|l|l||c|c|c|c|c|}
\cline{3-7}\multicolumn{2}{l|}{} &
$T'\ (s)$ & $S$ & $|\tau \!\setminus\! \tau'|$ & $|\tau' \!\setminus\! \tau|$ & $N_{\emptyset}$ \\ 
\hline\hline 
\multirow{5}{*}{\begin{sideways}\footnotesize Singular \end{sideways}}
& $b$ & $4.70$ & $1.17\ \text{\scriptsize $ \pm\, .03$}$ & $2.0$ & $0$ & $0$ \\ 
& $u$ & $4.70$ & $1.17\ \text{\scriptsize $ \pm\, .03$}$ & $1.0$ & $1.0$ & $0$ \\ 
& $u*$ & $4.70$ & $1.17\ \text{\scriptsize $ \pm\, .04$}$ & $8.3$ & $5.1$ & $0$ \\ 
& $p$ & $4.70$ & $1.17\ \text{\scriptsize $ \pm\, .03$}$ & $1.0$ & $0.5$ & $0$ \\ 
& $pk$ & $4.68$ & $1.17\ \text{\scriptsize $ \pm\, .03$}$ & $8.0$ & $17.6$ & $0$ \\ 
\hline
\multirow{5}{*}{\begin{sideways}\footnotesize Template \end{sideways}}
& $b$ & $4.69$ & $1.17\ \text{\scriptsize $ \pm\, .04$}$ & $2.0$ & $0$ & $0$ \\ 
& $u$ & $5.52$ & $1.38\ \text{\scriptsize $ \pm\, .05$}$ & $30.3$ & $2.9$ & $0$ \\ 
& $u*$ & $5.43$ & $1.35\ \text{\scriptsize $ \pm\, .04$}$ & $30.3$ & $6.9$ & $0$ \\ 
& $p$ & $5.50$ & $1.37\ \text{\scriptsize $ \pm\, .04$}$ & $1.0$ & $0.5$ & $0$ \\ 
& $pk$ & $5.41$ & $1.35\ \text{\scriptsize $ \pm\, .04$}$ & $8.0$ & $17.6$ & $0$ \\ 
\hline
\multirow{5}{*}{\begin{sideways}\footnotesize Agent \end{sideways}}
& $b$ & $5.62$ & $1.40\ \text{\scriptsize $ \pm\, .04$}$ & $29.6$ & $0.9$ & $0$ \\ 
& $u$ & $5.55$ & $1.38\ \text{\scriptsize $ \pm\, .04$}$ & $30.3$ & $2.9$ & $0$ \\ 
& $u*$ & $5.45$ & $1.36\ \text{\scriptsize $ \pm\, .04$}$ & $30.3$ & $6.9$ & $0$ \\ 
& $p$ & $5.52$ & $1.38\ \text{\scriptsize $ \pm\, .04$}$ & $1.0$ & $0$ & $0$ \\ 
& $pk$ & $5.42$ & $1.35\ \text{\scriptsize $ \pm\, .04$}$ & $8.0$ & $17.6$ & $0$ \\ 
\hline
\multirow{5}{*}{\begin{sideways}\footnotesize Rule \end{sideways}}
& $b$ & $3.78$ & $.94\ \text{\scriptsize $ \pm\, .03$}$ & $1.4\mathrm{e}5$ & $0$ & $0$ \\ 
& $u$ & $4.43$ & $1.02\ \text{\scriptsize $ \pm\, .03$}$ & $1.3\mathrm{e}5$ & $1.2\mathrm{e}4$ & $0$ \\ 
& $u*$ & $4.82$ & $1.16\ \text{\scriptsize $ \pm\, .03$}$ & $2.5\mathrm{e}4$ & $5.3\mathrm{e}3$ & $0$ \\ 
& $p$ & $4.82$ & $1.20\ \text{\scriptsize $ \pm\, .03$}$ & $5.0\mathrm{e}3$ & $0$ & $0$ \\ 
& $pk$ & $5.19$ & $1.08\ \text{\scriptsize $ \pm\, .03$}$ & $1.4\mathrm{e}4$ & $3.0\mathrm{e}4$ & $0$ \\ 
\hline
\end{tabular}

% Simulation time: 4.01E+00 (sigma = 1.13E-01)
% Reference trace size: 1.55E+05 (sigma = 1.52E+03)

  \end{center}
  \caption{
A benchmark of counterfactual resimulation. On average,
    $T = 4.01\pm .12$ s.  In addition,
    $|\tau| = 1.6\mathrm{e}5 \pm 1.5\mathrm{e}3$.}\label{tab:bench}
\end{table}
