%%%%%%%%%%%%%%%%%%%%%%%%%%%%%%%%%%%%%%%%%%%%%%%%%%%%%%%%%%%%%%%%%%%%%%
%% COUNTERFACTUAL STATEMENTS
%%%%%%%%%%%%%%%%%%%%%%%%%%%%%%%%%%%%%%%%%%%%%%%%%%%%%%%%%%%%%%%%%%%%%%

\section{A semantics for counterfactuals}

We start by formalizing the notion of an intervention. An intervention $\iota$ is a predicate $\textsf{blocked}_{\iota}[t, e]$ that determines whether or not event $e$ is blocked at time $t$. Given a predicate $\varphi$ over traces, we write the proposition \textit{``Had intervention $\iota$ happened in trace $\tau$, $\varphi$ would have been true with probability greater than $p \in [0,1]$''} as:
\[ \tau \models_p [\iota] \, \varphi. \]

To give an operational meaning to this statement, we invoke the continuous time Markov chain (CTMC) semantics of a Kappa model as defined and implemented in \cite{DanosEtAl-APLAS07,BoutillierEK17}. For the present purpose it is conceptually useful to think of a CTMC abstractly in terms of the random realization of ``potential events". A potential event is a pair $(r, \xi)$ where $r$ is a rule and $\xi$ an injective mapping from local agents involved in $r$ to global agents in reaction mixture.\footnote{For the sake of simplicity, we assume that no agent is created or deleted by a rule.} For every such potential event, we imagine a bell that rings at a time $t$ drawn from an exponential distribution $\lambda_r\exp(-\lambda_r t)$, where $\lambda_r$ is the stochastic rate constant of $r$. A simulation trace can be viewed as the realization of a random variable $T$ determined by the set $\omega$ of ring times: Starting with an initial mixture, when a bell rings at $t$, its associated potential event $(r, \xi)$ transforms the mixture according to $r$ if $\xi$ yields a valid embedding of the left hand side of $r$ in the current mixture and time advances by $t$. Otherwise, time advances and nothing happens---a null event. Repeat on the resulting mixture.

We can extend this viewpoint to include interventions. For an intervention $\iota$, we define the random variable $\hat T_\iota$ much in the same way as $T$, except that each time the bell rings, we require $\textsf{blocked}_\iota[t, e]$ to be false for the potential event $e=(r, \xi)$ to be considered. 

Counterfactual traces that are closest to the actual trace $\tau$ are then sampled by generating realizations of $\hat T_\iota$ that inherit, whenever possible, the subset of $\omega$ that made up $\tau$. An efficient implementation of this specification for sampling the conditional random variable ${\hat T}_\iota \,|\, (T = \tau)$ is available at \url{https://github.com/jonathan-laurent/kappa-counterfactuals}. We refer to this natural extension of CTMC semantics as \textit{counterfactual re-simulation} or \textit{co-simulation} for short.

Using co-simulation, we operationalize the counterfactual statement $ \tau \models_p [\iota] \, \varphi $ as:
\[ \tau \models_p [\iota] \, \varphi \quad \Leftrightarrow 
\quad \mathbf{P}( \varphi(\hat T_\iota) \ |\  T = \tau) \,\geq\, p \]
