% -*- TeX-master: "ijcai18.tex" -*-

\subsection{Implementation}\label{subsec:implementation}

There are two challenges in efficiently implementing counterfactual
resimulation. The first is a suitable representation for the sets of
divergent embeddings $\DEMBS{r}{m}$ to minimize the cost of their
update at each iteration. Since the Kappa simulator solves exactly
that problem for the sets of embeddings $\EMBS{r}{m}$
\cite{DanosEtAl-APLAS07}, we leverage most of that infrastructure. The
second consists in avoiding excessively many iterations of
Algorithm~\ref{alg:cosimulation} in which time is advanced in tiny
increments and the proposed event is rejected.
%(line~\ref{cosim:blocked}). 
Suppose, for example, that in our toy model
$pk$ has a very high rate constant and we wish to block, from a
specific time onward, \emph{all} events in which the sole kinase
becomes phosphorylated. Upon blocking one occurrence of the event, the
same event would want to happen again, and we would keep rejecting it
a huge number of times until a different rule fires. More generally,
event templates whose realization is bound to be blocked should be
removed efficiently before their realization is attempted and not be
counted in the system's divergent activity. We solve this problem
for a class of interventions we call \emph{regular}.
Specifically, an intervention $\iota$ is regular if
the predicate $\BLOCKED{\iota}{((r, \xi), t)}$ can be expressed as a
finite disjunction of formulae of the form
$(r \!=\! r') \wedge F(\xi{\restriction_{c}}) \wedge (t \!\in\! I)$ or
$G(r, \xi) \wedge (t \!=\! t')$, where $r'$ is a rule, $t'$ a time, $I$ a
time interval, $\xi{\restriction_{c}}$ the restriction of $\xi$ to a
single connected component $c$ of $L_{r'}$, and $F, G$ arbitrary
predicates. For regular interventions, our implementation is
guaranteed to either produce or consume an event at each iteration.

\begin{proposition}
  Sampling a counterfactual trace for a \emph{regular} intervention
  can be done in time $\mathcal{O}(n \cdot r \log|m|)$, where $n$ is
  the sum of the number of events in the reference trace and in the
  resulting counterfactual trace, $r$ is the number of rules in the
  model and $|m|$ the size of the reaction mixture.
\end{proposition}

We provide a benchmark of our implementation on a scaled-up version of
our toy model in Appendix~C. The average slowdown per
event compared to the Kappa simulator does not exceed 50\% for a
variety of interventions.

\medskip