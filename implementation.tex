% -*- TeX-master: "ijcai18.tex" -*-

\subsection{Implementation}\label{subsec:implementation}

There are two challenges in efficiently implementing counterfactual
resimulation. The first is a suitable representation for the sets of
divergent embeddings $\DEMBS{r}{m}$ to minimize the cost of their
update at each iteration. Since the Kappa simulator solves exactly
that problem for the sets of embeddings $\EMBS{r}{m}$
\cite{DanosEtAl-APLAS07}, we leverage most of that infrastructure. The
second consists in avoiding excessively many iterations of
Algorithm~\ref{alg:cosimulation} in which time is incremented but
nothing happens. Suppose, for example, that in our toy model $pk$ has
a very high rate constant and we wish to block \emph{all} events in
which the sole kinase becomes phosphorylated from a specific time
onward. If we were to proceed by blocking each event at the time it
results from the realization of its template, we would have to reject
an enormous amount of attempted events until a different rule
fires. We exploit the fact that the Kappa simulator stores all
possible embeddings of $L_r$ into a given mixture in a highly
efficient way at the price of an over-approximation that is corrected
at event time. Using this architecture we can remove all embeddings of
rule $pk$ from the store, thus guaranteeing that no $pk$ event will be
attempted only to be rejected. However, this restricts us to
interventions that comply with the same type of approximation used in
the Kappa simulator: componentwise embeddings of $L_{r}$. We refer to
compliant interventions as \emph{regular}.  Specifically, an
intervention $\iota$ is regular if the predicate
$\BLOCKED{\iota}{((r, \xi), t)}$ can be expressed as a finite
disjunction of formulae of the form
$(r = r') \wedge F(\xi{\restriction_{c}}) \wedge (t \in I)$ or
$G(r, \xi) \wedge (t = t')$, where $r'$ is a rule, $t'$ a time, $I$ a
time interval, $\xi{\restriction_{c}}$ the restriction of $\xi$ to a
single connected component $c$ of $L_{r'}$, and $F, G$ arbitrary
predicates. For regular interventions, our implementation is
guaranteed to either produce or consume an event at each iteration.

\begin{proposition}
  Sampling a counterfactual trace for a \emph{regular} intervention
  can be done in time $\mathcal{O}(n \cdot r \log|m|)$, where $n$ is
  the sum of the number of events in the reference trace and in the
  resulting counterfactual trace, $r$ is the number of rules in the
  model and $|m|$ the size of the reaction mixture.
\end{proposition}

We provide a benchmark of our implementation on a scaled-up version of
our toy model in Appendix~\ref{ap:benchmark}. The average slowdown per
event compared to the Kappa simulator does not exceed 50\% for a
variety of interventions.
