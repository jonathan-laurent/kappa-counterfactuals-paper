% -*- TeX-master: "ijcai18.tex" -*-

\subsection{Implementation}\label{subsec:implementation}

There are two challenges in writing an efficient implementation of
counterfactual resimulation. The first one is to find a suitable
representation for the sets of divergent embeddings $\DEMBS{r}{m}$ and
update it at each iteration for a minimal cost. To do this, we reuse
most of the infrastructure behind the Kappa simulator, which solves
the same problem for the sets of embeddings $\EMBS{r}{m}$. The second
challenge is to avoid most iterations of
Algorithm~\ref{alg:cosimulation} being wasted in {futile
  cycles}. Indeed, when a rule has a high activity but most of its
instances are to be blocked in the near future, then the test of
line~\ref{cosim:blocked} may fail repeatedly, leading to inefficiencies
that can be arbitrarly high. We solve this problem for a class of
interventions we call \emph{regular}. An intervention $\iota$ is said
to be regular if the predicate $\BLOCKED{\iota}{((r, \xi), t)}$ can be
expressed as
\[ \bigvee_i f_i(r, \xi^{(i)}) \,\wedge\, (t \in I)  \]

\begin{proposition}
  Sampling a counterfactual trace for a \emph{regular} intervention
  can be done in time $\mathcal{O}(n \cdot r \log|m|)$, where $n$ is
  the sum of the number of events in the reference trace and in the
  resulting counterfactual trace, $r$ is the number of rules in the
  model and $|m|$ the size of the reaction mixture.
\end{proposition}

A benchmark of our implementation on a scaled-up version of our
example model is available in Appendix~\ref{ap:benchmark}. In particular,
we show that the average slowdown per simulated event compared to the Kappa
simulator does not exceed 30\% for a variety of interventions.
