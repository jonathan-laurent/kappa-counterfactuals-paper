% -*- TeX-master: "ijcai18.tex" -*-

\begin{abstract}
  Models based on rules that express local and heterogenous mechanisms
  of stochastic interactions between structured agents are an
  important tool for investigating the dynamical behavior of complex
  systems, especially in molecular biology. Given a simulated trace of
  events, the challenge is to construct a causal diagram that explains
  how a phenomenon of interest occurred. Counterfactual analysis can
  provide distinctive insights, but its standard definition is not
  applicable in rule-based models because they are not readily
  expressible in terms of structural equations. We provide a semantics
  of counterfactual statements that addresses this challenge by
  sampling counterfactual trajectories that are probabilistically as
  close to the factual trace as a given intervention permits them to
  be. We then show how counterfactual dependencies give rise to
  explanations in terms of relations of enablement and prevention
  between events.  
\end{abstract}