\message{ !name(cosimulation.tex)}
\message{ !name(cosimulation.tex) !offset(-2) }
\begin{definition}[Divergent embedding]
  Let $M$ a mixture and $M_0$ a reference mixture.
\end{definition}

\begin{definition}[Divergent activity]
  
\end{definition}

\begin{theorem} Let $\tau$ a trace and $\iota$ an intervention. Let
  $I$ a time interval of width $\delta$ such that
  $\tau \cap I = \emptyset$. Then,
  \[\CProb{ \hat T_\iota \cap I = \emptyset }{ T=\tau,\
      \TSTATE{t}{\hat T_\iota} = s\ }
    \ =\ e^{-\alpha'(s, s_0) \cdot \delta}
  \]
  where $s_0 = \mathcal{S}_t(\tau)$ and
  \[\alpha'(s, s_0) = \sum_r \lambda_r |\DEMBS{r}{s, s_0}| \]
  is the total divergent activity of $s$ with respect to $s_0$.
\end{theorem}

Suppose you are simulating a counterfactual trace for intervention
$\iota$ and reference trace $\tau$, that is, you are drawing an
instance of $\hat T_\iota \,|\, \{T=\tau\}$. Suppose you have already
simulated the first $t$ seconds of this counterfactual trace and you
want to know the time of the next event. Because both $T$,
$\hat T_\iota$ and $\Sigma$ are markovian, the only only relevant
information about what has already been simulated is the current state
of the counterfactual world $s := \TSTATE{t}{\hat T_\iota}$.


If we write $\delta$ the time difference between $t$ and the next
event after $t$ in $\tau$, we are interested in the following
probability:
\begin{equation}
  \CProb{ \hat T_\iota \cap [t, t+\delta) = \emptyset  }
  % { \underbrace{T=\tau, M = \mathcal{S}_t(\hat T_\iota)}_{\Delta} }
  { T=\tau,\, s = \TSTATE{t}{\hat T_\iota} }
\end{equation}
Writing $I = [t, t+\delta)$, this quantity can be rewritten as
follows:
\begin{align}
  \ & \ \CProb{ \hat T_\iota \cap I = \emptyset }
      { T=\tau,\, s = \TSTATE{t}{\hat T_\iota} } \\[0.5em]
  = & \ \CProb{ \bigwedge_{s \vdash e} (\,e \notin \Sigma \cap I \,)  }{ T=\tau } \\[1em]
  = & \ \prod_{s \vdash e}\, \CProb{e \notin \Sigma \cap I }{ T=\tau }
\end{align}
Indeed, given that $s = \TSTATE{t}{\hat T_\iota}$, no counterfactual
event happens in time interval $I$ if and only if no potential event
that is triggerable from state $s$ is scheduled in $I$.  Besides,
potential events are scheduled independently and so we can decompose
the resulting probability as a product.

Let $e$ a potential event such that $s \vdash e$. The probability that
$e$ has not been scheduled in $I$ given that $T=\tau$ depends on
whether or not $e$ is triggerable from state $s_0 :=
\TSTATE{t}{T}$. Indeed, we assumed that $\tau$ contains no event in
time interval $I$. Therefore, if $s_0 \vdash e$, then $e$ cannot be
scheduled in $I$, without which it would have been observed in $\tau$.
Thus,
\[ s_0 \vdash e \ \Rightarrow\ \CProb{e \notin \Sigma \cap I}{ T=\tau
  } = 1 \] Besides, if $s_0 \not\vdash e$, then the observation
$\{ T=\tau \}$ gives absolutely no information on whether or not $e$
has been scheduled in $I$ and so
\[ s_0 \not\vdash e \ \Rightarrow\ \CProb{e \notin \Sigma \cap I}{
    T=\tau } = e^{-\lambda_e \cdot \delta} \] as the scheduling
process of a potential event is a Poisson process. Combining these two
results with (4), we have:
\begin{align}
  \CProb{ \hat T_\iota \cap I = \emptyset }{ \Delta }
  =\ \prod_{s \vdash e, \, s_0 \not\vdash e} e^{-\lambda_e} = e^{-\alpha'(s, s_0)\cdot\delta}
\end{align}
where
\[\alpha'(s, s_0) := \sum_{s \vdash e, \, s_0 \not\vdash e}
  \lambda_e \] This quantity can be rewritten as follows:
\begin{align}
  \alpha'(s, s_0) &= 
                    \sum_{(r, \xi)} \textbf{1}\{ s \vdash (r, \xi),\ s_o \not\vdash (r, \xi) \}\cdot\lambda_r \\
                  &= \sum_r \lambda_r\sum_\xi \textbf{1}\{ s \vdash (r, \xi),\ s_o \not\vdash (r, \xi) \} \\
                  &= \sum_r \lambda_r |\DEMBS{r}{s, s_0}|
\end{align}
which gives us the definition of divergent activity. This result can
be summarized in the following theorem:

\message{ !name(cosimulation.tex) !offset(-86) }
