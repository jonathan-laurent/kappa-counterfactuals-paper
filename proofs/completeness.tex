It is convenient to prove the following result instead, which is
slightly more general. In the proof, we write $\TIME{e}$ the time of
occurence of event $e$.
\begin{theorem*} Let $(\tau, \iota, \tau')$ be a counterfactual
  experiment. If $e$ is an event that belongs only to $\tau$
  or $\tau'$, then there exists an event $\hat e$ in
  $\tau$ that is blocked by $\iota$ and there is a directed path 
  from $\hat e$ to $e$.
\end{theorem*}
\begin{proof}
  We prove this theorem by induction on the number of events before
  $e$ in both $\tau$ and $\tau'$. Let's consider $e \in \tau$ such
  that $e \notin \tau'$. If $\BLOCKED{\iota}{e}$ is true, then we are
  done. Otherwise, by item ($\ref{valid-cex:co-occur}$) of
  Proposition~\ref{prop:valid-cex}, $e$ is not executable in
  $\tau'$. Therefore, there exists a site $s$ such that event $e$
  tests $s$ to a different value than it has in $\TSTATE{t}{\tau'}$.
  Let $t = \TIME{e}$.  Let's define $e_0$ the last event in $\tau$
  modifying $s$ that occurs strictly before time $t$, and $e_0'$ the
  last event in $\tau'$ modifying $s$ that occurs strictly before time
  $t$. These events modify $s$ to different values so they cannot be
  the same. Therefore, we have $\TIME{e_0} \neq \TIME{e_0'}$ with
  probability one.

  \begin{itemize}
  \item If $\TIME{e_0} < \TIME{e_0'}$, then we have
    $e_0 \notin \tau'$ and so we can apply the induction hypothesis on
    $e_0$. As a consequence, there exists a path from an event
    $\hat e$ in $\tau$ which is blocked by $\iota$ to $e_0$. Besides,
    $e_0$ enables $e$. Therefore, there is a path between $\hat e$
    and $e$.
  \item If $\TIME{e_0'} < \TIME{e_0}$, then we have
    $e'_0 \notin \tau$ and so we can apply the induction
    hypothesis on $e'_0$. As a consequence, there exists a path from
    an event $\hat e$ in $\tau$ which is blocked by $\iota$ to
    $e'_0$. Besides, $e_0'$ prevents $e$. Therefore, there is a path
    between $e$ and $\hat e$.
  \end{itemize}
  The same proof holds for when $e \in \tau'$ and $e \notin \tau$,
  using item~($\ref{valid-cex:co-occur2}$) of
  Proposition~\ref{prop:valid-cex} instead of
  item~($\ref{valid-cex:co-occur}$).
\end{proof}
This result implies Theorem~\ref{thm:completeness}. In particular,
\emph{any} directed path from $\tau$ to $\tau$ necessarily contains an
even number of prevention arrows as these arrows always go from
$\tau$ to $\tau'$ or the other way around.
