\begin{theorem*} Let $(\tau, \iota, \tau')$ be a counterfactual
  experiment. If $e$ is an event that belongs only to $\tau$
  (respectively $\tau'$), then there exists an event $\hat e$ in
  $\tau$ that is blocked by $\iota$ and there is a path of activation
  and inhibition arrows from $\hat e$ to $e$.
\end{theorem*}
\begin{proof}
  We prove this theorem by induction on the number of events before
  $e$ in both $\tau$ and $\tau'$. Let's consider $(e, t) \in \tau$
  such that $(e, t) \notin \tau'$. If $\BLOCKED{\iota}{(e, t)}$ is
  true, then we are done. Otherwise, by ($\ref{valid-cex:co-occur}$),
  $e$ is not triggerable in $\TSTATE{t}{\tau'}$. Therefore, there
  exists a site $s$ such that $s$ is not in the state which is
  required for $e$ to happen in mixture $\TSTATE{t}{\tau'}$. Let's
  define $(e_0, t_0)$ the last event in $\tau$ before time $t$ that
  modifies site $s$ and $(e_0', t'_0)$ the last event in $\tau'$
  before $e$ that also modifies $s$. These events modify $s$ to
  different values so they cannot be the same. Therefore, we have
  $t_0 \neq t_0'$ with probability one.

  \begin{itemize}
  \item If $\TIME{e_0} < \TIME{e_0'}$ (test of notation), then we have
    $(e_0, t_0) \notin \tau'$ and so we can apply the induction
    hypothesis on $e_0$. As a consequence, there exists a path from an
    event $\hat e$ in $\tau$ which is blocked by $\iota$ to
    $e_0$. Besides, $e_0$ activates $e$. Therefore, there is a path
    between $\hat e$ and $e$.
  \item If $t_0' < t_0$, then we have $(e'_0, t'_0) \notin \tau$ and
    so we can apply the induction hypothesis on $e'_0$. As a
    consequence, there exists a path from an event $\hat e$ in $\tau$
    which is blocked by $\iota$ to $e'_0$. Besides, $e_0'$ inhibits
    $e$.  Therefore, there is a path between $e$ and $\hat e$.
  \end{itemize}
  The same proof holds switching $\tau$ and $\tau'$.

\end{proof}
