\begin{proposition*}[Property of the divergent activity]
  For $\tau$ a trace
  and $\iota$ an intervention, let $I = (t, t+\delta)$ be a time interval
  such that $\tau \cap I = \emptyset$ and $m_0 =
  \TSTATE{t}{\tau}$. Then, we have
  \[\CProb{ \ATRAJ{} \cap I = \emptyset }{ T=\tau,\
      \TSTATE{t}{\ATRAJ{}} = m\ }
    \ =\ e^{-\alpha'(m, m_0) \cdot \delta}.
  \]
\end{proposition*}
\begin{proof}

  Given that $\TSTATE{t}{\ATRAJ{}} = m$, no counterfactual event
  happens in time interval $I$ if and only if no potential event that
  is triggerable from mixture $m$ is scheduled in $I$. Therefore,
  \vskip 0.0cm
  \begin{equation}\label{eqn:scheduled-decomposition}
    \begin{aligned}
      &\CProb{ \ATRAJ{} \cap I = \emptyset } { T=\tau,\,
        \TSTATE{t}{\ATRAJ{}} = m }
      \\[0.4em]
      &= \quad \CProb{ \bigwedge_{s \vdash e} (\,e \notin \Sigma \cap
        I \,) }{ T=\tau }
      \\[0.4em]
      &= \quad \prod_{s \vdash e}\, \CProb{e \notin \Sigma \cap I }{
        T=\tau }.
    \end{aligned}
  \end{equation}
  \vskip 0.2cm Let $e$ a potential event such that $m \vdash e$. The
  probability that $e$ has not been scheduled in $I$ given that
  $T=\tau$ depends on whether or not $e$ is triggerable from mixture
  $m_0 = \TSTATE{t}{\tau}$. Indeed, we assumed that $\tau$ contains no
  event in time interval $I$. Therefore, if $m_0 \vdash e$, then $e$
  cannot be scheduled in $I$, without which it would have been
  observed in $\tau$.  Thus,
  \begin{equation}\label{eqn:pscheduled1}
    m_0 \vdash e \ \ \Longrightarrow\ \ \CProb{e \notin \Sigma \cap I}{T=\tau } = 1.
  \end{equation}
  Besides, if $m_0 \not\vdash e$, then the observation $\{ T=\tau \}$
  gives no information on whether or not $e$ has been scheduled in
  $I$. Because potential events are scheduled according Poisson
  processes, we have:
  \begin{equation}\label{eqn:pscheduled2}
    m_0 \not\vdash e \ \ \Longrightarrow\ \ \CProb{e \notin \Sigma \cap I}{
      T=\tau } = e^{-\lambda_e \delta}
  \end{equation} where $\lambda_e$ is the rate of the rule associated with potential event $e$.
  Combining (\ref{eqn:pscheduled1}) and (\ref{eqn:pscheduled2}) with
  equation (\ref{eqn:scheduled-decomposition}),
  \begin{equation*}
    \CProb{ \ATRAJ{} \cap I = \emptyset }
    { T=\tau,\,\TSTATE{t}{\ATRAJ{}} = m }
    = \ 
    \exp{ \! \Big ( \!- \hspace{-1em} \sum_{m \vdash e, \, m_0 \not\vdash
        e} \hspace{-0.9em} \lambda_e \cdot \delta \Big ) }
  \end{equation*}
  % \begin{equation*}
  %   \begin{aligned}
  %     & \CProb{ \ATRAJ{} \cap I = \emptyset } { T=\tau,\,
  %     \TSTATE{t}{\ATRAJ{}} = m }
  %     \\[-0cm]
  %     &= \
  %     \hspace{-0.5em} \prod_{m \vdash e, \,
  %     m_0 \not\vdash e} \!\!\! e^{-\lambda_e \delta}
  %     \ = \
  %     \exp{ \Big ( - \!\!\! \sum_{m \vdash e, \, m_0 \not\vdash
  %     e} \hspace{-0.8em} \lambda_e \ \cdot \ \delta \Big ) }
  %   \end{aligned}
  % \end{equation*}
  Finally, we can recognize the divergent activity in the exponential
  above: \vskip 0.0cm
  \begin{equation*}
    \begin{aligned}
      \sum_{m \vdash e, \, m_0 \not\vdash e} \hspace{-0.8em} \lambda_e
      \
      &= \ \sum_{(r, \xi)} \lambda_r \textbf{1}\{ m \vdash (r, \xi),\ m_o \not\vdash (r, \xi) \} \\
      &= \ \sum_r \lambda_r\sum_\xi \textbf{1}\{ m \vdash (r, \xi),\ ,m_o \not\vdash (r, \xi) \} \\
      &= \ \sum_r \lambda_r |\DEMBS{r}{m, m_0}| \ = \ \alpha'(m, m_0).
      % \\ &= \ \alpha'(m, m_0).
    \end{aligned}
  \end{equation*}
  \vskip 0.2cm
  \noindent This concludes the proof.
\end{proof}
