% -*- TeX-master: "ijcai18.tex" -*-

\section{Introduction}\label{sec:intro}

Rule-based modeling languages for molecular biology, such as Kappa
\cite{DanosEtAl-CONCUR07} and BioNetGen \cite{bngl}, or organic
chemistry, such as M{\o}d \cite{moll}, can be used to write
mechanistic models of complex reaction systems. These approaches
consider entities that have a structure, and a distinction is made
between the transformation of a structure fragment (a pattern),
specified by a rule, and the reaction resulting from the application
of the rule to a combination of entities contextualizing the
fragment. The structure of bio-molecular entities is conveniently
represented as a graph and a rule is a graph-rewrite directive with a
rate constant that determines its propensity to apply. The stochastic
simulation of a rule collection generates a time series of rule
applications---henceforth events---that might reach a state of
interest in processes like the assembly of a molecular machine, the
activation of a transcription factor, or the synthesis of a specific
chemical compound.

While rule-based models provide compactness, transparency, and the
ability of handling combinatorial complexity, the perhaps most
significant advantage lies in their suitability for causal
analysis. This is because such analysis proceeds at the level of rules
and not reactions, thereby avoiding contamination with context that
defines a reaction yet is accidental to the application of the
underlying rule. Due to the concurrent nature of events it is
typically far from obvious how a given series of events attained a
particular outcome. Biologists often refer to a causal account or
explanation as a ``pathway", but have no formal framing for it.

The approach to causal analysis provided in
\cite{DBLP:conf/fsttcs/DanosFFHH12,DanosEtAl-CONCUR07} takes advantage
of rule structure to
\begin{inparaenum}[(i)]
\item \label{step:compress} compress a simulation trace into a minimal
  subset of events that are necessary and jointly sufficient to
  replicate the outcome of interest and
\item \label{step:highlight} highlight causal influences between
  events, exposing the extent of concurrency.
\end{inparaenum}
Such analysis is usually performed on a large sample of traces to the
outcome, thus recovering the salient pathways as those that are
statistically favored by the dynamics. This approach, however, suffers
from two drawbacks. First, the focus on necessity in step
(\ref{step:compress}) neglects events that are kinetically critical
(in that they dramatically increase the probability of observing the
outcome), yet are not logically necessary for achieving it. Second,
step (\ref{step:highlight}) is limited to a narrow notion of causal
influence that we may call \emph{enablement}. Put simply, an event $a$
is said to enable event $b$, if $a$ modifies an aspect of the state of
the world in such a way as to directly permit $b$ to happen. This
positively tinted version of influence is blind to the ubiquitous role
of inhibitory interactions in molecular biology.  Indeed, an event $a$
may cause an event $b$ without (transitively) enabling it, but instead
by preventing another event $c$ that would have prevented
$b$. Clearly, uncovering such an explanatory narrative is challenging
because it involves events that may \emph{not} occur in a simulation
trace ($c$ in this case).

We here propose an approach based on counterfactual reasoning that
complements the existing causal analysis of event series generated
from rule-based models. In the tradition of Lewis, Pearl and Halpern,
we investigate possible causal influences by answering questions of
the kind: \textit{Had event $e_1$ not occurred, would event $e_2$ have
  happened?}
% \cite{lewis1974causation,pearl2009causality,halpern2016actual},
Our contributions are as follows.
\ifshort \begin{inparaenum}[(1)] \else \begin{enumerate} \fi
\item We provide a semantics for counterfactual statements in the
  context of rule-based models, where the standard definition of
  counterfactuals based on structural equations
  \cite{pearl2009causality,halpern2016actual} does not apply.
\item We show how such statements can be evaluated by sampling
  \emph{counterfactual traces} that are meant to probabilistically
  ``hug" a given (factual) trace as much as an external intervention
  permits them to. To this end, we introduce an algorithm to generate
  counterfactual traces and provide an efficient implementation for
  the Kappa language.
\item We show how counterfactual dependencies between events can be
  systematically explained in terms of enablement and prevention
  relations that are more in line with biological reasoning.
\ifshort \end{inparaenum} \else \end{enumerate} \fi
