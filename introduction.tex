\section{Introduction}


Rule-based modeling languages for molecular biology, 
such as Kappa \cite{DanosEtAl-CONCUR07} and BioNetGen \cite{bngl},
or organic chemistry, such as M{\o}d \cite{moll}, can be used to write 
mechanistic models of complex reaction systems. 
In these approaches, chemical transformations are represented by 
local graph-rewrite rules equipped with stochastic firing rates. 
In a dynamical simulation, rules induce a time series of events that might reach
a state of interest in processes like the assembly of a molecular machine, 
the activation of a transcription factor, 
or the synthesis of a specific chemical compound. 

While rule-based models provide compactness, transparency, 
and the ability of handling combinatorial complexity, 
the perhaps most significant advantage lies in their suitability for 
causal analysis that takes into account the logically 
concurrent nature of interactions. 
The causal analysis \cite{DBLP:conf/fsttcs/DanosFFHH12,DanosEtAl-CONCUR07}
of event series generated by such models provides a formal definition of 
``pathway" and a means for revealing the emergence of pathways from low-level 
interactions. These methods take advantage of rule structure to
\begin{inparaenum}[(i)]
\item compress a given simulation trace into a minimal subset of events 
that are necessary and jointly sufficient to replicate 
a phenomenon of interest and 
\item highlight the direct causal influences between events,
exposing the extent of concurrency. 
\end{inparaenum}
We propose a distinct but complementary approach based on 
\textit{counterfactual reasoning} that improves causal explanations by 
\begin{inparaenum}[(i)]
\item being more sensitive to kinetics and 
\item properly accounting for the causal impact from inhibition between events.
\end{inparaenum}