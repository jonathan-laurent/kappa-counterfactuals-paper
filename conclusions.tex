\section*{Conclusions}

This paper proposes a methodology (and implementation) for giving meaning to counterfactual statements in the spirit of approaches based on structural equation models (SEMs) but in contexts where SEMs are not directly applicable. The case in point are rule-based models deployed with the goal of studying how a specific set of combinatorial and concurrent interactions underlie a complex process. Such problems are rampant in molecular systems biology. 

Structural equations represent explicitly (for example in terms of a
factor graph) aspects of the causal structure of a system. Yet, for
rule-based models that structure is implicit and has first to be
revealed. We argue that counterfactual statements play an important
role to this end. The challenge, then, is to give a semantics to
counterfactual statements when, unlike in SEM approaches, no explicit
causal structure is available to begin with. We break that impasse by
proposing a technique we call counterfactual simulation that is
cognate to the idea of counterfactual experiments in SEM. Using
counterfactual simulation, we show that causally explanatory diagrams
involving relations of enablement and prevention between events can be
constructed.

Our method leaves open many questions regarding best practices, which
we hope to address in future work. Such issues depend on the
availability of empirically meaningful models. On a more conceptual
side, we wonder whether explanatory diagrams, such as
Figure~\ref{fig:cex}, constructed via counterfactual experiments as we
define them here might constitute a basis for obtaining structural
equations from complex models. These equations would always be
relative to an outcome of interest defined at the outset. Replacing a
rule-based model with such equations could enable targeted statistical
analysis to estimate model parameters, for which simulations would be
too expensive.
