% -*- TeX-master: "ijcai18.tex" -*-

\section*{Conclusions and future work}

This paper proposes a methodology and implementation for giving
meaning to counterfactual statements in stochastic rule-based
graph-rewrite systems that have found application in molecular systems
biology, although our method is not specific to such systems. A
widespread approach to the analysis of counterfactuals is based on
structural equation models (SEMs), which encode the causal structure
of a system. The models we consider consist of many concurrently
applicable rules that encode many local patterns of action, such as
individual protein-protein interactions or chemical transformations. A
graph representing the dependencies among all possible molecular
observables in such a system would be of forbidding combinatorial
complexity.

Our approach defines a semantics for counterfactuals in such systems
that is nonetheless very much along the lines of Halpern and Pearl
\cite{pearl2009causality,halpern2016actual}, despite the absence of
explicit structural equations. We proceed by modifying the
continuous-time Monte Carlo algorithm that generates traces in
rule-based models in such a way as to sample counterfactual
trajectories that stay probabilistically as close to the original
(factual) trajectory as an intervention causing divergence permits
them to be. We then show how to construct causal diagrams that explain
counterfactual dependencies in terms of enablement and prevention
relations between events. Enablement is a standard causal relation
between events within factual or within counterfactual traces, whereas
prevention shows up as a relation between pairs of events, one in the
factual and the other in the counterfactual trace. In particular,
prevention can involve events that were not observed in the factual
trace. This results in explanatory diagrams that resonate more with
the ubiquitous presence of inhibitory interactions in biology and that
can capture subtle kinetic aspects of rule-based models.  Finally, our
completeness result according to which any counterfactual dependency
can be ``explained'' in terms of enablement and prevention not only
increases our confidence in our account of counterfactuals, but also
connects two visions of causality that are typically treated
disjointly (see the comment after Theorem~\ref{thm:completeness}).


Despite a sound theoretical foundation and an effective implementation
based on a state-of-the-art simulator for rule-based models
\cite{DanosEtAl-APLAS07,BoutillierEK17}, our technique is in need of
rigorous practical assessment using large-scale models. However, such
evaluation requires first addressing difficult practical
questions. Most notably, which counterfactual experiments are worth
trying? It is entirely unclear a priori which interventions are
informative for traces that include many millions of events. At
present, we can only offer tentative directions for future study.

One way to proceed is by developing heuristics, such as identifying
correlations between events in samples of factual traces (or in a
single long trace). In our toy model, the occurrence of $p$ is often
preceded by the occurrence of $pk$. This correlation, together with
the absence of $pk$ from some initial causal account---such as the one
presently achievable in a fully automated fashion based on enablement
alone, (Figure~\ref{fig:dumb-story})---suggests to try a
counterfactual experiment on $pk$. More generally, if
\begin{inparaenum}[(i)]
\item a context $\mathcal C$ in which an event $e$ occurs is
  frequently more specific than is required by the left-hand side of
  the underlying rule and
\item this observation cannot be explained by the current causal
  narrative,
\end{inparaenum} then a counterfactual experiment in which we block
the last event responsible for at least part of $\mathcal C$ seems
worthwhile in order to assess whether the current causal narrative
needs to be updated.  Another important practical questions is whether
interventions should block events punctually or ``knock out'' the
associated potential event for a defined timeframe (our framework
accomodates both approaches). 

On a more conceptual side, we are investigating principled ways to
``glue'' together explanatory diagrams (such as Figure~\ref{fig:cex})
corresponding to different counterfactual experiments in a single
``summary diagram'' that would synthetize the current understanding of
the causal structure of a system relative to an outcome of
interested. More speculatively, we wonder whether such a summary
diagram might constitute a basis for obtaining approximate structural
equations from complex models. Replacing a rule-based model with such
equations could enable targeted statistical analysis to estimate model
parameters, for which simulations would be too expensive.

\ifreview
\ifincludeappendices
\else
%\medskip
\paragraph{Note to the reviewers}
A version of this paper with appendices containing the proofs of
Proposition~\ref{prop:div-activity} and Theorem~\ref{thm:completeness}
along with a benchmark of our implementation of counterfactual
resimulation on a scaled-up version of our example model can be found
following this
\underline{\href{https://www.dropbox.com/sh/2fwji0its0o0ciq/AABfLZ-GO2wCE2x3h3ulUbB-a?dl=0}{link}}
(anonymous).
%\url{http://www.lamsade.dauphine.fr/~lang/IJCAI-ECAI-2018/FAQ-authors.html}
\fi
\fi