% -*- TeX-master: "ijcai18.tex" -*-

\newcommand{\PCFST}[0]{\ProbParen{\CFST{}}}

\newcommand{\ItAbduction}[0]{(\textit{abduction})}
\newcommand{\ItAction}[0]{(\textit{action})}
\newcommand{\ItPrediction}[0]{(\textit{prediction})}


\section{Evaluating Counterfactual
  Statements}\label{sec:counterfactual}

In our example, the counterfactual statement to be
assessed is: ``Had $pk$ not happened, $p$ would not have happened."
Our account in the previous section suggests that $pk$ played a role,
but it is also clear that given the stochastic nature of rule firing
$p$ could well have happened even in the absence of $pk$; it just is
unlikely. In a stochastic setting, counterfactual statements are not
either true or false, but have degrees of likelihood. To assess that
likelihood is our task.

Given an original (factual) trace $\tau$, a naive approach might be to sample
counterfactual traces, each of which starts with the state of the system
attained in $\tau$ just before event $pk$ happened, but in which we skip over
$pk$ and then run an unconstrained simulation from that point onward. In this
approach traces would quickly diverge from the original, distorting the causal
role that $pk$ played specifically in it. The question here is not what causal
role $pk$ can play in principle, but what role it actually did play in $\tau$.
Counterfactual statements are undetachable from the context in which they are
formulated.

Pearl's standard account of counterfactuals \cite{pearl2009causality}
is based on performing \textit{``surgical interventions''} on a
structural equation model (SEM). A SEM features a finite sequence
$(x_1, \dots, x_n)$ of variables, each associated to a
\emph{functional equation} of the form
$x_i = f_i(x_1, \dots, x_{i-1}, u_i)$, where $f_i$ is a deterministic
function and $u_i$ a random variable. Ideally, each $f_i$ defines an
independent and autonomous physical mechanism. This is partially
enforced by the requirement that the $u_i$ must be mutually
independent. Given some observation $e$, the probability of the
counterfactual statement ``had $x_j$ been equal to $a$, $\psi$ would
have been true'' is evaluated following a three-steps process:
\begin{inparaenum}[]
\item \ItAbduction{} compute the distribution $p_e$ of values for
  $\vec u$ given observation $e$, then
\item \ItAction{} intervene in the model by replacing the defining
  equation for $x_j$ by ``$x_j = a$'' and finally
\item \ItPrediction{} compute the probability that $\psi$ is true in
  this new model when $\vec{u}$ is distributed according to $p_e$.
\end{inparaenum}

Because of their combinatorial nature, rule-based models do not have a
natural encoding in terms of structural equations. It is, therefore,
not clear how Pearl's construction would apply.

\subsection{A Semantics for Counterfactuals}
\label{subsec:counterfactuals-semantics}

An intervention $\iota$ prevents the occurrence of selected events and
is defined by a predicate $\BLOCKED{\iota}{\cdot}$ ranging over
events. \longversion{(Our framework can accomodate a much broader
  range of interventions, but we make this restriction for ease of
  exposition.)}  Given a predicate $\psi$ over traces, we write the
statement \textit{``Had intervention $\iota$ happened in trace $\tau$,
  $\psi$ would have been true''} as $\CFST{}$.

To assign a probability to this statement, it is useful to
reconceptualize the CTMC induced by a model so as to isolate the
source of randomness from everything else. This can be done as
follows:
\begin{inparaenum}[(i)]
\item Consider all possible event templates $(r, \xi)$, where $\xi$
  maps into a large enough set of global identifiers. For each event
  template, imagine a Poisson process in which a bell rings at time
  intervals drawn independently from an exponential distribution with
  parameter $\lambda_r$. These Poisson processes are all gathered in a
  random variable $\Sigma$ that we call \emph{schedule}. It features
  the sequence of ringing times for every bell and plays the same role
  as the variable $\vec{u}$ in a SEM.
\item A simulation trace can be viewed as the following deterministic
  function $T(\Sigma)$ of the random variable $\Sigma$: starting with
  the initial mixture and moving through time, whenever a bell rings,
  its associated event template $e$ is realized by transforming the
  current mixture $m$ if $\TRIGGERABLE{m}{e}$. For example, if the
  current mixture $m$ is given as in Figure~\ref{fig:mixture} and the
  bell linked to \textit{``apply rule $b$ on substrate $3$ and kinase
    $4$''} rings, a bond is created between these two agents. In
  contrast, the bell linked to \textit{``apply rule $b$ on substrate
    $1$ and kinase $2$'}' would have no effect. We write $T(\sigma)$
  for the trace generated from schedule $\sigma$.
\end{inparaenum}

We extend this viewpoint to include interventions. For an intervention
$\iota$, we define the altered trace $\ATRAJ{}$ much in the same way
as $T$, but also requiring $\BLOCKED{\iota}{(e, t)}$ to be false for
$e$ to be realized when its bell rings at $t$.  Given an observed
trace $\tau$, an intervention $\iota$ and a predicate $\psi$, the
probability of $\CFST{}$ can now be computed according to Pearl's
three-steps strategy: \ItAbduction{} condition the distribution of
$\Sigma$ by the observation that $T=\tau$, then \ItAction{} alter the
behavior of the simulation with intervention $\iota$ and
\ItPrediction{} compute the probability of $\psi$ in the resulting
setting. This results in the following definition.

\begin{definition}[Semantics of counterfactual statements]\label{def:counterfactuals}
  For $\tau$ an observed trace, $\iota$ an intervention and $\psi$ a
  predicate on traces, the probability of the counterfactual statement
  \textit{``had intervention $\iota$ happened in trace $\tau$,
    predicate $\psi$ would have been true''} is defined as:
  \[ \PCFST{} \eqdef \ \ProbParen{\psi(\ATRAJ{}) \ |\ T = \tau}. \]
\end{definition}

\subsection{The Counterfactual Resimulation Algorithm}
\label{subsec:cosim-algo}

Following Definition~\ref{def:counterfactuals}, we estimate the
probability of the counterfactual statement $\CFST{}$ by sampling
instances of the random variable $\CTRAJ{}$. Such instances are called
\emph{counterfactual traces}. Intuitively, they give an account of
what else trace $\tau$ could have been, had intervention $\iota$
happened.  We introduce Algorithm~\ref{alg:cosimulation}, a variation
of the Doob-Gillespie algorithm, to sample a counterfactual trace
efficiently given a reference trace $\tau$ and an intervention
$\iota$. We call it \emph{counterfactual resimulation}, since it works
by going through every event of $\tau$, resimulating only those parts
of $\tau$ that are affected by $\iota$. In particular, when $\iota$ is
the trivial intervention ($\BLOCKED{\iota}{\cdot} = \text{false}$), it
returns $\tau$.

This algorithm relies on a modified notion of activity we call
\emph{divergent activity}. We define the set of \emph{divergent
  embeddings} of the left-hand side of a rule $r$ into mixture $m$ and
relative to $m_0$ as
% \tryinline{\DEMBS{r}{m, m_0} \eqdef \EMBS{r}{m} \setminus
% \EMBS{r}{m_0}.}
$\DEMBS{r}{m, m_0} \eqdef \EMBS{r}{m} \setminus \EMBS{r}{m_0}.$
Equivalently, a divergent embedding is an embedding whose codomain
features a \emph{divergent site}, that is, a site whose state differs
across $m$ and $m_0$. The {divergent activity} of a rule $r$ in
mixture $m$ relative to $m_0$ is then the product
$\lambda_r|\DEMBS{r}{m, m_0}|$. The \emph{total divergent activity} of
the system, $\alpha'(m, m_0)$, is the sum of all divergent
activities. Finally, we use the notation $\TSTATE{t}{\tau}$ to refer
to the mixture at time $t$ in $\tau$\longversion{, which is obtained
  from the initial mixture after updating it for each event in turn up
  to time $t$ in $\tau$}.

\newcommand{\EVF}[0]{e_{\text{f}}}
\newcommand{\EVCF}[0]{e_{\text{c}}}

\begin{algorithm}
\caption{Counterfactual resimulation}\label{alg:cosimulation}
\begin{spacing}{1.3}
\algsetup{indent=1.5em}
\begin{algorithmic}[1]
\vspace{0.2cm}
\STATE $t \gets 0$
\STATE $m \gets\ $ initial mixture
\WHILE{ $t < t_\text{\,end}$ }
  \STATE $m_0 \gets \TSTATE{t}{\tau}$
  \STATE $(\EVF{}, t_{\text{f}}) \gets $ first event of $\tau$ in time interval $(t, \infty)$
  \vspace{0.1cm}
  \STATE $\alpha' \gets \sum_r {\lambda_r |\DEMBS{r}{m, m_0}|}$
  \vspace{0.1cm}
  \STATE draw $\delta \sim \textsc{Exp}(\alpha') $
  \STATE $t_{\text{c}} \gets t + \delta$
  % \STATE
  \IF { $t_{\text{c}} < t_{\text{f}}$ }
      \STATE draw a rule $r$ with prob.
      $\propto \, \lambda_r |\DEMBS{r}{m, m_0}|$
      \STATE  draw a divergent embedding $\xi \in \DEMBS{r}{m, m_0}$
      \STATE {$e \gets (r, \xi)$}
      \IF {$ \neg \, \BLOCKED{\iota}{e, t_{\text{c}}} $ }
          \STATE $t \gets t_{\text{c}}$
          \STATE $m \gets \UPDATE{m}{e}$
          \STATE log event $(e, t)$
      \ENDIF
  \ELSE
      \STATE {$e \gets \EVF{}$}
      \STATE $t \gets t_{\text{f}}$
      \IF {$ \neg \, \BLOCKED{\iota}{e, t} $ \AND $ \TRIGGERABLE{m}{e} $ }
          \STATE $m \gets \UPDATE{m}{e}$
          \STATE log event $(e, t)$
      
      \ENDIF 
  \ENDIF
\ENDWHILE
\vspace{0.1cm}
\end{algorithmic}
\end{spacing}
\end{algorithm}

The role and relevance of the concept of divergent activity in
counterfactual resimulation is summarized by the following
proposition, where $\tau \cap I = \emptyset$ is a shortcut for ``no
event of trace $\tau$ occurs in the time interval $I$''.
\begin{proposition}[Property of the divergent
  activity]\label{prop:div-activity}
  For $\tau$ a trace and $\iota$ an intervention, let
  $I = (t, t+\delta)$ be a time interval such that
  $\tau \cap I = \emptyset$ and $m_0 = \TSTATE{t}{\tau}$. Then, we
  have
  \[\CProb{ \ATRAJ{} \cap I = \emptyset }{ T=\tau,\
      \TSTATE{t}{\ATRAJ{}} = m\ }
    \ =\ e^{-\alpha'(m, m_0) \cdot \delta}.
  \]
\end{proposition}
\noindent At every iteration of Algorithm~\ref{alg:cosimulation}, the
divergent activity $\alpha'$ determines the probability that an event
happens in the counterfactual trace prior to the next event in the
factual trace $\tau$ (test of line \ref{cosim:cev}).  A proof of
Proposition~\ref{prop:div-activity} is given in
Appendix~A. %\ref{ap:div-activity}.
It is the main step in establishing:

\begin{theorem}%[Correctness of counterfactual resimulation]
  The counterfactual resimulation algorithm correctly
  samples instances of $\,\CTRAJ{}$.
  %as defined in
  %section~\ref{subsec:counterfactuals-semantics}.
\end{theorem}

% -*- TeX-master: "ijcai18.tex" -*-

\subsection{Implementation}\label{subsec:implementation}

There are two challenges in efficiently implementing counterfactual
resimulation. The first is a suitable representation for the sets of
divergent embeddings $\DEMBS{r}{m}$ to minimize the cost of their
update at each iteration. Since the Kappa simulator solves exactly
that problem for the sets of embeddings $\EMBS{r}{m}$
\cite{DanosEtAl-APLAS07}, we leverage most of that infrastructure. The
second consists in avoiding excessively many iterations of
Algorithm~\ref{alg:cosimulation} in which time is advanced in tiny
increments and the proposed event is rejected.
%(line~\ref{cosim:blocked}). 
Suppose, for example, that in our toy model
$pk$ has a very high firing rate and we wish to block, from a
specific time onward, \emph{all} events in which the sole kinase
becomes phosphorylated. Upon blocking one occurrence of the event, the
same event would want to happen again, and we would keep rejecting it
a huge number of times until a different rule fires. More generally,
event templates whose realization is bound to be blocked should be
removed efficiently before their realization is attempted and not be
counted in the system's divergent activity. We solve this problem
for a class of interventions we call \emph{regular}.
Specifically, an intervention $\iota$ is regular if
the predicate $\BLOCKED{\iota}{((r, \xi), t)}$ can be expressed as a
finite disjunction of formulae of the form
$(r \!=\! r') \wedge F(\xi{\restriction_{c}}) \wedge (t \!\in\! I)$ or
$G(r, \xi) \wedge (t \!=\! t')$, where $r'$ is a rule, $t'$ a time, $I$ a
time interval, $\xi{\restriction_{c}}$ the restriction of $\xi$ to a
single connected component $c$ of $L_{r'}$, and $F, G$ arbitrary
predicates. For regular interventions, our implementation is
guaranteed to either produce or consume an event at each iteration.

\begin{proposition}
  Sampling a counterfactual trace for a \emph{regular} intervention
  can be done in time $\mathcal{O}(n \cdot r \log|m|)$, where $n$ is
  the sum of the number of events in the reference trace and in the
  resulting counterfactual trace, $r$ is the number of rules in the
  model and $|m|$ the size of the reaction mixture.
\end{proposition}

We provide a benchmark of our implementation on a scaled-up version of
our toy model in Appendix~\ref{ap:benchmark}. The average slowdown per
event compared to the Kappa simulator does not exceed 50\% for a
variety of interventions.

\medskip

Returning to our running example, sampling counterfactual traces repeatedly for
\RefTrace{} would reveal that, with very high probability, ``event $p$ would not
have happened, had $pk$ not happened''. However, we can go further by using
counterfactual traces to \textit{explain} this observation using enablement and
prevention arrows.
