\section{Counterfactual simulation}\label{sec:counterfactual}

Counterfactual statements are tricky because their truth is
context-dependent. The statement ``Had it not rained, the driver might
have arrived earlier" can fail to be true in many ways. Intuitively,
how easily the driver could have arrived earlier depends on how great
a departure from actuality is required for it to be case
\cite{Lewis1973}. This is why counterfactual reasoning is tied to
modal logic. The standard approach is to require that the consequent
in a counterfactual be true in some of those possible worlds (in which
the antecedent holds) that \textit{are most similar to the actual
  world}. If the counterfactual statement is true in all of these
worlds, we can replace ``might" with ``would".  We now operationalize
this approach in the context of Kappa traces and interpolate between
``might" and ``would" using probabilities.

\subsection{A semantics for counterfactuals}

We start by formalizing the notion of an intervention. An intervention
$\iota$ (``blocking $pk$" in our example) is a predicate
$\BLOCKED{\iota}{t, e}$ that determines whether or not event $e$ is
blocked at time $t$. Given a predicate $\varphi$ over traces, we write
the proposition \textit{``Had intervention $\iota$ happened in trace
  $\tau$, $\varphi$ would have been true with probability greater than
  $p \in [0,1]$''} as:
\[ \tau \models_p [\iota] \, \varphi.
\]

To give an operational meaning to this statement, we invoke the
continuous time Markov chain (CTMC) semantics of a Kappa model as
defined and implemented in
\cite{DanosEtAl-APLAS07,BoutillierEK17}. For the present purpose it is
conceptually useful to think of a CTMC abstractly in terms of the
random realization of ``potential events". A potential event is a pair
$(r, \xi)$ where $r$ is a rule and $\xi$ an injective mapping from
local agents involved in $r$ to global agents in a huge virtual
mixture of many instances of all possible molecular
species.\footnote{For the sake of simplicity, we assume that no agent
  is created or deleted by a rule.} For every such potential event, we
imagine a bell that rings at a time $t$ drawn from an exponential
distribution $\lambda_r\exp(-\lambda_r t)$, where $\lambda_r$ is the
stochastic rate constant of $r$. A simulation trace can be viewed as
the realization of a random variable $T$ determined by the set
$\omega$ of ring times: Starting with an initial mixture, when a bell
rings at $t$, its associated potential event $(r, \xi)$ transforms the
mixture according to $r$ if $\xi$ yields a valid embedding of the left
hand side of $r$ in the current mixture and time advances by
$t$. Otherwise, time advances and nothing happens---a null
event. Repeat on the resulting mixture.

We can extend this viewpoint to include interventions. For an
intervention $\iota$, we define the random variable $\ATRAJ{}$ much in
the same way as $T$, except that each time the bell rings, we require
$\BLOCKED{\iota}{t, e}$ to be false for the potential event
$e=(r, \xi)$ to be considered.  Counterfactual traces that are closest
to the actual trace $\tau$ are then sampled by generating realizations
of $\ATRAJ{}$ that inherit, whenever possible, the subset of $\omega$
that made up $\tau$. 
% An efficient implementation of this specification
% for sampling the conditional random variable $\CTRAJ{}$ is available at
% \begin{center}
%   \url{https://github.com/jonathan-laurent/kappa-counterfactuals}.
% \end{center} 
We refer to this natural extension of CTMC semantics as
\textit{counterfactual re-simulation} or \textit{co-simulation} for
short. Using co-simulation, we can operationalize the counterfactual
statements as follows.

\begin{definition}[Semantics of counterfactual statements] We write
  $\tau \models_p [\iota] \, \varphi$ the counterfactual statement
  \textit{``had intervention $\iota$ happened in trace $\tau$,
    predicate $\varphi$ would have been true with probability greater
    than $p$''}.  It is defined as follows:
  \[ \tau \models_p [\iota] \, \varphi \quad \Longleftrightarrow \quad
    \mathbf{P}( \varphi(\ATRAJ{}) \ |\ T = \tau) \,\geq\, p \]
\end{definition}

\subsection{Sampling counterfactual traces}

Counterfactual traces can be sampled using the algorithm described in
Listing~\ref{alg:cosimulation}.

\newcommand{\EVF}[0]{e_{\text{f}}}
\newcommand{\EVCF}[0]{e_{\text{c}}}

\begin{algorithm}
\caption{Counterfactual resimulation}\label{alg:cosimulation}
\begin{spacing}{1.3}
\algsetup{indent=1.5em}
\begin{algorithmic}[1]
\vspace{0.2cm}
\STATE $t \gets 0$
\STATE $m \gets\ $ initial mixture
\WHILE{ $t < t_\text{\,end}$ }
  \STATE $m_0 \gets \TSTATE{t}{\tau}$
  \STATE $(\EVF{}, t_{\text{f}}) \gets $ first event of $\tau$ in time interval $(t, \infty)$
  \vspace{0.1cm}
  \STATE $\alpha' \gets \sum_r {\lambda_r |\DEMBS{r}{m, m_0}|}$
  \vspace{0.1cm}
  \STATE draw $\delta \sim \textsc{Exp}(\alpha') $
  \STATE $t_{\text{c}} \gets t + \delta$
  % \STATE
  \IF { $t_{\text{c}} < t_{\text{f}}$ }
      \STATE draw a rule $r$ with prob.
      $\propto \, \lambda_r |\DEMBS{r}{m, m_0}|$
      \STATE  draw a divergent embedding $\xi \in \DEMBS{r}{m, m_0}$
      \STATE {$e \gets (r, \xi)$}
      \IF {$ \neg \, \BLOCKED{\iota}{e, t_{\text{c}}} $ }
          \STATE $t \gets t_{\text{c}}$
          \STATE $m \gets \UPDATE{m}{e}$
          \STATE log event $(e, t)$
      \ENDIF
  \ELSE
      \STATE {$e \gets \EVF{}$}
      \STATE $t \gets t_{\text{f}}$
      \IF {$ \neg \, \BLOCKED{\iota}{e, t} $ \AND $ \TRIGGERABLE{m}{e} $ }
          \STATE $m \gets \UPDATE{m}{e}$
          \STATE log event $(e, t)$
      
      \ENDIF 
  \ENDIF
\ENDWHILE
\vspace{0.1cm}
\end{algorithmic}
\end{spacing}
\end{algorithm}
\begin{theorem} Let $\tau$ a trace and $\iota$ an intervention. Let
  $I = (t, t+\delta)$ a time interval such that
  $\tau \cap I = \emptyset$. Then,
  \[\CProb{ \ATRAJ{} \cap I = \emptyset }{ T=\tau,\
      \TSTATE{t}{\ATRAJ{}} = m\ }
    \ =\ e^{-\alpha'(m, m_0) \cdot \delta}
  \]
  where $m_0 = \TSTATE{t}{\tau}$.
\end{theorem}
\begin{proof}

  Given that $\TSTATE{t}{\ATRAJ{}} = m$, no counterfactual event
  happens in time interval $I$ if and only if no potential event that
  is triggerable from mixture $m$ is scheduled in $I$. Therefore,
  \vskip 0.0cm
  \begin{equation}\label{eqn:scheduled-decomposition}
    \begin{aligned}
      &\CProb{ \ATRAJ{} \cap I = \emptyset } { T=\tau,\,
        \TSTATE{t}{\ATRAJ{}} = m }
      \\[0.4em]
      &= \quad \CProb{ \bigwedge_{s \vdash e} (\,e \notin \Sigma \cap
        I \,) }{ T=\tau }
      \\[0.4em]
      &= \quad \prod_{s \vdash e}\, \CProb{e \notin \Sigma \cap I }{
        T=\tau }.
    \end{aligned}
  \end{equation}
  \vskip 0.2cm Let $e$ a potential event such that $m \vdash e$. The
  probability that $e$ has not been scheduled in $I$ given that
  $T=\tau$ depends on whether or not $e$ is triggerable from mixture
  $m_0 = \TSTATE{t}{\tau}$. Indeed, we assumed that $\tau$ contains no
  event in time interval $I$. Therefore, if $m_0 \vdash e$, then $e$
  cannot be scheduled in $I$, without which it would have been
  observed in $\tau$.  Thus,
  \begin{equation}\label{eqn:pscheduled1}
    m_0 \vdash e \ \ \Longrightarrow\ \ \CProb{e \notin \Sigma \cap I}{T=\tau } = 1.
  \end{equation}
  Besides, if $m_0 \not\vdash e$, then the observation $\{ T=\tau \}$
  gives no information on whether or not $e$ has been scheduled in
  $I$. Because potential events are scheduled according Poisson
  processes, we have:
  \begin{equation}\label{eqn:pscheduled2}
    m_0 \not\vdash e \ \ \Longrightarrow\ \ \CProb{e \notin \Sigma \cap I}{
      T=\tau } = e^{-\lambda_e \delta}.
  \end{equation}
  Combining (\ref{eqn:pscheduled1}) and (\ref{eqn:pscheduled2}) with
  equation (\ref{eqn:scheduled-decomposition}),
 \begin{equation*}
       \CProb{ \ATRAJ{} \cap I = \emptyset }
       { T=\tau,\,\TSTATE{t}{\ATRAJ{}} = m }
       = \ 
      \exp{ \! \Big ( \!- \hspace{-1em} \sum_{m \vdash e, \, m_0 \not\vdash
      e} \hspace{-0.9em} \lambda_e \cdot \delta \Big ) }
  \end{equation*}
  % \begin{equation*}
  %   \begin{aligned}
  %      & \CProb{ \ATRAJ{} \cap I = \emptyset } { T=\tau,\,
  %       \TSTATE{t}{\ATRAJ{}} = m }
  %     \\[-0cm]
  %     &= \
  %     \hspace{-0.5em} \prod_{m \vdash e, \,
  %       m_0 \not\vdash e} \!\!\! e^{-\lambda_e \delta}
  %     \ = \ 
  %     \exp{ \Big ( - \!\!\! \sum_{m \vdash e, \, m_0 \not\vdash
  %     e} \hspace{-0.8em} \lambda_e \ \cdot \ \delta \Big ) }
  %   \end{aligned}
  % \end{equation*}
  Finally, we can recognize the divergent activity in the exponential
  above:
  \vskip 0.0cm
  \begin{equation*}
    \begin{aligned}
      \sum_{m \vdash e, \, m_0 \not\vdash e} \hspace{-0.8em} \lambda_e \ 
      &= \ \sum_{(r, \xi)} \lambda_r \textbf{1}\{ m \vdash (r, \xi),\ m_o \not\vdash (r, \xi) \} \\
      &= \ \sum_r \lambda_r\sum_\xi \textbf{1}\{ m \vdash (r, \xi),\ ,m_o \not\vdash (r, \xi) \} \\
      &= \ \sum_r \lambda_r |\DEMBS{r}{m, m_0}|
      \ = \ \alpha'(m, m_0).
      %\\ &= \ \alpha'(m, m_0).
    \end{aligned}
  \end{equation*}
  \vskip 0.2cm
  \noindent This concludes the proof.
\end{proof}