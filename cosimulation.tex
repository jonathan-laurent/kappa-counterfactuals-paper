% -*- TeX-master: "ijcai18.tex" -*-

\newcommand{\PCFST}[0]{\ProbParen{\CFST{}}}

\newcommand{\ItAbduction}[0]{(\textit{abduction})}
\newcommand{\ItAction}[0]{(\textit{action})}
\newcommand{\ItPrediction}[0]{(\textit{prediction})}


\section{Evaluating Counterfactual
  Statements}\label{sec:counterfactual}

In our example, the counterfactual statement to be assessed is: ``Had
$pk$ not happened, $p$ would not have happened."  Our account in the
previous section suggests that $pk$ played a role, but it is also
clear that given the stochastic nature of rule firing $p$ could well
have happened even in the absence of $pk$; it just is unlikely. In a
stochastic setting, counterfactual statements are not either true or
false, but have degrees of likelihood. To assess that likelihood is
our task.

Given an original (factual) trace $\tau$, a naive approach might be to
sample counterfactual traces, each of which starts with the state of
the system attained in $\tau$ just before event $pk$ happened, but in
which we skip over $pk$ and then run an unconstrained simulation from
that point onward. In this approach traces would quickly diverge from
the original, distorting the causal role that $pk$ played specifically
in it. The question here is not what causal role $pk$ can play in
principle, but what role it actually did play in $\tau$.
Counterfactual statements are undetachable from the context in which
they are formulated.

Pearl's standard account of counterfactuals \cite{pearl2009causality}
is based on performing \textit{``surgical interventions''} on a
structural equation model (SEM). A SEM features a finite sequence
$(x_1, \dots, x_n)$ of variables, each associated to a
\emph{functional equation} of the form
$x_i = f_i(x_1, \dots, x_{i-1}, u_i)$, where $f_i$ is a deterministic
function and $u_i$ a random variable. Ideally, each $f_i$ defines an
independent and autonomous physical mechanism. This is partially
enforced by the requirement that the $u_i$ must be mutually
independent. Given some observation $e$, the probability of the
counterfactual statement ``had $x_j$ been equal to $a$, $\psi$ would
have been true'' is evaluated following a three-step process:
\begin{inparaenum}[]
\item \ItAbduction{} compute the distribution $p_e$ of values for
  $\vec u$ given observation $e$, then
\item \ItAction{} intervene in the model by replacing the defining
  equation for $x_j$ by ``$x_j = a$'' and finally
\item \ItPrediction{} compute the probability that $\psi$ is true in
  this new model when $\vec{u}$ is distributed according to $p_e$.
\end{inparaenum}

Because of their dynamic nature, rule-based models are not readily
expressible in terms of structural equations. However, Pearl's
construction generalizes to our setting, assuming a \emph{structural}
refinement of Kappa's probabilistic semantics where deterministic
causal mechanisms are separated from stochastic aspects.

\iffalse provided that we separate the deterministic causal mechanisms
of a model from its stochastic aspects. This means reinterpreting a
temporal trace as a deterministic function of a separately specifiable
series of mutually independent stochastic choices, allowing us to
reuse the randomness inherent to a specific actual trace in the
construction of counterfactual versions induced by intervention.  \fi

\iffalse Because of their dynamic nature, rule-based
models do not have structural equations that are readily available at
the outset. However, Pearl's construction generalizes to our setting,
provided that we separate the randomness of simulation from its
deterministic causal mechanisms. \fi

%\subsection{A Refinement of Kappa's CTMC Semantics}
%\subsection{A Structural Account of Kappa}
\subsection{A Refined Semantics for Kappa}
\label{subsec:semantics-refinement}

Answering counterfactual questions about a dynamical system requires
more than a probability distribution over its traces. It requires a
structural portrayal of this system in terms of stable and
deterministic causal mechanisms that can be altered separately (the
$f_i$ in SEMs). Besides, for Pearl's three-step construction to
apply, randomness must be isolated in a collection of processes that
are blind to the system's observable state and history. We propose
such a decomposition for Kappa models. It is motivated by a standard
construction in physics that justifies the use of the Doob-Gillespie
algorithm for simulating reaction systems \cite{gillespie1977exact}.

% The CTMC semantics of Kappa that is provided in
% section~\ref{sec:background}

Our refined semantics reconceptualizes the CTMC induced by a model as
follows:
\begin{inparaenum}[(i)]
\item Consider all possible event templates $(r, \xi)$, where $\xi$
  maps into a large enough set of global identifiers. For each event
  template, imagine a Poisson process in which a bell rings at time
  intervals drawn independently from an exponential distribution with
  parameter $\lambda_r$. These Poisson processes are all gathered in a
  random variable $\Sigma$. A realization of $\Sigma$ is called
  a \emph{schedule} and it features a sequence of ringing times
  for every bell.
\item With every schedule $\sigma$, we associate a unique trace
  $\DetTrace(\sigma)$ that is generated as follows: starting with the
  initial mixture and moving through time, whenever a bell rings, its
  associated event template $e$ is realized (by transforming the
  current mixture $m$ into $\UPDATE{m}{e}$) if and only if
  $\TRIGGERABLE{m}{e}$. For example, if the current mixture $m$ is
  given as in Figure~\ref{fig:mixture} and the bell linked to
  \textit{``apply rule $b$ on substrate $3$ and kinase $4$''} rings, a
  bond is created between these two agents. In contrast, the bell
  linked to \textit{``apply rule $b$ on substrate $1$ and kinase
    $2$'}' would have no effect.
\item Finally, the dynamic behavior of our model is captured by the random
  trace $T \eqdef \DetTrace(\Sigma)$, which can also be sampled
  efficiently using the Doob-Gillepsie algorithm introduced in
  section~\ref{sec:background}.
\end{inparaenum}

Intuitively, $\Sigma$ determines when the opportunity for a reaction
happens and between which molecules. It plays the same role as the
random vector $\vec u$ in a SEM. In contrast,
$\sigma \mapsto \DetTrace(\sigma)$ is a deterministic function that
controls what a reaction does when it occurs.  It corresponds to the
$f_i$ in a SEM and is likewise the target of interventions.


\subsection{A Semantics for Counterfactuals}
\label{subsec:counterfactuals-semantics}

We define an intervention $\iota$ as a predicate
$\BLOCKED{\iota}{\cdot}$ ranging over events. The purpose of the
predicate is to act as a filter preventing the occurrence of selected
events. Given a predicate $\psi$ over traces, we write the statement
\textit{``Had intervention $\iota$ happened in trace $\tau$, $\psi$
  would have been true''} as $\CFST{}$, borrowing a notation from
\cite{halpern2016actual}.

% By construction, the random trace $T(\Sigma)$ is distributed
% according to the CTMC semantics of Kappa discussed in
% section~\ref{sec:background}.

For an intervention $\iota$ and a schedule $\sigma$, we define the
altered trace $\DetTrace_\iota(\sigma)$ much in the same way as
$\DetTrace(\sigma)$, but also requiring $\BLOCKED{\iota}{(e, t)}$ to
be false for $e$ to be realized when its bell rings at time $t$.
Then, we define $T_\iota \eqdef \DetTrace_\iota(\Sigma)$.  Given an
observed trace $\tau$, an intervention $\iota$ and a predicate $\psi$,
the probability of $\CFST{}$ can now be defined according to Pearl's
three-step strategy: \ItAbduction{} condition the distribution of
$\Sigma$ by the observation that $\DetTrace(\Sigma)\!=\!\tau$, then
\ItAction{} alter the behavior of $\DetTrace$ with intervention
$\iota$ and \ItPrediction{} consider the probability that $\psi$ holds
on $\DetTrace_\iota(\Sigma)$. This results in the following
definition.

\begin{definition}[Semantics of counterfactual statements]
  \label{def:counterfactuals}
  For $\tau$ an observed trace, $\iota$ an intervention and $\psi$ a
  predicate on traces, the probability of the counterfactual statement
  \textit{``had intervention $\iota$ happened in trace $\tau$,
    predicate $\psi$ would have been true''} is defined as:
  \[ \PCFST{} \eqdef \ \ProbParen{\psi(\ATRAJ{}) \,\, | \,\, T =
      \tau}. \]
\end{definition}

Following Definition~\ref{def:counterfactuals}, we estimate the
probability of the counterfactual statement $\CFST{}$ by sampling
instances of the random variable $\CTRAJ{}$. 
%(i.e.
%$\DetTrace_\iota(\Sigma) \,|\, \{ \DetTrace(\Sigma) \!=\! \tau \}$).
These are called \emph{counterfactual traces}. Intuitively,
they give an account of what else trace $\tau$ could have been, had
intervention $\iota$ happened.


\subsection{An Example}\label{subsec:counterfactuals-example}

% \newcommand{\RTrace}{\DetTrace(\Sigma)}
% \newcommand{\ATrace}{\DetTrace_\iota(\Sigma)}

\newcommand{\RTrace}{T} \newcommand{\ATrace}{T_\iota} Let us
illustrate our definitions by manually sampling a counterfactual trace
for the example trace $\tau$ given in (\ref{example-trace})\, and the
intervention $\iota$ that consists in blocking every application of
rule $pk$:
% \[\BLOCKED{\iota}{((r, \xi), t)} \ = \ (r = pk).\]
$\BLOCKED{\iota}{((r, \xi), t)} \!=\! (r \!=\! pk)$.
% (Alternative interventions are proposed in
% Appendix~\ref{ap:benchmark}.)
For this, we must draw an instance of $\ATrace$, conditioned on the
observation $\RTrace \!=\! \tau$.

Let us assume that $\RTrace \!=\! \tau$. Then, the first event of
$\ATrace$ has to coincide with the first event of $\tau$ (namely $b$).
Indeed, suppose that $(e, t)$ belongs to $\ATrace$, with $t$ prior to
the time of the first event of $\tau$. Thus, $e$ is scheduled in
$\Sigma$ at time $t$ and realizable in the initial mixture, which is
shared between $\ATrace$ and $\RTrace$. As a consequence, $(e, t)$
also belongs to $\RTrace$ and therefore to $\tau$, which is a
contradiction. Continuing this line of reasoning, $\ATrace$ and $\tau$
must coincide until an event of $\tau$ is blocked by $\iota$.

After $pk$ is blocked in $\ATrace$, the current mixtures in $T$ and
$T_\iota$ start diverging (the kinase is phosphorylated in the former
and unphosphorylated in the latter). We call these mixtures
\emph{factual mixture} and \emph{counterfactual mixture}, respectively.
The next event to happen in $\tau$ is the second binding event $b$. We
argue that it also has to be the next event to happen in
$T_\iota$. Indeed, the only way an event $(e, t)$ can happen in
$\ATrace$ before $b$ while not happening in $\RTrace$ is if $e$ is
realizable in the counterfactual mixture and not in the factual
one. This is only true if $e$ is an instance of rule $pk$ and
applications of this rule -- if scheduled -- would be blocked by
$\iota$ anyway.

After $b$ happens in both $\ATrace$ and $\tau$, the event template
associated with rule $u$ (fast unbinding) becomes realizable in the
counterfactual mixture, but not in the factual one. Therefore, the
observation $T\!=\!\tau$ provides no useful information about whether
or not $u$ is scheduled in $\Sigma$ before $p$ happens in $\tau$. In
fact, the probability that this is not the case is exactly equal to
$\exp(-\lambda_u\delta)$, where $\delta$ is the length of the time
interval between $b$ and $p$ in $\tau$. Given the rates in our model,
$\delta$ is typically of the order of
$(\lambda_{u^*}+\lambda_p)^{-1}$. Therefore, $\lambda_u\delta \gg 1$
and it is very likely that a fast unbinding event happens in $\ATrace$
before $p$ happens in $\tau$, preventing $p$ to happen in $\ATrace$.
This gives us the following counterfactual trace:
\begin{align}
  \label{counterfactual-trace} 
  b,\ \ u,\ \ \strikeout{pk}, \ \ b, \ \ \pmb{u}, \ \ \strikeout{\,p\,}, 
  \ \ \strikeout{u^*}, \ \ \cdots
\end{align} 
where events that are striked out are events of $\tau$ that do not
appear in $\ATrace$ and events in bold are proper to $\ATrace$.


\subsection{The Counterfactual Resimulation Algorithm}
\label{subsec:cosim-algo}

We introduce Algorithm~\ref{alg:cosimulation}, a
variation of the Doob-Gillespie algorithm, to sample a counterfactual
trace efficiently given a reference trace $\tau$ and an intervention
$\iota$. We call it \emph{counterfactual resimulation}, since it works
by going through every event of $\tau$, resimulating only those parts
of $\tau$ that are affected by $\iota$. In particular, when $\iota$ is
the trivial intervention ($\BLOCKED{\iota}{\cdot} = \text{false}$), it
returns $\tau$.

This algorithm relies on a modified notion of activity we call
\emph{divergent activity}. We define the set of \emph{divergent
  embeddings} of the left-hand side of a rule $r$ into mixture $m$ and
relative to $m_0$ as
% \tryinline{\DEMBS{r}{m, m_0} \eqdef \EMBS{r}{m} \setminus
% \EMBS{r}{m_0}.}
$\DEMBS{r}{m, m_0} \eqdef \EMBS{r}{m} \setminus \EMBS{r}{m_0}.$
Equivalently, a divergent embedding is an embedding whose codomain
features a \emph{divergent site}, that is, a site whose state differs
across $m$ and $m_0$. The {divergent activity} of a rule $r$ in
mixture $m$ relative to $m_0$ is then the product
$\lambda_r|\DEMBS{r}{m, m_0}|$. The \emph{total divergent activity} of
the system, $\alpha'(m, m_0)$, is the sum of all divergent
activities. Finally, we use the notation $\TSTATE{t}{\tau}$ to refer
to the mixture at time $t$ in $\tau$\longversion{, which is obtained
  from the initial mixture after updating it for each event in turn up
  to time $t$ in $\tau$}.

\newcommand{\EVF}[0]{e_{\text{f}}}
\newcommand{\EVCF}[0]{e_{\text{c}}}

\begin{algorithm}
\caption{Counterfactual resimulation}\label{alg:cosimulation}
\begin{spacing}{1.3}
\algsetup{indent=1.5em}
\begin{algorithmic}[1]
\vspace{0.2cm}
\STATE $t \gets 0$
\STATE $m \gets\ $ initial mixture
\WHILE{ $t < t_\text{\,end}$ }
  \STATE $m_0 \gets \TSTATE{t}{\tau}$
  \STATE $(\EVF{}, t_{\text{f}}) \gets $ first event of $\tau$ in time interval $(t, \infty)$
  \vspace{0.1cm}
  \STATE $\alpha' \gets \sum_r {\lambda_r |\DEMBS{r}{m, m_0}|}$
  \vspace{0.1cm}
  \STATE draw $\delta \sim \textsc{Exp}(\alpha') $
  \STATE $t_{\text{c}} \gets t + \delta$
  % \STATE
  \IF { $t_{\text{c}} < t_{\text{f}}$ }
      \STATE draw a rule $r$ with prob.
      $\propto \, \lambda_r |\DEMBS{r}{m, m_0}|$
      \STATE  draw a divergent embedding $\xi \in \DEMBS{r}{m, m_0}$
      \STATE {$e \gets (r, \xi)$}
      \IF {$ \neg \, \BLOCKED{\iota}{e, t_{\text{c}}} $ }
          \STATE $t \gets t_{\text{c}}$
          \STATE $m \gets \UPDATE{m}{e}$
          \STATE log event $(e, t)$
      \ENDIF
  \ELSE
      \STATE {$e \gets \EVF{}$}
      \STATE $t \gets t_{\text{f}}$
      \IF {$ \neg \, \BLOCKED{\iota}{e, t} $ \AND $ \TRIGGERABLE{m}{e} $ }
          \STATE $m \gets \UPDATE{m}{e}$
          \STATE log event $(e, t)$
      
      \ENDIF 
  \ENDIF
\ENDWHILE
\vspace{0.1cm}
\end{algorithmic}
\end{spacing}
\end{algorithm}

The role and relevance of the concept of divergent activity in
counterfactual resimulation is summarized by the following
proposition, where $\tau \cap I = \emptyset$ is a shortcut for ``no
event of trace $\tau$ occurs in the time interval $I$''.
\smallskip
\begin{proposition}[Property of the divergent
  activity]\label{prop:div-activity}
  For $\tau$ a trace and $\iota$ an intervention, let
  $I = (t, t+\delta)$ be a time interval such that
  $\tau \cap I = \emptyset$ and $m_0 = \TSTATE{t}{\tau}$. Then, we
  have
  \[\CProb{ \ATRAJ{} \cap I = \emptyset }{ T=\tau,\
      \TSTATE{t}{\ATRAJ{}} = m\ }
    \ =\ e^{-\alpha'(m, m_0) \cdot \delta}.
  \]
\end{proposition}
\smallskip
\noindent At every iteration of Algorithm~\ref{alg:cosimulation}, the
divergent activity $\alpha'$ determines the probability that an event
happens in the counterfactual trace prior to the next event in the
factual trace $\tau$ (test of line \ref{cosim:cev}). A proof of
Proposition~\ref{prop:div-activity} is given in
Appendix~\ref{ap:div-activity}.
It is the main step in establishing:

\begin{theorem}%[Correctness of counterfactual resimulation]
  The counterfactual resimulation algorithm correctly samples
  instances of $\,\CTRAJ{}$.
  % as defined in
  % section~\ref{subsec:counterfactuals-semantics}.
\end{theorem}


 
% -*- TeX-master: "ijcai18.tex" -*-

\subsection{Implementation}\label{subsec:implementation}

There are two challenges in efficiently implementing counterfactual
resimulation. The first is a suitable representation for the sets of
divergent embeddings $\DEMBS{r}{m}$ to minimize the cost of their
update at each iteration. Since the Kappa simulator solves exactly
that problem for the sets of embeddings $\EMBS{r}{m}$
\cite{DanosEtAl-APLAS07}, we leverage most of that infrastructure. The
second consists in avoiding excessively many iterations of
Algorithm~\ref{alg:cosimulation} in which time is advanced in tiny
increments and the proposed event is rejected.
%(line~\ref{cosim:blocked}). 
Suppose, for example, that in our toy model
$pk$ has a very high firing rate and we wish to block, from a
specific time onward, \emph{all} events in which the sole kinase
becomes phosphorylated. Upon blocking one occurrence of the event, the
same event would want to happen again, and we would keep rejecting it
a huge number of times until a different rule fires. More generally,
event templates whose realization is bound to be blocked should be
removed efficiently before their realization is attempted and not be
counted in the system's divergent activity. We solve this problem
for a class of interventions we call \emph{regular}.
Specifically, an intervention $\iota$ is regular if
the predicate $\BLOCKED{\iota}{((r, \xi), t)}$ can be expressed as a
finite disjunction of formulae of the form
$(r \!=\! r') \wedge F(\xi{\restriction_{c}}) \wedge (t \!\in\! I)$ or
$G(r, \xi) \wedge (t \!=\! t')$, where $r'$ is a rule, $t'$ a time, $I$ a
time interval, $\xi{\restriction_{c}}$ the restriction of $\xi$ to a
single connected component $c$ of $L_{r'}$, and $F, G$ arbitrary
predicates. For regular interventions, our implementation is
guaranteed to either produce or consume an event at each iteration.

\begin{proposition}
  Sampling a counterfactual trace for a \emph{regular} intervention
  can be done in time $\mathcal{O}(n \cdot r \log|m|)$, where $n$ is
  the sum of the number of events in the reference trace and in the
  resulting counterfactual trace, $r$ is the number of rules in the
  model and $|m|$ the size of the reaction mixture.
\end{proposition}

We provide a benchmark of our implementation on a scaled-up version of
our toy model in Appendix~\ref{ap:benchmark}. The average slowdown per
event compared to the Kappa simulator does not exceed 50\% for a
variety of interventions.

\medskip

Returning to our running example, sampling counterfactual traces
repeatedly for \RefTrace{} would reveal that, with very high
probability, ``event $p$ would not have happened, had $pk$ not
happened''. However, we can go further by using counterfactual traces
to \textit{explain} this observation using enablement and prevention
arrows.
