% -*- TeX-master: "ijcai18.tex" -*-

\newcommand{\PCFST}[0]{\ProbParen{\CFST{}}}

\newcommand{\ItAbduction}[0]{(\textit{abduction})}
\newcommand{\ItAction}[0]{(\textit{action})}
\newcommand{\ItPrediction}[0]{(\textit{prediction})}


\section{Evaluating counterfactual
  statements}\label{sec:counterfactual}

In the context of our toy model, the counterfactual statement we must
assess is: ``If $pk$ had not happened, $p$ would not have happened."
Our account in the previous section suggests that $pk$ played a role,
but it is also clear that given the stochastic nature of rule firing
$p$ could well have happened even in the absence of $pk$; it just is
unlikely. Counterfactual statements are not either true or false, but
have degrees of likelihood. To assess that likelihood is our task.

Given an original (factual) trace $\tau$, a naive approach might be to
sample counterfactual traces, each of which starts with the state of
the system attained in $\tau$ just before event $pk$ happened, but in
which we skip over $pk$ and then run an unconstrained simulation from
that point onward. In this approach traces would quickly diverge from
the original, distorting the causal role that $pk$ played
specifically in it. The question here is not what causal role
$pk$ can play in principle, but what role it actually
did play in $\tau$. In this sense, counterfactual statements
are undetachable from the context in which they are formulated.

Pearl's standard account of counterfactual statements
\cite{pearl2009causality} is based on performing \textit{``surgical
  interventions''} on a structural equation model (SEM). A SEM
features a finite sequence $(x_1, \dots, x_n)$ of variables, each
associated to a \emph{functional equation} of the form
$x_i = f_i(x_1, \dots, x_{i-1}, u_i)$, where $f_i$ is a deterministic
function and $u_i$ is a random variable. Ideally, each $f_i$ defines
an independent and autonomous physical mechanism. This is partially
enforced by the requirement that the $u_i$ must be mutually
independent. Given some observation $e$,
the probability of the counterfactual statement ``had $x_j$ been equal
to $a$, $\psi$ would have been be true'' is evaluated following a
three-steps process:
\begin{inparaenum}[]
\item \ItAbduction{} compute the distribution $p_e$ of values
  for $\vec u$ given observation $e$, then
\item \ItAction{} intervene in the model by replacing the defining
  equation for $x_j$ by ``$x_j = a$'' and finally
\item \ItPrediction{} compute the probability that $\psi$ is true in
  this new model when $\vec{u}$ is distributed according to $p_e$.
\end{inparaenum}

Because traces generated from rule-based models do not have a natural
encoding in terms of structural equations, Pearl's construction does
not apply straightforwardly in our case.

\subsection{A semantics for counterfactuals}
\label{subsec:counterfactuals-semantics}

An intervention $\iota$ prevents the occurrence of selected events. It
is defined by a predicate $\BLOCKED{\iota}{\cdot}$ ranging over
events. \longversion{(Our framework can accomodate a much broader range of
interventions, but we make this restriction for ease of exposition.)}
Given a predicate $\psi$ over traces, we write the statement
\textit{``Had intervention $\iota$ happened in trace $\tau$, $\psi$
  would have been true''} as $\CFST{}$.

To assign a probability to this statement, it is useful to
reconceptualize the CTMC induced by a model in such a way that the
source of randomness is factored out from the system's state. This can
be done as follows:
\begin{inparaenum}[(i)]
\item Consider all possible potential events $(r, \xi)$, where $\xi$
  maps into a large enough set of global identifiers. If no rule
  creates a new agent, as we shall assume for the sake of simplicity,
  those global identifiers are given by the initial mixture. For each
  one of these potential events, imagine a bell that rings
 according to a Poisson process of parameter
  $\lambda_r$, i.e.\@ the time intervals between consecutive
  ringings are drawn independently from an exponential distribution with
  parameter $\lambda_r$. These Poisson processes are all gathered in
  a random variable $\Sigma$ that we call \emph{schedule}. It features
  the sequence of ring times for every bell and plays the same role as
  the variable $\vec{u}$ in a SEM.
\item A simulation trace can be viewed as a deterministic
  function of $\Sigma$: starting with the initial mixture and
  moving through time, whenever a bell rings, its associated potential
  event $e$ transforms the current mixture $m$ if
  $\TRIGGERABLE{m}{e}$. For example, if the current mixture $m$ is given as in 
  Figure~\ref{fig:mixture} and the bell labeled
  \textit{``apply rule $b$ on substrate $3$ and kinase $4$''} rings, then a bond is created
  between these two agents. However, the bell labeled
  \textit{``apply rule $b$ on substrate $1$ and kinase $2$'}' would have no effect 
  on $m$. We write $T(\sigma)$ the
  particular trace generated from schedule $\sigma$ and
  $T = T(\Sigma)$.
\end{inparaenum}

We extend this viewpoint to include interventions. For an
intervention $\iota$, we define the altered trace $\ATRAJ{}$ much in
the same way as $T$, except that each time the bell associated to $e$
rings at time $t$, we also require $\BLOCKED{\iota}{(e, t)}$ to be
false for $e$ to be executed.  Given an observed trace $\tau$, an
intervention $\iota$ and a predicate $\psi$,
the probability of $\CFST{}$ can now be computed according to Pearl's
three-steps strategy: \ItAbduction{} condition the distribution of
$\Sigma$ by the observation that $T=\tau$, then \ItAction{} alter
the behavior of the simulation with intervention $\iota$ and
\ItPrediction{} compute the probability of $\psi$ in the resulting
setting. This results in the following definition.

\begin{definition}[Semantics of counterfactual statements]\label{def:counterfactuals}
  For $\tau$ an observed trace, $\iota$ an intervention and $\psi$ a
  predicate on traces, the probability of the counterfactual statement
  \textit{``had intervention $\iota$ happened in trace $\tau$,
    predicate $\psi$ would have been true''} is defined as:
  \[ \PCFST{} \eqdef \ \ProbParen{\psi(\ATRAJ{}) \ |\ T = \tau}. \]
\end{definition}

\subsection{The counterfactual resimulation algorithm}
\label{subsec:cosim-algo}

Following Definition~\ref{def:counterfactuals}, we estimate the
probability of the counterfactual statement
$\CFST{}$ by sampling instances of the random
variable $\CTRAJ{}$. Such instances are \emph{counterfactual
  traces}. Intuitively, they give an account of what else trace $\tau$
could have been had intervention $\iota$ happened.  We introduce
Algorithm~\ref{alg:cosimulation}, a variation of the Doob-Gillespie
algorithm, to sample a counterfactual trace efficiently given a
reference trace $\tau$ and an intervention $\iota$. We call it
\emph{counterfactual resimulation}, since it works by going
through every event of $\tau$, resimulating only those parts of $\tau$
that are affected by $\iota$. In particular, when $\iota$ is
the trivial intervention ($\BLOCKED{\iota}{\cdot} = \text{false}$), it
returns $\tau$.

This algorithm relies on a modified notion of activity we call
\emph{divergent activity}. We define the set of \emph{divergent
  embeddings} of the left-hand side of a rule $r$ into mixture $m$
and relative to $m_0$ as
%\tryinline{\DEMBS{r}{m, m_0} \eqdef \EMBS{r}{m} \setminus \EMBS{r}{m_0}.}
\[\DEMBS{r}{m, m_0} \eqdef \EMBS{r}{m} \setminus \EMBS{r}{m_0}.\]
Equivalently, a divergent embedding is an embedding whose codomain
features a \emph{divergent site}, that is, a site whose state differs
across $m$ and $m_0$. The {divergent activity} of a rule $r$ in
mixture $m$ relative to $m_0$ is then the product
$\lambda_r|\DEMBS{r}{m, m_0}|$. The \emph{total divergent activity} of
the system, $\alpha'(m, m_0)$, is the sum of all divergent
activities. Finally, we use the notation $\TSTATE{t}{\tau}$ to refer
to the mixture at time $t$ in $\tau$\longversion{, which is obtained
  from the initial mixture after updating it for each event in turn up
  to time $t$ in $\tau$}.


\newcommand{\EVF}[0]{e_{\text{f}}}
\newcommand{\EVCF}[0]{e_{\text{c}}}

\begin{algorithm}
\caption{Counterfactual resimulation}\label{alg:cosimulation}
\begin{spacing}{1.3}
\algsetup{indent=1.5em}
\begin{algorithmic}[1]
\vspace{0.2cm}
\STATE $t \gets 0$
\STATE $m \gets\ $ initial mixture
\WHILE{ $t < t_\text{\,end}$ }
  \STATE $m_0 \gets \TSTATE{t}{\tau}$
  \STATE $(\EVF{}, t_{\text{f}}) \gets $ first event of $\tau$ in time interval $(t, \infty)$
  \vspace{0.1cm}
  \STATE $\alpha' \gets \sum_r {\lambda_r |\DEMBS{r}{m, m_0}|}$
  \vspace{0.1cm}
  \STATE draw $\delta \sim \textsc{Exp}(\alpha') $
  \STATE $t_{\text{c}} \gets t + \delta$
  % \STATE
  \IF { $t_{\text{c}} < t_{\text{f}}$ }
      \STATE draw a rule $r$ with prob.
      $\propto \, \lambda_r |\DEMBS{r}{m, m_0}|$
      \STATE  draw a divergent embedding $\xi \in \DEMBS{r}{m, m_0}$
      \STATE {$e \gets (r, \xi)$}
      \IF {$ \neg \, \BLOCKED{\iota}{e, t_{\text{c}}} $ }
          \STATE $t \gets t_{\text{c}}$
          \STATE $m \gets \UPDATE{m}{e}$
          \STATE log event $(e, t)$
      \ENDIF
  \ELSE
      \STATE {$e \gets \EVF{}$}
      \STATE $t \gets t_{\text{f}}$
      \IF {$ \neg \, \BLOCKED{\iota}{e, t} $ \AND $ \TRIGGERABLE{m}{e} $ }
          \STATE $m \gets \UPDATE{m}{e}$
          \STATE log event $(e, t)$
      
      \ENDIF 
  \ENDIF
\ENDWHILE
\vspace{0.1cm}
\end{algorithmic}
\end{spacing}
\end{algorithm}

The role and relevance of the concept of divergent activity in
counterfactual resimulation is summarized by the
following theorem, where $\tau \cap I = \emptyset$ is a
shortcut for the proposition ``no event of trace $\tau$ occurs in
the time interval $I$''.
\begin{proposition}[Property of the divergent activity]\label{prop:div-activity}
  For $\tau$ a trace
  and $\iota$ an intervention, let $I = (t, t+\delta)$ be a time interval
  such that $\tau \cap I = \emptyset$ and $m_0 =
  \TSTATE{t}{\tau}$. Then, we have
  \[\CProb{ \ATRAJ{} \cap I = \emptyset }{ T=\tau,\
      \TSTATE{t}{\ATRAJ{}} = m\ }
    \ =\ e^{-\alpha'(m, m_0) \cdot \delta}.
  \]
\end{proposition}
\noindent At every iteration of Algorithm~\ref{alg:cosimulation}, the
divergent activity $\alpha'$ determines the probability that an event
happens in the counterfactual trace prior to the next event in the
factual trace $\tau$ (test of line \ref{cosim:cev}).  A proof of
Proposition~\ref{prop:div-activity} is given in
Appendix~\ref{ap:div-activity}.

Returning to our running example, sampling counterfactual traces
repeatedly for \RefTrace{} would reveal that ``event $p$ would not
have happened had $pk$ not happened'' with very high probability.
However, we can go further by using counterfactual traces to
\textit{explain} this influence using enablement and prevention
arrows.
