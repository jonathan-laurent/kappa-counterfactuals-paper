% -*- TeX-master: "ijcai18.tex" -*-

\newcommand{\PCFST}[0]{\mathbf{P}\left( \,\UPDATE{\tau}{[\iota]} \models \psi \,\right)}
\newcommand{\ItAbduction}[0]{(\textbf{abduction})}
\newcommand{\ItAction}[0]{(\textbf{action})}
\newcommand{\ItPrediction}[0]{(\textbf{prediction})}


\section{Evaluating counterfactual statements}\label{sec:counterfactual}

%\subsection{On the nature of  counterfactual statements}

In the context of our toy model, the counterfactual statement we must assess is: ``If $pk$ had not happened, $p$ would not have happened." Our account in the previous section suggests that $pk$ played a role, but it is also clear that given the stochastic nature of rule firing $p$ could well have happened even in the absence of $pk$; it just is unlikely. Counterfactual statements are not either true or false, but have degrees of likelihood. To assess that likelihood is our task.

Given an original (factual) history $\tau$, a naive approach might be to sample counterfactual histories, each of which starts with the state of the system attained in $\tau$ just before event $pk$ happened, but in which we skip over $pk$ and then run an unconstrained simulation from that point onward. In this approach traces would quickly diverge from the original, distorting the causal role that $pk$ played \emph{specifically} in it.  The question here is not what causal role $pk$ \emph{can} play in principle, but what role it actually \emph{did} play in $\tau$. To this end, we need a procedure to sample counterfactual histories that are in a probabilistic sense as close to the original as the removal of $pk$ permits. In this reasoning we are following the long-standing insight \cite{lewis1974causation,pearlXX} that counterfactual statements cannot be detached from the context in which they are formulated, here the trace $\tau$.

The view just outlined follows the approach to counterfactual reasoning with structural equations models \cite{PearlXX,HitchcockXX}. We bend history through surgical intervention (the blocking of $pk$) and make use of evidence (the observation of $\tau$) by sampling counterfactual histories that probabilistically shadow $\tau$. This will be restated in more detail at the end of the section. 

The novel aspect comes from executing Pearl's strategy in the context of a rule-based model for which, in general, it is impractical (if at all possible) to provide a manageable set of structural equations \emph{explicitly}. Rule-based models are meant to capture molecular systems of staggering complexity, whose explicit representation in terms of kinetic differential equations or stochastic master equations or graphical models whose observables are molecular species would trigger a combinatorial explosion that cannot be contained. However, rule-based models, like structural equations, are \emph{mechanistic}, each rule being the formalization of (ideally) an empirically identified local mechanism of interaction. We believe, therefore, that the basic concepts of evaluating counterfactuals in structural equation models carry over, but require a technique for appropriately sampling counterfactual traces. As a bonus, by extending existing causal analysis to pairs of factual and counterfactual traces, we can construct the graphical model \emph{specific to the phenomenon of interest} that was not available in explicit form at the outset.

\subsection{A semantics for counterfactuals}
\label{sec:counterfactuals-semantics}

An intervention $\iota$ prevents the occurrence of an event. It is defined by a predicate $\BLOCKED{\iota}{\cdot}$ ranging over events. (Our framework can accomodate a much broader range of interventions, but we make this restriction for ease of exposition.)  Given a predicate $\psi$ over traces, we write the statement \textit{``Had intervention $\iota$ happened in trace $\tau$, $\psi$ would have been true''} as $\UPDATE{\tau}{[\iota]} \models \psi$ (read ``$\iota$ applied to $\tau$ satisfies $\psi$).

To assign a probability to this statement, it is useful to reconceptualize 
the CTMC induced by a set of rules.
\begin{inparaenum}[(i)]
\item \label{item:poisson}
Consider all possible semi-events $(r, \xi)$, where $\xi$ maps into a large enough set of global identifiers. If no rule creates a new agent, as we shall assume for the sake of simplicity, the global identifiers are conserved and given by the initial mixture. Associate with each semi-event a Poisson process with rate $\lambda_r$, that is, generate a monotonically increasing series of time points $t_{r,i}$, $i=1,\ldots$ with intervals $\Delta t_r=t_{r,i+1}-t_{r,i}$ exponentially distributed, $\Prob{\Delta t_r=t}=\lambda_r\exp(-\lambda_r t)$. Run the series long enough to cover the anticipated runtime of the CTMC. In accordance with the physical assumptions underlying chemical kinetics, each semi-event by itself is independent of all others and so are each of its time points. 
\item \label{item:union}
Take the union of all points, $\cup_{r,i}t_{r,i}$, and sort in increasing order. Call this the schedule $\sigma$. 
\item \label{item:chop}
Determine the earliest semi-event $e=(r, \xi)$ that is executable in the current mixture and update the mixture according to $e$ (see section~\ref{sec:background} for definitions). This generates an event $(e, t_{r,k})$ with $t_{r,k}$ the corresponding time point. Delete from $\sigma$ all time points up to and including $t_{r,k}$. 
\item \label{item:repeat}
If $\sigma$ is not empty, return to (\ref{item:chop}). 
\end{inparaenum}
This terribly inefficient procedure is not meant for implementation. Its purpose is to conceptualize a particular CTMC as a random variable $\Sigma$ with value $\sigma$ (a vector of random numbers) that is converted into a trace $\tau$ by the deterministic steps (\ref{item:chop}) and (\ref{item:repeat}). For example, if the current mixture is given by Figure~\ref{fig:mixture} and the scan through $\sigma$ encounters the semi-event \emph{``apply rule $b$ to substrate $3$ and kinase $4$''}, then a bond is created between these two agents. If the scanned semi-event were \emph{``apply rule $b$ to substrate $1$ and kinase $2$'}', nothing happens. Note that associating a Poisson process to a semi-event allows for the latter to happen repeatedly. For example, a particular substrate and kinase might bind and unbind several times. It is straightforward to include interventions in this procedure by requiring  $\BLOCKED{\iota}{(e, t)}$ to be false for the event $(e,t)$ to be accepted and true to be skipped over. We can now define $\PCFST{}$.
\begin{definition}[Semantics of counterfactual statements]\label{def:counterfactuals}
Given an observed trace $\tau$, an intervention $\iota$, a random variable $\ATRAJ{}$ taking as values traces to which $\iota$ was applied, and a predicate $\psi$ on traces, the probability of the counterfactual statement \textit{``Had intervention $\iota$ occurred in trace $\tau$, predicate $\psi$ would have been true''} is defined as:
  \[ \PCFST{} \ \eqdef \
    \mathbf{P}(\, \psi(\ATRAJ{} \ |\ T = \tau\,)). \]
\end{definition}

\subsection{The counterfactual resimulation algorithm}

We refer to instances of the conditional random variable $\CTRAJ{}$ as \emph{counterfactual traces} and introduce a variation of the CTMC algorithm to efficiently sample them.  We call it \emph{counterfactual resimulation}, since it works by going through every event of $\tau$, resimulating only those parts of $\tau$ that are affected by $\iota$.

\begin{algorithm}
\caption{Counterfactual resimulation}\label{alg:cosimulation}
\begin{spacing}{1.3}
\algsetup{indent=1.5em}
\begin{algorithmic}[1]
\vspace{0.2cm}
\STATE $t \gets 0$
\STATE $m \gets\ $ initial mixture
\WHILE{ $t < t_\text{\,end}$ }
  \STATE $m_0 \gets \TSTATE{t}{\tau}$
  \STATE $(e_{\text{f}}, t_{\text{f}}) \gets $ first event of $\tau$ in time interval $(t, \infty)$
  \vspace{0.1cm}
  \STATE $\alpha' \gets \sum_r {\lambda_r |\DEMBS{r}{m, m_0}|}$
  \vspace{0.1cm}
  \STATE draw $\delta \sim \textsc{Exp}(\alpha') $
  \STATE $t_{\text{c}} \gets t + \delta$
  % \STATE
  \IF { $t_{\text{c}} < t_{\text{f}}$ }
      \STATE draw a rule $r$ with prob.
      $\propto \, \lambda_r |\DEMBS{r}{m, m_0}|$
      \STATE  draw a divergent embedding $\xi \in \DEMBS{r}{m, m_0}$
      \STATE {$e \gets (r, \xi)$}
      \IF {$ \neg \, \BLOCKED{\iota}{e, t_{\text{c}}} $ }
          \STATE $t \gets t_{\text{c}}$
          \STATE $m \gets \UPDATE{m}{e}$
          \STATE log event $(e, t)$
      \ENDIF
  \ELSE
      \STATE $t \gets t_{\text{f}}$
      \IF {$ \neg \, \BLOCKED{\iota}{e, t} $ \AND $ \TRIGGERABLE{m}{e} $ }
          \STATE $m \gets \UPDATE{m}{e}$
          \STATE log event $(e, t)$
      
      \ENDIF 
  \ENDIF
\ENDWHILE
\vspace{0.1cm}
\end{algorithmic}
\end{spacing}
\end{algorithm}

To explicate Algorithm~\ref{alg:cosimulation} we need the notion of \emph{divergent activity} of rule $r$, which we define by first introducing the set of \emph{divergent embeddings} of the left-hand side of $r$ into mixture $m$ relative to a reference mixture $m_0$: \[\DEMBS{r}{m, m_0} \eqdef \EMBS{r}{m} \setminus \EMBS{r}{m_0}.\] Equivalently, a divergent embedding is an embedding whose codomain features a \emph{divergent site}, that is, a site whose state differs across $m$ and $m_0$. The {divergent activity} of a rule $r$ in mixture $m$ relative to $m_0$ is then defined as the product $\lambda_r|\DEMBS{r}{m, m_0}|$. The \emph{total divergent activity} of the system, $\alpha'(m, m_0)$, is the sum of all divergent activities. We use the notation $\TSTATE{t}{\tau}$ to refer to the mixture at time $t$, which obtains from the initial mixture after updating it for each event in turn up to time $t$ in $\tau$. 

For convenience we provide a cursory summary of Algorithm~\ref{alg:cosimulation}. The factual trace $\tau$ guides the construction of a counterfactual trace $\ATRAJ{}$ as follows. The portion of $\tau$ up to the intervention (the blocking of a single event, say) is copied over to $\ATRAJ{}$. At this point, the factual and counterfactual mixture start to differ, giving rise to divergent rule activities. The next candidate event in $\ATRAJ{}$ is determined as in the standard Algorithm~\ref{alg:gillespie}, but using divergent activities. If the time of the candidate event is earlier than the next event in the factual trace, the counterfactual mixture and trace are updated with the candidate event; if the time is later, the event in the factual trace occurs at its factual time in the counterfactual scenario instead, but only if the counterfactual mixture admits it, otherwise nothing happens. The simulated time of $\ATRAJ{}$ is advanced accordingly in all cases. Repeat.

The role and relevance of the concept of divergent activity in the counterfactual resimulation algorithm can be summarized by the following theorem, where we write $\tau \cap I = \emptyset$ as a shortcut for the proposition ``no event of trace $\tau$ occurs in the time interval $I$''.
\begin{theorem}[Property of the divergent activity]\label{thm:div-activity}
Let $\tau$ be a trace and $\iota$ an intervention. Let $I = (t, t+\delta)$ be    a time interval such that $\tau \cap I = \emptyset$ and $m_0 =\TSTATE{t}{\tau}$. Then, we have
  \[\CProb{ \ATRAJ{} \cap I = \emptyset }{ T=\tau,\
      \TSTATE{t}{\ATRAJ{}} = m\ }
    \ =\ e^{-\alpha'(m, m_0) \cdot \delta}.
  \]
\end{theorem}
\noindent At every iteration of Algorithm~\ref{alg:cosimulation}, the divergent activity $\alpha'$ determines the probability that an event happens in the counterfactual trace prior to the next event in the factual trace $\tau$ (test of line \ref{cosim:cev}).  A proof of Theorem~\ref{thm:div-activity} is given in Appendix~\ref{ap:div-activity}.

We return to the correspondence of our approach to Pearl's three-step procedure. The rules of a model do not themselves contain random variables representing unknown causal factors. Rather, randomness stems solely from the execution schedule $\Sigma$, which represents the intrinsic randomness of chemical kinetics. In section~\ref{sec:counterfactuals-semantics} we showed that the series of dice rolls $\Sigma=\sigma$ uniquely defines a trace. Thus, the step in which evidence is used to update knowledge about random factors in structural equations simply becomes the observation of the factual trace: The $\Sigma$ of the trace is the evidence that provides (complete) information about the randomness that made it the trace it is. The second step, in which a surgical intervention happens is the same in structural equations and here. The third step, in which the modified structural equations are used to compute probabilities, becomes the sampling of modified traces $\ATRAJ{}$ conditioned by $\Sigma=\sigma$ (the evidence). By virtue of the resimulation Algorithm~\ref{alg:cosimulation} $\ATRAJ{}$ inherits as many events from the factual history as the surgical intervention permits it to. This is an operationalization of Lewis' closest worlds criterion without going off the metaphysical cliff.

%\section{Counterfactual simulation}\label{sec:counterfactual}

Counterfactual statements are tricky because their truth is
context-dependent. The statement ``Had it not rained, the driver might
have arrived earlier" can fail to be true in many ways. Intuitively,
how easily the driver could have arrived earlier depends on how great
a departure from actuality is required for it to be case
\cite{Lewis1973}. This is why counterfactual reasoning is tied to
modal logic. The standard approach is to require that the consequent
in a counterfactual be true in some of those possible worlds (in which
the antecedent holds) that \textit{are most similar to the actual
  world}. If the counterfactual statement is true in all of these
worlds, we can replace ``might" with ``would".  We now operationalize
this approach in the context of Kappa traces and interpolate between
``might" and ``would" using probabilities.

\subsection{A semantics for counterfactuals}

We start by formalizing the notion of an intervention. An intervention
$\iota$ (``blocking $pk$" in our example) is a predicate
$\BLOCKED{\iota}{t, e}$ that determines whether or not event $e$ is
blocked at time $t$. Given a predicate $\varphi$ over traces, we write
the proposition \textit{``Had intervention $\iota$ happened in trace
  $\tau$, $\varphi$ would have been true with probability greater than
  $p \in [0,1]$''} as:
\[ \tau \models_p [\iota] \, \varphi.
\]

To give an operational meaning to this statement, we invoke the
continuous time Markov chain (CTMC) semantics of a Kappa model as
defined and implemented in
\cite{DanosEtAl-APLAS07,BoutillierEK17}. For the present purpose it is
conceptually useful to think of a CTMC abstractly in terms of the
random realization of ``potential events". A potential event is a pair
$(r, \xi)$ where $r$ is a rule and $\xi$ an injective mapping from
local agents involved in $r$ to global agents in a huge virtual
mixture of many instances of all possible molecular
species.\footnote{For the sake of simplicity, we assume that no agent
  is created or deleted by a rule.} For every such potential event, we
imagine a bell that rings at a time $t$ drawn from an exponential
distribution $\lambda_r\exp(-\lambda_r t)$, where $\lambda_r$ is the
stochastic rate constant of $r$. A simulation trace can be viewed as
the realization of a random variable $T$ determined by the set
$\omega$ of ring times: Starting with an initial mixture, when a bell
rings at $t$, its associated potential event $(r, \xi)$ transforms the
mixture according to $r$ if $\xi$ yields a valid embedding of the left
hand side of $r$ in the current mixture and time advances by
$t$. Otherwise, time advances and nothing happens---a null
event. Repeat on the resulting mixture.

We can extend this viewpoint to include interventions. For an
intervention $\iota$, we define the random variable $\ATRAJ{}$ much in
the same way as $T$, except that each time the bell rings, we require
$\BLOCKED{\iota}{t, e}$ to be false for the potential event
$e=(r, \xi)$ to be considered.  Counterfactual traces that are closest
to the actual trace $\tau$ are then sampled by generating realizations
of $\ATRAJ{}$ that inherit, whenever possible, the subset of $\omega$
that made up $\tau$. 
% An efficient implementation of this specification
% for sampling the conditional random variable $\CTRAJ{}$ is available at
% \begin{center}
%   \url{https://github.com/jonathan-laurent/kappa-counterfactuals}.
% \end{center} 
We refer to this natural extension of CTMC semantics as
\textit{counterfactual re-simulation} or \textit{co-simulation} for
short. Using co-simulation, we can operationalize the counterfactual
statements as follows.

\begin{definition}[Semantics of counterfactual statements] We write
  $\tau \models_p [\iota] \, \varphi$ the counterfactual statement
  \textit{``had intervention $\iota$ happened in trace $\tau$,
    predicate $\varphi$ would have been true with probability greater
    than $p$''}.  It is defined as follows:
  \[ \tau \models_p [\iota] \, \varphi \quad \Longleftrightarrow \quad
    \mathbf{P}( \varphi(\ATRAJ{}) \ |\ T = \tau) \,\geq\, p \]
\end{definition}

\subsection{Sampling counterfactual traces}

Counterfactual traces can be sampled using the algorithm described in
Listing~\ref{alg:cosimulation}.

\begin{algorithm}
\caption{Counterfactual resimulation}\label{alg:cosimulation}
\begin{spacing}{1.3}
\algsetup{indent=1.5em}
\begin{algorithmic}[1]
\vspace{0.2cm}
\STATE $t \gets 0$
\STATE $m \gets\ $ initial mixture
\WHILE{ $t < t_\text{\,end}$ }
  \STATE $m_0 \gets \TSTATE{t}{\tau}$
  \STATE $(e_{\text{f}}, t_{\text{f}}) \gets $ first event of $\tau$ in time interval $(t, \infty)$
  \vspace{0.1cm}
  \STATE $\alpha' \gets \sum_r {\lambda_r |\DEMBS{r}{m, m_0}|}$
  \vspace{0.1cm}
  \STATE draw $\delta \sim \textsc{Exp}(\alpha') $
  \STATE $t_{\text{c}} \gets t + \delta$
  % \STATE
  \IF { $t_{\text{c}} < t_{\text{f}}$ }
      \STATE draw a rule $r$ with prob.
      $\propto \, \lambda_r |\DEMBS{r}{m, m_0}|$
      \STATE  draw a divergent embedding $\xi \in \DEMBS{r}{m, m_0}$
      \STATE {$e \gets (r, \xi)$}
      \IF {$ \neg \, \BLOCKED{\iota}{e, t_{\text{c}}} $ }
          \STATE $t \gets t_{\text{c}}$
          \STATE $m \gets \UPDATE{m}{e}$
          \STATE log event $(e, t)$
      \ENDIF
  \ELSE
      \STATE $t \gets t_{\text{f}}$
      \IF {$ \neg \, \BLOCKED{\iota}{e, t} $ \AND $ \TRIGGERABLE{m}{e} $ }
          \STATE $m \gets \UPDATE{m}{e}$
          \STATE log event $(e, t)$
      
      \ENDIF 
  \ENDIF
\ENDWHILE
\vspace{0.1cm}
\end{algorithmic}
\end{spacing}
\end{algorithm}
\section{Counterfactual simulation}\label{sec:counterfactual}

Counterfactual statements are tricky because their truth is
context-dependent. The statement ``Had it not rained, the driver might
have arrived earlier" can fail to be true in many ways. Intuitively,
how easily the driver could have arrived earlier depends on how great
a departure from actuality is required for it to be case
\cite{Lewis1973}. This is why counterfactual reasoning is tied to
modal logic. The standard approach is to require that the consequent
in a counterfactual be true in some of those possible worlds (in which
the antecedent holds) that \textit{are most similar to the actual
  world}. If the counterfactual statement is true in all of these
worlds, we can replace ``might" with ``would".  We now operationalize
this approach in the context of Kappa traces and interpolate between
``might" and ``would" using probabilities.

\subsection{A semantics for counterfactuals}

We start by formalizing the notion of an intervention. An intervention
$\iota$ (``blocking $pk$" in our example) is a predicate
$\BLOCKED{\iota}{t, e}$ that determines whether or not event $e$ is
blocked at time $t$. Given a predicate $\varphi$ over traces, we write
the proposition \textit{``Had intervention $\iota$ happened in trace
  $\tau$, $\varphi$ would have been true with probability greater than
  $p \in [0,1]$''} as:
\[ \tau \models_p [\iota] \, \varphi.
\]

To give an operational meaning to this statement, we invoke the
continuous time Markov chain (CTMC) semantics of a Kappa model as
defined and implemented in
\cite{DanosEtAl-APLAS07,BoutillierEK17}. For the present purpose it is
conceptually useful to think of a CTMC abstractly in terms of the
random realization of ``potential events". A potential event is a pair
$(r, \xi)$ where $r$ is a rule and $\xi$ an injective mapping from
local agents involved in $r$ to global agents in a huge virtual
mixture of many instances of all possible molecular
species.\footnote{For the sake of simplicity, we assume that no agent
  is created or deleted by a rule.} For every such potential event, we
imagine a bell that rings at a time $t$ drawn from an exponential
distribution $\lambda_r\exp(-\lambda_r t)$, where $\lambda_r$ is the
stochastic rate constant of $r$. A simulation trace can be viewed as
the realization of a random variable $T$ determined by the set
$\omega$ of ring times: Starting with an initial mixture, when a bell
rings at $t$, its associated potential event $(r, \xi)$ transforms the
mixture according to $r$ if $\xi$ yields a valid embedding of the left
hand side of $r$ in the current mixture and time advances by
$t$. Otherwise, time advances and nothing happens---a null
event. Repeat on the resulting mixture.

We can extend this viewpoint to include interventions. For an
intervention $\iota$, we define the random variable $\ATRAJ{}$ much in
the same way as $T$, except that each time the bell rings, we require
$\BLOCKED{\iota}{t, e}$ to be false for the potential event
$e=(r, \xi)$ to be considered.  Counterfactual traces that are closest
to the actual trace $\tau$ are then sampled by generating realizations
of $\ATRAJ{}$ that inherit, whenever possible, the subset of $\omega$
that made up $\tau$. 
% An efficient implementation of this specification
% for sampling the conditional random variable $\CTRAJ{}$ is available at
% \begin{center}
%   \url{https://github.com/jonathan-laurent/kappa-counterfactuals}.
% \end{center} 
We refer to this natural extension of CTMC semantics as
\textit{counterfactual re-simulation} or \textit{co-simulation} for
short. Using co-simulation, we can operationalize the counterfactual
statements as follows.

\begin{definition}[Semantics of counterfactual statements] We write
  $\tau \models_p [\iota] \, \varphi$ the counterfactual statement
  \textit{``had intervention $\iota$ happened in trace $\tau$,
    predicate $\varphi$ would have been true with probability greater
    than $p$''}.  It is defined as follows:
  \[ \tau \models_p [\iota] \, \varphi \quad \Longleftrightarrow \quad
    \mathbf{P}( \varphi(\ATRAJ{}) \ |\ T = \tau) \,\geq\, p \]
\end{definition}

\subsection{Sampling counterfactual traces}

Counterfactual traces can be sampled using the algorithm described in
Listing~\ref{alg:cosimulation}.

\begin{algorithm}
\caption{Counterfactual resimulation}\label{alg:cosimulation}
\begin{spacing}{1.3}
\algsetup{indent=1.5em}
\begin{algorithmic}[1]
\vspace{0.2cm}
\STATE $t \gets 0$
\STATE $m \gets\ $ initial mixture
\WHILE{ $t < t_\text{\,end}$ }
  \STATE $m_0 \gets \TSTATE{t}{\tau}$
  \STATE $(e_{\text{f}}, t_{\text{f}}) \gets $ first event of $\tau$ in time interval $(t, \infty)$
  \vspace{0.1cm}
  \STATE $\alpha' \gets \sum_r {\lambda_r |\DEMBS{r}{m, m_0}|}$
  \vspace{0.1cm}
  \STATE draw $\delta \sim \textsc{Exp}(\alpha') $
  \STATE $t_{\text{c}} \gets t + \delta$
  % \STATE
  \IF { $t_{\text{c}} < t_{\text{f}}$ }
      \STATE draw a rule $r$ with prob.
      $\propto \, \lambda_r |\DEMBS{r}{m, m_0}|$
      \STATE  draw a divergent embedding $\xi \in \DEMBS{r}{m, m_0}$
      \STATE {$e \gets (r, \xi)$}
      \IF {$ \neg \, \BLOCKED{\iota}{e, t_{\text{c}}} $ }
          \STATE $t \gets t_{\text{c}}$
          \STATE $m \gets \UPDATE{m}{e}$
          \STATE log event $(e, t)$
      \ENDIF
  \ELSE
      \STATE $t \gets t_{\text{f}}$
      \IF {$ \neg \, \BLOCKED{\iota}{e, t} $ \AND $ \TRIGGERABLE{m}{e} $ }
          \STATE $m \gets \UPDATE{m}{e}$
          \STATE log event $(e, t)$
      
      \ENDIF 
  \ENDIF
\ENDWHILE
\vspace{0.1cm}
\end{algorithmic}
\end{spacing}
\end{algorithm}
\section{Counterfactual simulation}\label{sec:counterfactual}

Counterfactual statements are tricky because their truth is
context-dependent. The statement ``Had it not rained, the driver might
have arrived earlier" can fail to be true in many ways. Intuitively,
how easily the driver could have arrived earlier depends on how great
a departure from actuality is required for it to be case
\cite{Lewis1973}. This is why counterfactual reasoning is tied to
modal logic. The standard approach is to require that the consequent
in a counterfactual be true in some of those possible worlds (in which
the antecedent holds) that \textit{are most similar to the actual
  world}. If the counterfactual statement is true in all of these
worlds, we can replace ``might" with ``would".  We now operationalize
this approach in the context of Kappa traces and interpolate between
``might" and ``would" using probabilities.

\subsection{A semantics for counterfactuals}

We start by formalizing the notion of an intervention. An intervention
$\iota$ (``blocking $pk$" in our example) is a predicate
$\BLOCKED{\iota}{t, e}$ that determines whether or not event $e$ is
blocked at time $t$. Given a predicate $\varphi$ over traces, we write
the proposition \textit{``Had intervention $\iota$ happened in trace
  $\tau$, $\varphi$ would have been true with probability greater than
  $p \in [0,1]$''} as:
\[ \tau \models_p [\iota] \, \varphi.
\]

To give an operational meaning to this statement, we invoke the
continuous time Markov chain (CTMC) semantics of a Kappa model as
defined and implemented in
\cite{DanosEtAl-APLAS07,BoutillierEK17}. For the present purpose it is
conceptually useful to think of a CTMC abstractly in terms of the
random realization of ``potential events". A potential event is a pair
$(r, \xi)$ where $r$ is a rule and $\xi$ an injective mapping from
local agents involved in $r$ to global agents in a huge virtual
mixture of many instances of all possible molecular
species.\footnote{For the sake of simplicity, we assume that no agent
  is created or deleted by a rule.} For every such potential event, we
imagine a bell that rings at a time $t$ drawn from an exponential
distribution $\lambda_r\exp(-\lambda_r t)$, where $\lambda_r$ is the
stochastic rate constant of $r$. A simulation trace can be viewed as
the realization of a random variable $T$ determined by the set
$\omega$ of ring times: Starting with an initial mixture, when a bell
rings at $t$, its associated potential event $(r, \xi)$ transforms the
mixture according to $r$ if $\xi$ yields a valid embedding of the left
hand side of $r$ in the current mixture and time advances by
$t$. Otherwise, time advances and nothing happens---a null
event. Repeat on the resulting mixture.

We can extend this viewpoint to include interventions. For an
intervention $\iota$, we define the random variable $\ATRAJ{}$ much in
the same way as $T$, except that each time the bell rings, we require
$\BLOCKED{\iota}{t, e}$ to be false for the potential event
$e=(r, \xi)$ to be considered.  Counterfactual traces that are closest
to the actual trace $\tau$ are then sampled by generating realizations
of $\ATRAJ{}$ that inherit, whenever possible, the subset of $\omega$
that made up $\tau$. 
% An efficient implementation of this specification
% for sampling the conditional random variable $\CTRAJ{}$ is available at
% \begin{center}
%   \url{https://github.com/jonathan-laurent/kappa-counterfactuals}.
% \end{center} 
We refer to this natural extension of CTMC semantics as
\textit{counterfactual re-simulation} or \textit{co-simulation} for
short. Using co-simulation, we can operationalize the counterfactual
statements as follows.

\begin{definition}[Semantics of counterfactual statements] We write
  $\tau \models_p [\iota] \, \varphi$ the counterfactual statement
  \textit{``had intervention $\iota$ happened in trace $\tau$,
    predicate $\varphi$ would have been true with probability greater
    than $p$''}.  It is defined as follows:
  \[ \tau \models_p [\iota] \, \varphi \quad \Longleftrightarrow \quad
    \mathbf{P}( \varphi(\ATRAJ{}) \ |\ T = \tau) \,\geq\, p \]
\end{definition}

\subsection{Sampling counterfactual traces}

Counterfactual traces can be sampled using the algorithm described in
Listing~\ref{alg:cosimulation}.

\input{algos/cosimulation}
\input{proofs/cosimulation}