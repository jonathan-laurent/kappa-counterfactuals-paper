\section{Counterfactual simulation}

In this section, we derive a variation of the Gillespie algorithm for
simulating counterfactual traces. That is, given a reference trace
$\tau$ and an intervention $\iota$, we show how to simulate an
instance of $\CTRAJ{}$. Before we proceed, it is useful to refresh our
memory about the original Gillespie algorithm, which is summarized in
Listing~\ref{alg:gillespie}.


In Gillespie's algorithm, the activity of a rule $r$ is defined as the
product $\lambda_r|\EMBS{r}{M}|$ of its reaction rate by the number of
embeddings of its left hand side in the current reaction mixture.
Then, simulating a trace works by repeating the following steps:
\begin{inparaenum}[1)]
\item draw the time before the next simulation event from an
  exponential distribution of parameter the total activity $\alpha$ of
  the rules and increment the current time by this amount
\item draw a rule $r$ with probability proportional to its activity
\item pick an instance of the left hand side of $r$ uniformly in the
  current mixture and rewrite it.
\end{inparaenum}
A key property of Kappa's CTMC semantics that can be used to establish
the corectness of this algorithm is the following.
%
%
\begin{proposition}\label{prop:gillespie}
  Let $I=[t,\; t+\delta]$ a time interval and $m$ a mixture. Then,
  \[ \CProb{ T \cap I = \emptyset }{ \TSTATE{t}{T} = m } =
    e^{-\alpha(m) \cdot \delta} \] where $\alpha(m)$ is the total
  activity of mixture $m$.
\end{proposition}
%
%
% \input{proofs/gillespie-pty}
A similar theorem can be proved for counterfactual traces, involving a
modified notion of activity we call \emph{divergent activity}.
Indeed, suppose we are simulating a counterfactual trace and the
current mixture at time $t$ is $m$. Let $m_0$ the state of the
reference trace at time $t$. We say that a site is divergent if it has
different states in $m$ and $m_0$.  Besides, an embedding of the left
hand side of a rule $r$ into $m$ that features at least one divergent
site is said to be divergent. We write $\DEMBS{r}{m, m_0}$ the set of
all such embeddings.  Finally, we define the divergent activity of a
rule as the product of its rate by the number of divergent embeddings
of its left hand side into $m$ and write $\alpha'(m, m_0)$ the total
divergent activity:
\[\alpha'(m, m_0) = \sum_r \lambda_r |\DEMBS{r}{m, m_0}|. \]
Then, we can state the counterpart of Proposition~\ref{prop:gillespie}
for counterfactual simulation.
\begin{proposition}\label{prop:cosim-waiting}
  Let $\tau$ a trace and $\iota$ an intervention.  Let $I$ a time
  interval of width $\delta$ such that $\tau \cap I =
  \emptyset$. Then,
  \[
    \CProb{ \hat T_\iota \cap I = \emptyset } { T=\tau,\
      \TSTATE{t}{\hat T_\iota} = m\ } \,=\ e^{-\alpha'(m, m_0) \cdot
      \delta}
  \]
  where $m_0 = \TSTATE{t}{T}$ and $\alpha'(m, m_0)$ is the divergent
  activity of $m$ in respect to $m_0$.
\end{proposition}

% As summarized on Listing~\ref{alg:gillespie}, simulating a trace
% using Gillespie's algorithm works by repeating the following steps,
% starting from the initial mixture:

\begin{algorithm}
\caption{Doob-Gillespie algorithm}\label{alg:gillespie}
\begin{spacing}{1.2}
\begin{algorithmic}
\vspace{0.2cm}
  \STATE $t \gets 0$
  \STATE $m \gets\ $ initial mixture
  \vspace{0.1cm}
  \WHILE{ $t < t_\text{\,end}$ }
      \vspace{0.1cm}
      \STATE $\alpha \gets \sum_r {\lambda_r |\EMBS{r}{m}|}$
      \vspace{0.1cm}
      \STATE draw $\delta \sim \textsc{Exp}(\alpha) $
      \STATE $t \gets t + \delta$
      \STATE draw a rule $r$ with probability
      $\propto \ \lambda_r |\EMBS{r}{m}|$
      \STATE draw an embedding $\xi$ uniformly in $\EMBS{r}{m}$
      %\STATE update $m$ by triggering event $((r, \xi), t)$
      \STATE $m \gets \UPDATE{m}{(r, \xi)}$
      \STATE log event $((r, \xi), t)$
  \ENDWHILE
\vspace{0.1cm}
\end{algorithmic}
\end{spacing}
\end{algorithm} \newcommand{\EVF}[0]{e_{\text{f}}}
\newcommand{\EVCF}[0]{e_{\text{c}}}

\begin{algorithm}
\caption{Counterfactual resimulation}\label{alg:cosimulation}
\begin{spacing}{1.3}
\algsetup{indent=1.5em}
\begin{algorithmic}[1]
\vspace{0.2cm}
\STATE $t \gets 0$
\STATE $m \gets\ $ initial mixture
\WHILE{ $t < t_\text{\,end}$ }
  \STATE $m_0 \gets \TSTATE{t}{\tau}$
  \STATE $(\EVF{}, t_{\text{f}}) \gets $ first event of $\tau$ in time interval $(t, \infty)$
  \vspace{0.1cm}
  \STATE $\alpha' \gets \sum_r {\lambda_r |\DEMBS{r}{m, m_0}|}$
  \vspace{0.1cm}
  \STATE draw $\delta \sim \textsc{Exp}(\alpha') $
  \STATE $t_{\text{c}} \gets t + \delta$
  % \STATE
  \IF { $t_{\text{c}} < t_{\text{f}}$ }
      \STATE draw a rule $r$ with prob.
      $\propto \, \lambda_r |\DEMBS{r}{m, m_0}|$
      \STATE  draw a divergent embedding $\xi \in \DEMBS{r}{m, m_0}$
      \STATE {$e \gets (r, \xi)$}
      \IF {$ \neg \, \BLOCKED{\iota}{e, t_{\text{c}}} $ }
          \STATE $t \gets t_{\text{c}}$
          \STATE $m \gets \UPDATE{m}{e}$
          \STATE log event $(e, t)$
      \ENDIF
  \ELSE
      \STATE {$e \gets \EVF{}$}
      \STATE $t \gets t_{\text{f}}$
      \IF {$ \neg \, \BLOCKED{\iota}{e, t} $ \AND $ \TRIGGERABLE{m}{e} $ }
          \STATE $m \gets \UPDATE{m}{e}$
          \STATE log event $(e, t)$
      
      \ENDIF 
  \ENDIF
\ENDWHILE
\vspace{0.1cm}
\end{algorithmic}
\end{spacing}
\end{algorithm}


% \begin{theorem}[Corectness of resimulation]
%   Let $\tau$ a trace and $\iota$ an intervention.  We write
%   $R_\iota(\tau)$ the random variable defined by the resimulation
%   algorithm. Then, ${\hat T}_\iota \,|\, \{T = \tau\}$ and
%   $R_\iota(\tau)$ have the same probability law.
% \end{theorem}


\subsection{Deriving the cosimulation algorithm}

Suppose you are simulating a counterfactual trace for intervention
$\iota$ and reference trace $\tau$, that is, you are drawing an
instance of $\hat T_\iota \,|\, \{T=\tau\}$. Suppose you have already
simulated the first $t$ seconds of this counterfactual trace and you
want to know the time of the next event. Because both $T$,
$\hat T_\iota$ and $\Sigma$ are markovian, the only only relevant
information about what has already been simulated is the current state
of the counterfactual world $s := \TSTATE{t}{\hat T_\iota}$.


If we write $\delta$ the time difference between $t$ and the next
event after $t$ in $\tau$, we are interested in the following
probability:
\begin{equation}
  \CProb{ \hat T_\iota \cap [t, t+\delta) = \emptyset  }
  % { \underbrace{T=\tau, M = \mathcal{S}_t(\hat T_\iota)}_{\Delta} }
  { T=\tau,\, s = \TSTATE{t}{\hat T_\iota} }
\end{equation}
Writing $I = [t, t+\delta)$, this quantity can be rewritten as
follows:
\begin{align}
  \ & \ \CProb{ \hat T_\iota \cap I = \emptyset }
      { T=\tau,\, s = \TSTATE{t}{\hat T_\iota} } \\[0.5em]
  = & \ \CProb{ \bigwedge_{s \vdash e} (\,e \notin \Sigma \cap I \,)  }{ T=\tau } \\[1em]
  = & \ \prod_{s \vdash e}\, \CProb{e \notin \Sigma \cap I }{ T=\tau }
\end{align}
Indeed, given that $s = \TSTATE{t}{\hat T_\iota}$, no counterfactual
event happens in time interval $I$ if and only if no potential event
that is triggerable from state $s$ is scheduled in $I$.  Besides,
potential events are scheduled independently and so we can decompose
the resulting probability as a product.

Let $e$ a potential event such that $s \vdash e$. The probability that
$e$ has not been scheduled in $I$ given that $T=\tau$ depends on
whether or not $e$ is triggerable from state $s_0 :=
\TSTATE{t}{T}$. Indeed, we assumed that $\tau$ contains no event in
time interval $I$. Therefore, if $s_0 \vdash e$, then $e$ cannot be
scheduled in $I$, without which it would have been observed in $\tau$.
Thus,
\[ s_0 \vdash e \ \Rightarrow\ \CProb{e \notin \Sigma \cap I}{ T=\tau
  } = 1 \] Besides, if $s_0 \not\vdash e$, then the observation
$\{ T=\tau \}$ gives absolutely no information on whether or not $e$
has been scheduled in $I$ and so
\[ s_0 \not\vdash e \ \Rightarrow\ \CProb{e \notin \Sigma \cap I}{
    T=\tau } = e^{-\lambda_e \cdot \delta} \] as the scheduling
process of a potential event is a Poisson process. Combining these two
results with (4), we have:
\begin{align}
  \CProb{ \hat T_\iota \cap I = \emptyset }{ \Delta }
  =\ \prod_{s \vdash e, \, s_0 \not\vdash e} e^{-\lambda_e} = e^{-\alpha'(s, s_0)\cdot\delta}
\end{align}
where
\[\alpha'(s, s_0) := \sum_{s \vdash e, \, s_0 \not\vdash e}
  \lambda_e \] This quantity can be rewritten as follows:
\begin{align}
  \alpha'(s, s_0) &= 
                    \sum_{(r, \xi)} \textbf{1}\{ s \vdash (r, \xi),\ s_o \not\vdash (r, \xi) \}\cdot\lambda_r \\
                  &= \sum_r \lambda_r\sum_\xi \textbf{1}\{ s \vdash (r, \xi),\ s_o \not\vdash (r, \xi) \} \\
                  &= \sum_r \lambda_r |\DEMBS{r}{s, s_0}|
\end{align}
which gives us the definition of divergent activity. This result can
be summarized in the following theorem:
\begin{theorem} Let $\tau$ a trace and $\iota$ an intervention. Let
  $I$ a time interval of width $\delta$ such that
  $\tau \cap I = \emptyset$. Then,
  \[\CProb{ \hat T_\iota \cap I = \emptyset }{ T=\tau,\
      \TSTATE{t}{\hat T_\iota} = s\ }
    \ =\ e^{-\alpha'(s, s_0) \cdot \delta}
  \]
  where $s_0 = \mathcal{S}_t(\tau)$ and
  \[\alpha'(s, s_0) = \sum_r \lambda_r |\DEMBS{r}{s, s_0}| \]
  is the total divergent activity of $s$ with respect to $s_0$.
\end{theorem}
During normal simulation of a Kappa model, the activity of the
reaction mixture can be interpreted as its propensity to change. In
contrast, during counterfactual simulation, the divergent activity can
be interpreted as the propensity of the counterfactual mixture to
diverge from the reference mixture.






% $M_0 := \mathcal{S}_t(\hat \tau)$
% = \CProb{ \bigwedge_e (\,e \notin \hat T_\iota \cap [t, t+\delta)\,)  }{ \Delta } \\
