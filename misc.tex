\section*{Other stuff}

Suppose you are simulating a counterfactual trace for intervention
$\iota$ and reference trace $\tau$, that is, you are drawing an
instance of $\CTRAJ{}$. Suppose you have already simulated the first
$t$ seconds of this counterfactual trace and you want to know the time
of the next event. Because both $T$, $\ATRAJ{}$ and $\Sigma$ are
markovian, the only only relevant information about what has already
been simulated is the current state of the counterfactual world
$s := \TSTATE{t}{\ATRAJ{}}$.

At some point, we have to define
\[\EMBS{r}{m} = \{ \xi \,:\, m \vdash (r, \xi) \}\]
and notice that
\[\DEMBS{r}{m, m_0} = \EMBS{r}{m} \setminus \EMBS{r}{m_0}.\]


\begin{definition}[Divergent embedding]
  Let $m$ a mixture and $m_0$ a reference mixture. A site $s$ is said
  to be \emph{divergent} if it is in a different state in $m$ and in
  $m_0$.  Then, an embedding $\xi$ of a rule $r$ into $m$ is said to
  be \emph{divergent} if it features a divergent site. We write
  $\DEMBS{r}{m, m_0}$ the set of such embeddings.
\end{definition}

\begin{definition}[Divergent activity]
  The \emph{divergent activity} of a rule $r$ is defined as the
  product of its reaction rate $\lambda_r$ by the cardinal of
  $\DEMBS{r}{m, m_0}$.  The total divergent activity of the system
  $\alpha'(m, m_0)$ is defined as the sum of the divergent activity of
  every rule:
  \[\alpha'(m, m_0) = \sum_r \lambda_r |\DEMBS{r}{m, m_0}| \]
\end{definition}

\subsection*{On co-simulation}

Indeed, suppose we are simulating a counterfactual trace and the
current mixture at time $t$ is $m$. Let $m_0$ the state of the
reference trace at time $t$. We say that a site is divergent if it has
different states in $m$ and $m_0$.  Besides, an embedding of the left
hand side of a rule $r$ into $m$ that features at least one divergent
site is said to be divergent. We write $\DEMBS{r}{m, m_0}$ the set of
all such embeddings.  Finally, we define the divergent activity of a
rule as the product of its rate by the number of divergent embeddings
of its left hand side into $m$ and write $\alpha'(m, m_0)$ the total
divergent activity:
\[\alpha'(m, m_0) = \sum_r \lambda_r |\DEMBS{r}{m, m_0}|. \]


During normal simulation of a Kappa model, the activity of the
reaction mixture can be interpreted as its propensity to change. In
contrast, during counterfactual simulation, the divergent activity can
be interpreted as the propensity of the counterfactual mixture to
diverge from the reference mixture.

\paragraph{Other concepts and notations}



\begin{itemize}
\item 

\item Writing $L_r$ and $R_r$ the left hand side and right hand side of a rule.

\item An \emph{embedding} of the left hand side $L_r$ of a rule $r$ to
  a mixture $m$ is a map $\xi$ from agents of $L_r$ to global agent
  identifiers such that\dots
\end{itemize}

Here, the result was simply $(b, p)$. Then, we computed activation
arrows within this sub-trace using the following rule: there is an
arrow between event $e$ and event $e'$ if
% \begin{inparaenum}[i)]
% \item $e$ happens before $e'$ and
% \item $e'$ tests a site $s$ such that $e$ is the last event before
%   $e'$ to modify $s$.
% \end{inparaenum}
$e$ is the last event before $e'$ that modifies some site to the state
it is tested for by $e'$. In general, this results in a \textsc{dag}
where nodes are events and edges are activation arrows.