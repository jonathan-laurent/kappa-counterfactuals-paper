\section*{Other stuff}

Suppose you are simulating a counterfactual trace for intervention
$\iota$ and reference trace $\tau$, that is, you are drawing an
instance of $\CTRAJ{}$. Suppose you have already simulated the first
$t$ seconds of this counterfactual trace and you want to know the time
of the next event. Because both $T$, $\ATRAJ{}$ and $\Sigma$ are
markovian, the only only relevant information about what has already
been simulated is the current state of the counterfactual world
$s := \TSTATE{t}{\ATRAJ{}}$.

During normal simulation of a Kappa model, the activity of the
reaction mixture can be interpreted as its propensity to change. In
contrast, during counterfactual simulation, the divergent activity can
be interpreted as the propensity of the counterfactual mixture to
diverge from the reference mixture.

The reason why some nodes appear thicker than others is discussed in
section~\ref{subsec:cex}.


In this reasoning, we are following the
long-standing insight \cite{Lewis1973,pearl2009causality} that
counterfactual statements refer to worlds that are closest to
the actual one
