\section{Motivating Example}

We illustrate the need for counterfactual reasoning on a toy example in Kappa. 
\TODO{Say something about what Kappa is?} 
\TODO{Say some more sentences about how this section is going to go.}

\subsection{A Kappa Model with Inhibition}
Consider a model with two types of agents, kinases $K$ and substrates $S$, 
interacting according to the rules depicted in Figure~\ref{fig:model}.

\TODO{Explain how the rules work.}
\TODO{Explain the dynamic simulation and the static analysis.}

\subsection{Where Existing Causal Analysis Falls Short}
For the sake of simplicity, consider an initial mixture $I$ with only a 
single kinase and a single substrate whose sites are free and unphosphorylated. 
We then ask: Starting from $I$, \textbf{how} is $p$ triggered ? 
We are not merely looking for an account of reachability but rather for 
causal narratives, that is, collections of necessary events connected by 
causal influences.

A stochastic simulation \cite{DanosEtAl-APLAS07} might produce
the following trace (events are labelled by the rules that induced them):
\[b,\ \  u,\ \  pK,\ \  b,\ \  p,\ \  u^{*},\ \  \cdots\]

Figure~\ref{fig:dumb-story} depicts the causal narrative explaining 
the occurrence of $p$ according to existing techniques
\cite{DBLP:conf/fsttcs/DanosFFHH12,DanosEtAl-CONCUR07}. 
The arrow between $b$ and $p$ is called an \textit{activation arrow}, 
meaning that $b$ modifies an aspect of state (by creating a link) that 
enables $p$ to happen.


This narrative, however, is blind to the critical role of $pK$ in the original 
trace. Looking at the rules in Figure~\ref{fig:model} one notes that:
\begin{inparaenum}[(i)]
\item the phosphorylation rule $p$ is slow
\item the average time $K$ and $S$ remain bound
depends on whether $K$ is phosphorylated, as manifest
in the two unbinding rules $u$ (fast, if $K$ is not phosphorylated) 
and $u^{*}$ (slow, if $K$ is phosphorylated).
\end{inparaenum}
It seems reasonable to assert that $p$ would probably not have happened had 
$pK$ not happened, as the opportunity for $p$ would have been cut short by a 
fast unbinding event. We therefore argue that $pK$, although it does not 
activate $b$ or $p$ directly, should be part of a causal narrative for $p$. 
Reasoning of this kind is \textit{counterfactual} and can be deployed 
to define causality \cite{lewis1974causation, lewis2000causation}.

\TODO{Talk at a high level about the challenges involved in doing this.}